\documentclass[12pt, oneside]{book}

\usepackage[T2A]{fontenc}			% кодировка
\usepackage[utf8]{inputenc}			% кодировка исходного текста
\usepackage[english,russian]{babel}	% локализация и переносы

\usepackage{amsmath}

\usepackage{amsthm, mathrsfs, mathtools, amssymb}

\usepackage{geometry}
\geometry{verbose,a4paper,tmargin=2cm,bmargin=2cm,lmargin=2.5cm,rmargin=1.5cm}

\usepackage{showframe}

\author{Кортенко~A.~М.}
\title{Конспект курса}
\date{\today}

\newtheorem{definition}{Определение}[section]
\newtheorem{theorem}{Теорема}[section]
\newtheorem{ex}{Пример}[section]
\newtheorem*{remark}{Замечание}

\newcommand{\toP}{\xrightarrow[]{P}} % сходимость по вероятности
\newcommand{\toPN}{\xrightarrow[]{\text{п.н.}}} % сходимость почти наверное

\begin{document}
  %\pagestyle{plain}

  \chapter{Лекция 1 - 2023-09-06 - Основные распределения МС}
\section{Гамма-распределение}
\[
  \gamma_{\alpha, \lambda} (x) = \begin{cases}
    \dfrac{\alpha^\lambda}{\Gamma(\lambda)} x^{\lambda-1} e^{-\alpha x}, x>0 \\
    0, x \leqslant 0
  \end{cases}, \lambda > 0, \alpha > 0.
\]

\subsection{Гамма-функция \texorpdfstring{$\Gamma(\lambda)$}{Gamma(lambda)} }
\begin{definition}
  $$\Gamma(\lambda) = \int_0^{+\infty} x^{\lambda-1} e^{-x} \, dx, \lambda > 0.$$
\end{definition}

\begin{itemize}
  \item $\Gamma(1) = \int_0^{+\infty} e^{-x} \, dx = 1$.
  \item $\Gamma(\frac{1}{2}) = \int_0^{+\infty} x^{-1/2} e^{-x} \, dx = 2 \int_0^{+\infty} e^{-x} \, d\sqrt{x} = 2 \int_0^{+\infty} e^{-y^2} \, dy = \sqrt{\pi}$.
  \item $\Gamma(\lambda+1) = \int_0^{+\infty} x^\lambda e^{-x} \, dx = 
    \left. - x^\lambda e^{-x} \right|_0^{+\infty} 
    + \int_0^{+\infty} \lambda x^{\lambda-1} e^{-x} dx 
    = \lambda \Gamma(\lambda)$.

  В частности, $\Gamma(n+1) = n!$. 
\end{itemize}

\begin{remark}
  \[
    \lambda = 1 \Rightarrow \Gamma_{\alpha, 1} = \begin{cases}
      \alpha e^{-\alpha x}, x>0 \\
      0, x\leqslant 0
    \end{cases}
  \]
\end{remark}

\section{Бета распределение}

\begin{definition}
  $\beta_{m, n} = \begin{cases}
    \dfrac{x^{m-1}  (1-x)^{n-1}}{B(m, n)}, 0 < x < 1 \\
    0, x \notin (0, 1)
  \end{cases}, m, n > 0$
\end{definition}

\begin{definition}
  $B(m, n) = \int\limits_0^1 w^{m-1} (1-w)^{n-1} dw, m, n>0$.
\end{definition}

\begin{remark}
  Частные случаи:
  \begin{itemize}
    \item $\beta_{1, 1} = \begin{cases}
        \frac{1}{B(1, 1)} = 1, 0<x<1 \\
        0, x\notin (0, 1)
      \end{cases}$ - равномерное распределение на $(0, 1)$.

    \item $\beta_{\frac{1}{2}, \frac{1}{2}} (x) = \begin{cases}
        \dfrac{1}{\pi \sqrt{x(1-x)}}, 0<x<1 \\
        0, x \notin (0, 1)
      \end{cases}$ - закон арксинуса
  \end{itemize}
\end{remark}

\subsection{Свойства бета-функции}

\begin{itemize}
  \item $B(m, n) = \int\limits_0^\infty \dfrac{u^{m-1}}{(1+u)^{m+n}} \, du$
    \begin{proof}
      \begin{multline*}        
        \int\limits_0^1 w^{m-1} (1-w)^{n-1} \, dw = \left[
                \begin{aligned}
                  w &= \frac{u}{1+u} \\
                  1-w &= \frac{1}{1+u} \\
                  dw &= \frac{1}{(1+u)^2} du
                \end{aligned}
              \right] = \\
              = \int\limits_0^\infty \dfrac{u^{m-1}}{(1+u)^{m-1} (1+u)^{n-1} (1+u)^2} \, du = \int_0^\infty \dfrac{u^{m-1}}{(1+u)^{m+n}} \, du.
      \end{multline*}
    \end{proof}

  \item $B(m, n) = \dfrac{\Gamma(m) \Gamma(n)}{\Gamma(m+n)}$
    \begin{proof}
      \begin{multline*}
        \Gamma(\lambda) = \int\limits_0^{+\infty} x^{\lambda-1} e^{-x} \, dx =
        \left[ \begin{aligned} x &= (1+u)v \\ dx &= (1+u) dv \end{aligned} \right] =
        \int\limits_0^{+\infty} (1+u)^\lambda v^{\lambda-1} e^{-(1+u)v} \, dv \\
        \Leftrightarrow
        \dfrac{\Gamma(m+n)}{(1+u)^{m+n}} = \int\limits_0^{+\infty} v^{m+n-1} e^{-(1+u)v} \, dv \Leftrightarrow %\\
        % \Leftrightarrow
        \dfrac{\Gamma(m+n) u^{m-1}}{(1+u)^{m+n}} = \int\limits_0^{+\infty} v^{m+n-1} u^{m-1} e^{-(1+u)v} \, dv\\
        \text{интегрируя по u: }
        B(m, n) \Gamma(m+n) = \int\limits_0^{+\infty} \int\limits_0^{+\infty} v^{m+n-1} u^{m-1} e^{-(1+u) v} \, dv \, du = \\
        = \int\limits_0^{+\infty} \int\limits_0^{+\infty} v^{m+n-1} u^{m-1} e^{-v} e^{-uv} \, dv \, du =
        \int\limits_0^{+\infty} v^{n-1} e^{-v} \, dv \int\limits_0^{+\infty} (uv)^{m-1} e^{-uv} \, d(uv) = \\
        = \Gamma(n) \Gamma(m)
      \end{multline*}
    \end{proof}

  \item $B(m+1, n+1) = \dfrac{1}{(m+k+1) C_{m+k}^m}$ - следует из предыдущего.
\end{itemize}

\section{Хи-квадрат распределение с n степенями свободы}

\begin{definition}
  \[
    p_{\chi^2(n)} (x) = \begin{cases}
       \dfrac{x^{\frac{n}{2} - 1}}{2^\frac{n}{2} \Gamma(\frac{n}{2})} e^{-\frac{n}{2}}, x>0 \\
       0, x\leqslant 0
    \end{cases}
  \]
\end{definition}

\subsection{Свойства хи-квадрат распределения}

\begin{itemize}
  \item $p_{\chi^2(1)} (x) = \begin{cases}
      \frac{1}{\sqrt{2\pi y}} e^{-\frac{y}{2}}, y>0 \\
      0, y\leqslant 0
    \end{cases}$ - совпадает с распределение квадрата стандартной нормально распределенной случайной величины.
    
    \begin{proof}
    
    \end{proof}

  \item Пусть $\xi_i$ независимы и распределены по стандартному нормальному закону. Тогда случайная величина $\eta = \sum_{i=1}^n \xi_i^2$ распределена по закону хи-квадрат с n степенями свободы.
  
    \begin{proof}
      % TODO proof 
    \end{proof}

  \item $p_{\chi^2(n)} = \gamma_{\frac{1}{2}, \frac{n}{2}} (x)$.
\end{itemize}

\section{Распределение Фишера-Снедекора F(m,n)}

\begin{definition}
  \[
    p_{F(m, n)} (x) = \begin{cases}
      \dfrac{m^\frac{m}{2} n^\frac{n}{2} x^{\frac{m}{2}-1}}{(n+mx)^\frac{m+n}{2} B(\frac{m}{2}, \frac{n}{2}}, x>0 \\
      0, x\leqslant 0
    \end{cases}
  \]
\end{definition}

% TODO left part of lection


  \chapter{Лекция 2 - 2023-09-13}

  \chapter{Лекция 3 - 2023-09-20}
  
  \begin{definition}
     $\widehat{\Theta_n} = f\left(x_1, x_2,\dots, x_n\right)$ - оценка параметра $\Theta$ (функция от выборки, статистика).
  \end{definition}

  %\begin{definition}
  %  Оценка параметра $\widehat{D_n} = f(x_1, x_2, \dots, x_n)$.
  %\end{definition}

  \section{Свойства оценок}
  
  \begin{definition}
    Оценка $\widehat{D_n}$ называется несмещенной, если $M[\widehat{D_n}]$.
  \end{definition}
  
  \[
    M f(\bar{x}) = \int\limits_{R_n} f(x_1, x_2, \dots, x_n) \prod\limits_{j=1}^{n} p_{x_i} (x_i) \, dx_1 dx_2 \dots dx_n.
  \]

  $$f(x_1, \dots, x_n) \toP 0, n \to \infty$$

  Если $f(x_1, \dots, x_n) \toPN 0$, то $\hat{D_n}$ - силнльно состоятельная.

  %Пусть $x_1, \dots, x_n$ - выборка.
  %$M X_i = a$.
  %$$\hat{a_n} = ?$$

  \section{Выборочное среднее}

  \begin{definition}
    Выборочное среднее $\hat{a_n} = \bar{X} = \frac{1}{n} \sum\limits_{i=1}^{n} x_i$.
  \end{definition}

  Свойства выборочного среднего

  1. $M \bar{X} = M(\frac{1}{n} \sum\limits_{i=1}^{n} x_i) = \dots$. Несмещенная

  2. Сильно состаятельная


  \section{Выборочная дисперсия}

  $$ S^2 = \frac{1}{n-1} \sum\limits_{i=1}^{n} \left(x_I-\bar{x}\right)^2 $$

  \begin{multline}
    M S^2 = \frac{1}{n-1} M(\sum_{i=1}^{n} (X_i - a - (\bar{X} - a)^2)^2 ) = \\
    = \frac{1}{n-1} \sum M (X_i-a)^2 - 2 M[ (\bar{X} - a) \sum (x_i - a) ] + M (\bar{x} -a)^2 = \\
    = \frac{1}{n-1} \left[ n \sigma^2 - 2 n M(\bar{X}-a)^2 + n M (\bar{x}-a)^2 \right] = \\
    = \frac{1}{n-1} \left[ n \sigma^2 - n \frac{\sigma^2}{n} \right] = \sigma^2.
  \end{multline}

  2. Состоятельность 

  \begin{multline*}
    S_n^2 = \frac{1}{n-1} \sum (X_i - \bar{X})^2 = \frac{n}{n-1} \left[ \frac{1}{n} \sum X_i^2 - \bar{X}^2 \right] = \\
    \text{В силу сильного ЗБЧ} \\
    \to \sigma^2
  \end{multline*}

  3. Дисперсия

   
  $$D S^2 = \dfrac{\mu_4}{n} - \dfrac{\sigma^4 (n-3)}{n (n-1)} = O \left(\frac{1}{n} \right)$$
  $\mu_4 = M X_i^4$ - четвертый момент

  4. Теорема достаточное условие состоятельности

  Пусть $\hat{\Theta_n}$ асимптотически несмешщенная оценка $\Theta$ и $D \bar{\Theta_n} \to 0, n \to \infty$, тогда $\hat{\Theta_n}$ - состоятельная оценка $\Theta$.

  Доказательство
  \begin{multline*}
    \forall \epsilon > 0: | \hat{\Theta_n} - \Theta| = |\Theta_n - M \Theta_n + M \Theta_n - \Theta| <= |\Theta_n - M\Theta_n| + | \Theta - M \Theta_n | \\
    P( | \Theta_n - M \Theta_n| > \epsilon) \leqslant \dfrac{|\Theta_n - M \Theta_n| ^2} {\epsilon^2} =  \dfrac{D \Theta_n} {\epsilon^2} \to 0, n \to \infty
  \end{multline*}

  3. Выборочная функция распределения

  \[
    I(x) = 1, x>0, I(x) = 0 = x<=0.
  \]

  Индикатор непрерывен слева.

  Определение выборочная функция распределения $\hat{F_n} (x) = \frac{1}{n} \sum I(x-X_i)$

  Свойства ВФР

  1. Несмещенность
  \begin{multline*}
    M \hat{F_n} (x) = M \dfrac{1}{n} \sum I(x-X_i)
  \end{multline*}

  2. сильно состоянельна

  3.  Теорема гаовенко-канослли

  \begin{multline*}
    sup_x |\hat{F_n} (x) - F(x)| \toPN 0.
  \end{multline*}

  4. Неравенство Дворецного-Кифери-Волфовица

  $$P(sup_x |\hat{F_n}(x) - F(x)| > \epsilon) \leqslant 2 e^{-2n \epsilon^2}$$

  Следствие. $P\left(sup_x |\hat{F_n} (x) - F(x)| <= \epsilon\right) >= 1 - 2 e^{-2n \epsilon^2} = 1-\alpha$.
  % $\epsilon = \sqrt{ \dfrac{ln \frac{2}{\alpha} }}{2n}$.

  $$\hat F_n$$

  5. $$0 \leqslant \hat F_n (x) \leqslant 1$$

  $$\lim_{x\to -\infty} \hat{F_n} (x) = 0$$
  
  $$F_n(+\infty) = 1$$

  $F_n(x)$ неубывает
  
  Теорема. Если выборка $x_1, \dots, x_n$ получена из закона распределения $F(x)$, то $F_n(x)$ - .. СВ

  \begin{multline*}
    P(F_n(x) = k/n) = C_n^k F(x)^k (1-F(x))^{n-k}
  \end{multline*}

  схема бернулли

  Доказательство. 
  \[
    \hat{F_n} \left(x\right) = \frac{1}{n} \sum I\left(x-X_i\right)
  \]

  Следствие
  1. $M \hat F_n (x) = \frac{1}{n} np = p = F(x)$
  2. $D \hat F_n (x) = \frac{1}{n^2} npq = \dfrac{F(X) (1 - F(x))}{n}$
  $\Leftrightarrow M (\hat F_n(x) - F(x))^2 \to 0$
  $\Leftrightarrow \hat F_n (x) \to^\text{ср.кв.} F(x)$

  4. Порядковая статистика

  $x_1, x_2, \dots, x_n$

  \begin{definition}
    Выборка упорядоченная по возрастанию называется вариационным рядом $X_{(1)} \leqslant \dots X_{(n)}$.
  \end{definition}

  $min (X) = X_{(1)}$ - минимальный член вариационного ряда.

  $\omega = X_{(n)} - X_{(1)}$ - размах выборки

\end{document}
