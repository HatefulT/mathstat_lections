\documentclass[12pt, oneside]{book}

\usepackage[T2A]{fontenc}			% кодировка
\usepackage[utf8]{inputenc}			% кодировка исходного текста
\usepackage[english,russian]{babel}	% локализация и переносы

\usepackage{amsmath}

\usepackage{amsthm, mathrsfs, mathtools, amssymb}

\usepackage{geometry}
\geometry{verbose,a4paper,tmargin=2cm,bmargin=2cm,lmargin=2.5cm,rmargin=1.5cm}

%\usepackage{showframe}

\usepackage[colorlinks=true, linkcolor=black, urlcolor=black]{hyperref}
\newtheoremstyle{example}% name
{0.7cm}% Space above
{0.7cm}% Space below
{\small}% Body font
{}% Indent amount
{\small\scshape}% Theorem head font
{.}% Punctuation after theorem head
{.5em}% Space after theorem head
{}% Theorem head spec (can be left empty, meaning ‘normal’)
\theoremstyle{example}
\newtheorem{ex}{Пример}
\numberwithin{ex}{chapter}

\theoremstyle{plain}
\newtheorem{theorem}{Теорема}
\newenvironment{thmbis}[1]
  {\renewcommand{\thetheorem}{\ref{#1}$'$}%
   \addtocounter{thm}{-1}%
   \begin{theorem}}
  {\end{theorem}}
\newtheorem{corollary}{Следствие}
\newtheorem*{corollary*}{Следствие}
\newtheorem{lemma}{Лемма}
\newtheorem{utv}{Утверждение}
\newtheorem*{utv*}{Утверждение}

\theoremstyle{definition}
\newtheorem{definition}{Определение}
\newenvironment{dfnbis}[1]
  {\renewcommand{\thedefinition}{\ref{#1}$'$}%
   \addtocounter{definition}{-1}%
   \begin{definition}}
  {\end{definition}}
\newtheorem*{definition*}{Определение}
\newtheorem{question}{Вопрос}

\theoremstyle{remark}
\newtheorem{remark}{Замечание}
\newtheorem*{remark*}{Замечание}
\numberwithin{remark}{section}

\author{Кортенко~A.~М.}
\title{Конспект курса}
\date{\today}

\newcommand{\toP}{\xrightarrow[]{\mathsf P}} % сходимость по вероятности
\newcommand{\toPN}{\xrightarrow[]{\text{п.н.}}} % сходимость почти наверное
\newcommand{\toD}{\xrightarrow[]{d}} % сходимость по распределению (слабая)
\newcommand{\toR}{\xrightarrow[]{r}} % сходимость по распределению (слабая)

\begin{document}
  \pagestyle{plain}

  \tableofcontents

  \section{Условное математическое ожидание относительно $\sigma$-алгебры и его свойства}
На данный момент мы знаем, что 
\begin{multline*}
  M(\eta \mid \xi) \equiv M(\eta \mid \xi = x) \Big|_{x = \xi}
  = \int\limits_{-\infty}^{+\infty} y p_{\eta} (y \mid \xi = x) \, dy \bigg|_{x = \xi}
  = \\ =
  \int\limits_{-\infty}^{+\infty} y \dfrac{p_{\xi\eta} (x, y)}{p_\xi (x)} \,
  dy\biggl|_{x=\xi} = \frac{\int\limits_{-\infty}^{\infty}y p_{\xi\eta}(x,
  y)\,dy}{\int\limits_{-\infty}^{\infty}p_{\xi\eta}(x, y)\,dy}\Biggl|_{x=\xi},
\end{multline*}
где $p_{\xi\eta} (x, y)$ --- известная плотность.

\paragraph{Частные случаи.} \textsc{Гауссовское распределение.}
Если $(\xi, \eta)$ --- гауссовский вектор, то
\begin{align*}
  \hat{\eta} &= M(\eta \mid \xi) = \varphi(\xi) = M\eta + \dfrac{\cov(\xi,
  \eta)}{D\xi} (\xi-M\xi), \\
    \Delta &= M(\eta - M(\eta\mid\xi))^2 = D\eta (1 - r_{\xi\eta}^2).
\end{align*}

\textsc{Многомерное гауссовское распределение.}
Пусть $(\bar{\xi}, \bar{\eta})$ гауссово. Тогда
\begin{align*}
  \hat{\bar{\eta}} &= M(\bar{\eta} \mid \bar{\xi}) = M\bar{\eta} +
  \Sigma_{\bar{\eta} \bar{\xi}} \Sigma_{\bar{\xi}}^{-1} (\bar{\xi} -
  M\bar{\xi}), \\
  \Delta &= M(\bar{\eta} - M(\bar{\eta} \mid \bar{\xi}))(\bar{\eta} -
  M(\bar{\eta} \mid \bar{\xi}))^{\mathsf T}
  = \ldots = \Sigma_{\bar{\eta}} - \Sigma_{\bar{\eta}\bar{\xi}} \Sigma_{\bar{\xi}}^{-1} \Sigma_{\bar{\xi} \bar{\eta}}.
\end{align*}

\begin{ex}\label{ex-bernoulli-with-random-parameter}
  $\xi \sim \mathscr{R}[0, 1]$; $n$ --- число опытов с вероятностью успеха $\xi$; $\eta
  \sim \text{Bernoulli}(\xi)$ --- число успехов. Требуется найти $ P(\eta = j)
  $, $ j = 0, \ldots, n $.

  Тогда 
  \[
    P(\eta = j \mid \xi = x) = C_n^j x^j (1-x)^{n-j}
    \Rightarrow
    P(\eta = j \mid \xi) = C_n^j \xi^j (1-\xi)^{n-j}. 
    % M(\eta \mid \xi = x) = nx \Rightarrow M(\eta \mid\xi) = n\xi.
  \]
  % Причём интуитивно хотелось бы, чтобы $M(\eta \mid \xi = x) = n\cdot x$.
  % И тогда получим, что $M(\eta \mid \xi) = n \xi$.
  Хочется, чтобы $M\eta = MM(\eta \mid \xi)$.
\end{ex}

\begin{definition}[условная вероятность относительно разбиения]
  Пусть дано конечное разбиение $ \mathscr{D} = \{D_j\} $, $ \sum_j D_j = \Omega
  $. Тогда \textsl{случайная величина}\footnote{Символ зависимости $
  P(A\mid\mathscr{D}) = P(A\mid\mathscr{D})(\omega) $ опускается.} $ P(A \mid \mathscr{D}) = \sum_j P(A\mid
  D_j) I(D_j)$ называется \emph{условной вероятностью события $ A $ относительно
  разбиения $ \mathscr D $}.
\end{definition}

\begin{definition}[условная вероятность относительно СВ]
  Пусть $\xi$ --- простая случайная величина (измеримая функция, принимающая
  конечное множество значений):
  $\xi(\omega) = \sum\limits_{j=1}^k x_j I_{D_j}(\omega)$, где
  $\mathscr{D} = \left\{ D_j \right\} $ -- конечное разбиение пространства элементарных
  исходов ($\sum_j D_j = \Omega$).

Тогда \textsl{случайная величина}
\[
  P(A \mid \xi) = P(A \mid \mathscr D) = \sum\limits_{j=1}^k P(A\mid
  D_j) I_{D_j} = \sum_{j=1}^k P(A\mid \xi = x_j)I_{D_j}
\]
называется \emph{условной вероятностью события $ A $
    относительно случайной
  величины $ \xi $}.
\end{definition}

Каждому разбиению $ \mathscr D $ соответствует порождённая им $
\sigma $-алгебра $ \mathscr A = \sigma(\mathscr D) $. Обратно, по каждой $ \sigma
$-алгебре $ \mathscr A $
восстанавливается единственное разбиение $
\mathscr D$ такое, что $ \sigma(\mathscr D) = \mathscr A $. Иными словами,
имеется взаимно однозначное соответствие между $ \sigma $-алгебрами и
разбиениями.

\begin{definition}[условная вероятность относительно конечной $ \sigma
  $-алгебры]
  \emph{Условной вероятностью события $ A $ относительно $ \sigma $-алгебры $
  \mathscr A $} называется \textsl{случайная величина} $ P(A\mid \mathscr A) =
  P(A\mid \mathscr{D}) $, где разбиение $ \mathscr{D} $ соответствует $ \sigma
  $-алгебре $ \mathscr A $ ($ \sigma(\mathscr D) = \mathscr A $).
\end{definition}

\paragraph{Свойства условной вероятности относительно разбиения.} Заметим, что
  \begin{enumerate}
    \item $\mathscr{D} = \Omega \Rightarrow P(A \mid \mathscr{D}) = P(A\mid \Omega) I_\Omega = P(A)$;
    \item $MP(A \mid \mathscr{D}) = \sum\limits_{j=1}^k P(A \mid D_j) M(I_{D_j}) =
      \sum\limits_{j=1}^k P(A\mid D_j) P(D_j) = P(A)$. 
      Это свойство мы будем пытаться
      сохранить и
      при определении математического ожидания относительно $\sigma$-алгебры.
  \end{enumerate}

% Согласно второму свойству, в примере \ref{ex-bernoulli-with-random-parameter} теперь несложно
% получить, что 

\addtocounter{ex}{-1}
\begin{ex}[продолжение]
\begin{multline*}
  P(\eta = j) = MP(\eta = j \mid \xi) = MC^j_n\xi^j(1-\xi)^{n-j} =
  \int\limits_{-\infty}^{+\infty}g(x) p_{\xi}(x)\,dx = \\
  = \int\limits_{0}^{1}C_n^jx^j(1-x)^{n-j}\,dx = C^j_n
  \int\limits_{0}^{1}x^j(1-x)^{n-j}\,dx
  = C^j_n B(j+1, n-j+1) = \\
  = \frac{n!}{j!(n-j)!} \cdot
  \frac{\Gamma(j+1)}{\Gamma(n+2)} = \frac{n!}{j!(n-j)!}\cdot
  \frac{j!(n-j)!}{(n+1)!} = \frac{1}{n+1}.
\end{multline*}
%TODO: чего-то не хватает там где гамма-функции
\end{ex}

Тогда  
\[
  M\eta = \sum_{j=0}^n j P(\eta = j) = \frac{1}{n+1}\sum_{j=0}^n j =
  \frac{1}{n+1} \cdot \frac{n(n+1)}{2} = \frac{n}{2}.
\]



% \begin{definition}[УМО относительно разбиения]
%   Пусть $\mathscr{D} = \left\{ D_j \right\} $ --- конечное разбиение
%   пространства элементарных
%   исходов ($\sum_j D_j = \Omega$). 
% \end{definition}

% \begin{definition}[УМО относительно конечной алгебры]
%   Пусть $\mathscr{A}$ --- конечная алгебра. Тогда, согласно теореме из курса
%   функ. анализа, 
%   она порождается конечным разбиением $\mathscr{D}$. Тогда 
% \end{definition}

\begin{definition}[УМО относительно случайной величины]
  Пусть теперь $\xi$ и $\eta$ --- простые случайные величины:
  $\xi = \sum_j x_j I_{D_j}$, $\eta = \sum_i y_i I_{A_i}$.
  Тогда УМО равно
  \begin{multline*}
    M(\eta \mid \xi) = \sum_j \sum_i y_i P(A_i \mid D_j) I_{D_j} = \\ =
    \sum_j \sum_i y_i M(I_{A_i} \mid D_j) I_{D_j} = 
    \sum_j M \bigg( \sum_i y_i I_{A_i} \, \Big| \, D_j \bigg) I_{D_j}
    = \sum_j M(\eta \mid D_j) I_{D_j}.
  \end{multline*}
  Аналогично, меняя порядок суммирования, получим $ M(\eta\mid\xi)= \sum_i y_i
  P(A_i \mid \mathscr{D}) $.
\end{definition}

\paragraph{Свойства.}
\begin{enumerate}
  \item \textsc{Линейность}. $M(\alpha_1 \eta_1 + \alpha_2 \eta_2 \mid \mathscr{A}) =
      \alpha_1 M(\eta_1 \mid \mathscr{A}) + \alpha_2 M(\eta_2 \mid \mathscr{A})$;
  \item $M(C \mid \mathscr{A}) = C$ ($ C $ --- константа);
  \item $MM(\eta\mid  \mathscr{A}) = M\eta$.
    \begin{proof}
      Пусть $\mathscr{A}$ порождена разбиением $\mathscr{D} = D_1 + D_2 + \dots
      + D_k$. Тогда
      \[
        MM(\eta \mid  \mathscr{A}) = M \sum_i y_i P(\eta = y_i \mid
        \mathscr{A}) = \sum_{i=1}^l y_i MP(\eta = y_i \mid \mathscr{A}) =
        \sum_{i=1}^l y_i P(\eta=y_i) = M\eta.
      \]
    \end{proof}

  \item \begin{definition}
      Говорят, что СВ $\eta$ \emph{измерима относительно $\mathscr{A}$ ($
      \xi, \mathscr{D}$}), если
      $A_{\eta} \subseteq \mathscr{A}$ ($A_\eta \subseteq A_\xi$, $ A_\eta
      \subseteq A_{\mathscr D}) $.
    \end{definition}

    Тогда $M(\eta \mid  \mathscr{A})$ измеримо относительно $\mathscr{A}$:
      $M(\eta \mid  \mathscr{A}) = \sum M(\eta \mid  D_j) I_{D_j}$.

    \item Если $\zeta$ измерима относительно $\mathscr{A}$, то $M(\eta \zeta
      \mid  \mathscr{A}) = \zeta M(\eta \mid  \mathscr{A})$.
    \begin{proof}
      $\xi = \sum_j x_j I_{D_j}$, $\zeta = \sum_s z_s I_{D_s}$, $\eta = \sum_i y_i I_{A_i}$.

      Тогда левая часть:
      \begin{multline*}
        M(\eta\zeta \mid  \mathscr{A}) = M \bigg( \sum_i \sum_s y_i z_s I_{A_i
        D_s} \, \Big| \,  \mathscr{A} \bigg)
        = \sum_i \sum_s y_i z_s M ( I_{A_i D_s} \mid  \mathscr{A} ) = \\
        = \sum_i \sum_s y_i z_s M \bigg( \sum_j M(I_{A_i D_s} \mid  D_j) I_{D_j}
          \bigg) 
        = \sum_i \sum_s y_i z_s P(A_i \mid  D_s) I_{D_s}.
      \end{multline*}

      Правая часть:
      \[
        \zeta M(\eta \mid  \mathscr{A}) = \sum_s z_s I_{D_s} \sum_i y_i \sum_j
        P(A_i \mid  D_j) I_{D_j}
        = \sum_s z_s I_{D_s} \sum_i y_i P(A_i \mid  D_s).
      \]
    \end{proof}

  \item $\mathscr{A}_1 \subseteq \mathscr{A}_2 \Rightarrow M(\eta \mid
    \mathscr{A}_1) = M( M(\eta \mid  \mathscr{A}_2) \mid  \mathscr{A}_1)$;

  \item Если $\eta$ не зависит от $\xi$ (от $\mathscr{A}$), то $M(\eta \mid  \mathscr{A}) = M\eta$.
\end{enumerate}

\begin{theorem}[Радон -- Никодим]\label{theorem-radon-nikodim}
  Для множества, системы подмножеств и меры $(X, \mathscr{A}, \mu)$ назовём
  \emph{зарядом}
  некоторый интеграл $\Phi(B) = \int_B f(x) \mu(dx)$ (можно мыслить себе как новую меру).

  Если $(\mu(B) = 0 \Rightarrow \Phi(B) = 0)$ ($\Phi$ непрерывна относительно
  меры $\mu$),
  то существует такая функция $ \tilde f $, что 
  $\Phi(B) = \int_B \tilde f(x) dx$, где $\tilde f$ --- измеримая относительно
  меры $\mu$.

  Функцию $\tilde f$ также называют производной Радона -- Никодима.
\end{theorem}

\begin{definition}[Общее определение УМО]
  Для вероятностного пространства $(\Omega, \mathscr{A}, P)$ 
  $\Phi(B) = \int_B \eta(\omega) P(d\omega)$, причем имеет место $P(B) = 0
  \Rightarrow \Phi(B) = 0 $. Тогда по теореме \ref{theorem-radon-nikodim}
  существует $\hat{\eta}$ --- измеримая относительно $\mathscr{A}$, такая что
  $\hat{\eta} (\omega) \equiv M(\eta \mid  \xi)$.
\end{definition}

\begin{ex}
  Если $\xi$ и $\eta$ -- простые СВ, то 
  для любого $ t $
  \[
    \Phi(D_t) = \int_{D_t} \eta(\omega) P(d\omega) = \sum_i y_i P(A_i D_t).
  \]
  С другой стороны
  \[
    \int_{D_t} M(\eta \mid  \xi) P(d\omega) = \int_{D_t} \sum_j \sum_i y_i
    P(A\mid D_j) I_{D_j} P(d\omega)
    = \sum_i y_i P(A_i\mid D_t) P(D_t) = \sum_i y_i P(A_i D_t).
  \]
\end{ex}




% $P(\eta \mid  \mathscr{A}) \equiv M(I_A \mid  \mathscr{A})$


  \section{Мартингалы. Субмартингалы}

% TODO начало лекции

\paragraph{Свойства мартингалов}

\begin{enumerate}
  \item % TODO дописать

  \item Пусть $S_n = \sum_{j=0}^n \xi_j$, где $\xi_i$ -- попарно независимы,
    тогда $S_n$ -- мартингал $\Leftrightarrow M\xi_j = 0 \forall j$;
    \begin{proof}
      \begin{multline*}
        M(S_{n+1} | \mathcal{F}_n) = M(S_n + \xi_{n+1} | \mathcal{F}_n)
        = M(S_n | \mathcal{F}_n) + M(\xi_{n+1} | \mathcal{F}_n) = \\
        = S_n + M\xi_{n+1} \equiv S_n \Rightarrow M\xi_{n+1} = 0 \forall n.
      \end{multline*}
      (но $M\xi_0$ может быть не равен $0$)
    \end{proof}

  \item Если $\eta$ -- интегрируемая случайная величина ($M|\eta| < \infty$), тогда
    последовательность $\xi_n = M( \eta | \mathcal{F}_n )$ является мартингалом.
    \begin{proof}
      \[
        M(\xi_{n+1} | \mathcal{F}_n) = M( M(\eta | \mathcal{F}_{n+1}) | \mathcal{F}_n )
        = M(\eta | \mathcal{F}_n) = \xi_n.
      \]
    \end{proof}
\end{enumerate}

\begin{theorem}[Неравенство Йенсена]
  Если $g(x)$ -- выпуклая вниз функция, тогда $g(M\xi) \leqslant Mg(\xi)$, причём
  это неравенство сохраняется и для условного математического ожидания:
  пусть $\mathcal{A}$ -- некоторая произвольная алгебра, тогда
  $g(M(\xi|\mathcal{A})) \leqslant M(g(\xi) | \mathcal{A})$.

  На самом деле эта теорема является свойством интеграла Лебега -- свойством
  сходимости под интегралом Лебега.
\end{theorem}


\subsection{Субмартингалы, теорема Дуба, квадратическая характеристика}

\begin{definition}
  Стохастическая последовательность $\{Y_n\}_{n \geqslant 0}$ называется
  \emph{субмартингалом} относительно потока $ \left\{ \mathcal{F}_n \right\}_{n\geqslant 0} $,
  если
  \[
    M(Y_{n+1} | \mathcal{F}_n) \geqslant Y_n - \text{$P$-п.н.}.
  \]

  И называется \emph{супермартингалом}, если
  $M(Y_{n+1} | \mathcal{F}_n) \leqslant Y_n.$
\end{definition}


\begin{theorem}[Дуб]
  Всякий субмартингал раскладывается $P$-п.н. единственным образом в сумму мартингала
  и предсказуемой стохастической последовательности:
  \[
    Y_n = \underbrace{m_n}_{\text{мартингал}} + \underbrace{A_n}_{\text{предск.}}, n \geqslant 0.
  \]
  при этом $ \left\{ A_n \right\}  $ называется компенсатором $Y_n$.
\end{theorem}
\begin{proof}
  \[
    Y_n = Y_0 + \sum_{j=0}^{n-1} (Y_{j+1} - Y_j)
    = \underbrace{Y_0 + \sum_{j=0}^{n-1} (Y_{j+1} - M(Y_{j+1}|\mathcal{F}_j))}_{m_n}
    + \underbrace{\sum_{j=0}^{n-1} (M(Y_{j+1} | \mathcal{F}_j) - Y_j)}_{A_n},
  \]
  причём $A_n$ -- предсказуема, так как содержит $Y_j$ только до $Y_n$, и является
  $\mathcal{F}_n$-измеримой;
  $m_n$ -- мартингал, т.к.
  \[
    M( m_{n+1} | \mathcal{F}_n ) =
    M\left( m_n + Y_{n+1} - M(Y_{n+1} |\mathcal{F}_n) | \mathcal{F}_n \right) = m_n + 0.
  \]
\end{proof}

\begin{corollary}
  Если $\{m_n\}$ -- мартингал, тогда, в силу неравенства Йенсена,
  \[
    M(m_{n+1}^2 | \mathcal{F}_n) \geqslant \left( \underbrace{M(m_{n+1}|\mathcal{F}_n)}_{m_n} \right)^2
  \]
  или $Y_n = m_n^2$ -- субмартингал
\end{corollary}

\begin{definition}
  Компенсатор субмартингала $m_n^2$ называется \emph{квадратической характеристикой} $m_n$:
  \[
    \langle m \rangle_n = A_n, \quad 
    m_n^2 = \xi_k + A_k,
  \]
  причём
  \[
    \langle m \rangle_n =
    \sum_{k=0}^{n-1} \left[M( m_{k+1}^2 | \mathcal{F}_k) - m_k^2 \right] =
    \sum_{k=0}^{n-1} \left[ M( m_{k+1}^2 - m_k^2 | \mathcal{F}_k ) \right] =
    \sum_{k=0}^{n-1} \left[ M( (m_{k+1}^2 - m_k^2)^2 | \mathcal{F}_k ) \right] 
  \]
  (последнее несложно доказать, если раскрыть квадрат разности, а далее
  $M(m_k m_{k+1} | \mathcal{F}_k) = m_k^2$)
\end{definition}

\begin{ex}
  Пусть $m_k = \sum_{j=0}^k \xi_j$, где $\xi_0, \xi_1, \dots, \xi_n$ -- независимые
  случайные величины с $M\xi_i = 0, i \geqslant 1$.
  \[
    \langle m\rangle_n = \sum_{j=0}^{n-1} M \left( (m_{j+1} - m_j)^2 | \mathcal{F}_j \right) 
    = \sum_{j=0}^{n-1} M \left( \xi_{j+1}^2 | \mathcal{F}_j \right) 
    = \sum_{j=0}^{n-1} M\left(\xi_{j+1}^2\right) = \sum_{j=0}^{n-1} D\xi_{j+1}
  \]
  -- некоторая постоянная (если $M\xi_0 \neq 0$, то первое слагаемое не будет равно дисперсии
  $\xi_0$).
\end{ex}

\begin{theorem}
  Если последовательность $ \left\{ X_k \right\}_{k \geqslant 0} $ -- мартингал и 
  $\exists \alpha > 1 : \sup_n M|X_k|^\alpha < \infty$ (равномерно интегрируема ???), то
  $\exists X_{\infty}, $ что
  \begin{enumerate}
    \item $X_n \toPN X_\infty$;
    \item $X_n \xRightarrow[]{\text{ср.}} X_\infty$, ($M|X_n - X_\infty| \to 0$);
    \item $X_n = M(X_\infty | \mathcal{F}_n)$.
  \end{enumerate}

  Использовать эту теорему можно для проверки условий усиленного закона больших чисел.
\end{theorem}

\begin{lemma}[Кронекер]
  Пусть $\sum_{k=1}^\infty a_k < \infty$, $b_n \uparrow \infty$ (монотонно стремиться к $\infty$),
  тогда $\dfrac{1}{b_n} \sum_{k=1}^n a_k b_n \to 0$.
\end{lemma}

\begin{proof}
  $\varepsilon_k = \dfrac{\xi_k - a}{k}, M\varepsilon_k = 0$, $X_n = \sum_{k=1}^n \varepsilon_k$ -- мартингал.

  $MX_n^2 = M \left( \sum_{k=1}^n \varepsilon_k^2 \right) = \sum_{k=1}^n M\varepsilon_k^2 = \sum_{k=1}^n \dfrac{M(\xi_k - a)^2}{k^2} \leqslant C \sum_{k=1}^n \dfrac{1}{k^2} \leqslant C \zeta(2)$,
  где $\zeta(x)$ -- это функция Римана.

  Тогда $\exists X_\infty : P( X_n \to X_\infty ) = 1$;

  \[
    \dfrac{1}{n} \sum_{k=1}^n \xi_k = \dfrac{1}{n} \sum_{k=1}^n (\xi_k - a) + a
    = a + \underbrace{\dfrac{1}{n} \sum_{k=1}^n \dfrac{\xi_k-a}{k} k}_{\to 0 \text{(по лемме Кронекера)}}
  \]
\end{proof}

\begin{theorem}
  Если $ \left\{ X_n \right\}_{n \geqslant 0} $ -- квадратично интегрируемый мартингал и $P( \langle X\rangle_n \to \infty ) = 1$, то $\dfrac{X_n}{\langle X\rangle_n} \toPN 0$.
\end{theorem}
\begin{ex}
  Дана авторегрессионная последовательность:
  $X_{n+1} = \theta X_n + \varepsilon_{n+1}, n \geqslant 0, X_0 = 0$,
  где $\varepsilon_j$ -- последовательность независимых случайных величин с $M\varepsilon_k = 0$.

  \[
    \hat{\theta} = \dfrac{ \sum_{k=0}^{n-1} X_{k+1} X_k }{ \sum_{k=0}^{n-1} X_k^2 } =
    \dfrac{ \sum_{k=0}^{n-1} (\theta X_k + \varepsilon_{k+1}) X_k }{ \sum_{k=0}^{n-1} X_k^2 } =
    \theta + \dfrac{ \sum_{k=0}^{n-1} \varepsilon_{k+1} X_k }{ \sum_{k=0}^{n-1} X_k^2 } =
    \theta + \dfrac{m_n}{\langle m \rangle_n}
  \]

  \[
    \dfrac{m_n}{\langle m \rangle_n} \toPN 0 \Leftrightarrow \hat{\theta} -
    \text{сильно состоятельная}
  \]
\end{ex}
\begin{proof}
  $\mathcal{F}_k = \sigma(\varepsilon_0, \varepsilon_1, \dots, \varepsilon_k)$
  
  Докажем, что $m_n = \sum_{k=0}^{n-1} \varepsilon_{k+1} X_k$ -- действительно является мартингалом:
  \[
    M(m_{n+1} | \mathcal{F}_n) =
    M \left( \sum_{k=0}^n X_k \varepsilon_{k+1} | \mathcal{F}_n \right) =
    M \left( m_n + X_n \varepsilon_{n+1} | \mathcal{F}_n \right) =
    m_n + X_n \cdot \cancelto{0}{M\varepsilon_{n+1}}
  \]

  \[
    \langle m \rangle_n = \sum_{k=0}^{n-1} M \left(  X_k^2 \varepsilon^2_{k+1} | \mathcal{F}_n \right)  = \sum_{k=0}^{n-1} X_k^2 \underbrace{M\varepsilon_{k+1}^2}_{D\varepsilon = 1} = \sum_{k=0}^{n-1} X_k^2
  \]

  т.о. если $\sum X_k^2 \to \infty$, то $\hat{\theta} \toPN 0$;

  % изучить экспериментально, к чему сходиться последовательность $\dfrac{m_n}{ \sqrt{\langle m\rangle_n} }$
\end{proof}

  \chapter{Лекция 3 - 2023-09-20}
\section{Мотивация}
Пусть $ X = X_1, X_2, \ldots, X_n $ есть выборка (реализация случайной величины
$ \xi $, причём $ \{\xi_n\} $ независимы) из \textsl{известного}
распределения $ F(x, \theta) $, зависящего от \textsl{неизвестного} параметра $
\theta$.

\textsc{Задача}. Оценить значение параметра $ \theta $.

\begin{definition}
   Назовём функцию от выборки
	 \[
		 \widehat \Theta_n = f\left(x_1, x_2,\dots, x_n\right)
	 \]
	 \emph{оценкой параметра $\Theta$}, или \emph{статистикой}.
\end{definition}

Перечислим некоторые \textbf{свойства оценок}.
\begin{definition}
Оценку $ \widehat \Theta_n $ параметра назовём несмещённой, если 
\[
		\mathsf M \widehat \Theta_n = \theta.
\]
\end{definition}
При этом математическим ожиданием здесь считаем %ничего не понял
\begin{gather*}
  \M f(\bar{x}) = \int\limits_{R_n} f(x_1, x_2, \dots, x_n)
	\prod\limits_{j=1}^{n} p_{x_i} (x_i) \, dx_1 dx_2 \dots dx_n.
%	f(x_1, \dots, x_n) \toP 0, \qquad n \to \infty. TODO: разобраться
\end{gather*}

\begin{definition}
	Оценка $ \widehat \Theta_n $ называется \emph{асимптотически несмещённой},
	если  
	\[
		\mathsf M \widehat \Theta_n \to \theta \quad \text{при } n\to\infty.
	\]
\end{definition}

\begin{definition}
	Оценка $\widehat \Theta_n$ называется \emph{состоятельной}, если  
	\[
		\widehat \Theta_n \toP \theta \quad \text{при } n\to\infty.
	\]
\end{definition}

\begin{definition}
Наконец, если  
\[
	\widehat \Theta_n \toPN \theta \quad \text{при } n\to\infty,
\]
то оценку $ \widehat \Theta_n $ называют \emph{сильно состоятельной}.
\end{definition}



%Пусть $x_1, \dots, x_n$ - выборка.
%$M X_i = a$.
%$$\hat{a_n} = ?$$

\section{Выборочное среднее}

\begin{definition}
	\emph{Выборочным средним} называют величину
	\[
		\hat a_n = \bar{X} = \frac{1}{n} \sum\limits_{i=1}^{n} X_i.
	\]
\end{definition}
Будем рассматривать $ \hat a_n $ как некоторую оценку математического ожидания
$ \mathsf M X_i = a $ рассматриваемой случайной величины. Обозначим кроме того
$ \mathsf D X_i = \sigma^2 $.

Перечислим и докажем \textbf{свойства выборочного среднего}.
\begin{enumerate}
	\item \textsc{Несмещённость}.
		\[
			\M \bar{X} = \M\left(\frac{1}{n} \sum\limits_{i=1}^{n} x_i\right) =
			\frac{na}{n} = a.
		\]
	\item \textsc{Сильная состоятельность}. 
		Нужно доказать, что
	\[
		\frac{1}{n} \sum_{i=1}^n X_i \toPN \M X_i = a.
	\]
	Это условие согласно усиленному закону больших чисел выполняется тогда и только тогда,
	когда существует и конечно $ \M
	X_i = a < \infty$.
	\item \textsc{Дисперсия}. Дисперсия выборочного среднего стремится к нулю при увеличении $ n $.
		Действительно,  
		\[
			\mathsf D\bar X = \frac{1}{n^2} \sum_{i=1}^n \mathsf D X_i =
			\frac{\sigma^2}{n}.
		\]
		
\end{enumerate}

\section{Выборочная дисперсия (исправленная)}
\begin{definition} 
	\emph{Выборочной дисперсией} будем называть величину
\[ 
	S^2 = \frac{1}{n-1} \sum\limits_{i=1}^{n} \left( X_i-\bar{X}\right)^2.
\]
\end{definition}

Перечислим \textbf{свойства выборочной дисперсии}.
\begin{enumerate}
	\item Математическое ожидание выборочной дисперсии $ \mathsf M S^2 = \sigma^2
		$ равно истинной дисперсии. 
	\begin{multline*}
  \mathsf M S^2 = \frac{1}{n-1} \mathsf M\left[\sum_{i=1}^{n} \left(X_i - a - \left(\bar{X} - a\right)\right)^2 \right] = \\
  = \frac{1}{n-1} \left[\sum \mathsf M (X_i-a)^2 - 2 \mathsf M\left[ \left(\bar{X} - a\right) \sum
	(X_i - a) \right] + \mathsf M (\bar{X} -a)^2\right] = \\
  = \frac{1}{n-1} \left[ n \sigma^2 - 2 n \mathsf M(\bar{X}-a)^2 + n \mathsf M (\bar{X}-a)^2 \right] = \\
  = \frac{1}{n-1} \left[ n \sigma^2 - n \frac{\sigma^2}{n} \right] = \sigma^2.
\end{multline*}
\item \textsc{Состоятельность}. 
\[
	S_n^2 = \frac{1}{n-1} \sum_{i=1}^n (X_i - \bar{X})^2 = \frac{n}{n-1} \left[
\frac{1}{n} \sum_{i=1}^n X_i^2 - \bar{X}^2 \right] \toPN \sigma^2 \quad
\text{при } n \to \infty.
\]
Действительно,
\[
	\frac{n}{n-1} \to 1, \qquad \bar X^2 \to a^2, 
\]
и, наконец,
\[
	\frac{1}{n} \sum_{i=1}^n X^2_i \toPN \sigma^2 + a^2
\]
в силу усиленного закона больших чисел.
\item \textsc{Дисперсия}.
\[
\mathsf D S^2 = \dfrac{\mu_4}{n} - \frac{\sigma^4 (n-3)}{n (n-1)} = O \left(\frac{1}{n}
\right),
\]
где $\mu_4 = \mathsf M (X_i-a)^4$ --- четвертый момент.
\end{enumerate}

\begin{theorem}[достаточное условие состоятельности]

Пусть $\widehat\Theta_n$ --- асимптотически несмещённая оценка $\Theta$ и
$\mathsf D \bar\Theta_n \to 0$ при $ n \to \infty$.

Тогда $\widehat\Theta_n$ --- состоятельная оценка $\Theta$.
\end{theorem}
\begin{proof}
	Запишем условие состоятельности оценки $ \widehat\Theta_n $ в следующем виде:
\[
  | \widehat\Theta_n - \Theta| = |\widehat \Theta_n - \mathsf M \widehat\Theta_n
	+ \mathsf M \widehat\Theta_n
	- \Theta| \leqslant
	|\widehat\Theta_n - \mathsf M\widehat\Theta_n| + | \mathsf M\widehat\Theta_n -
	\Theta| \toP 0,
\]
где второе слагаемое стремится к нулю ввиду ассимптотической несмещённости $
\widehat\Theta_n $. Для сколь угодно малого $ \varepsilon > 0 $ имеем кроме того
\[
  P( | \widehat\Theta_n - \mathsf M \widehat\Theta_n| > \varepsilon) \leqslant
	\frac{|\widehat\Theta_n - \mathsf M
	\widehat\Theta_n| ^2} {\varepsilon^2} =  \frac{\mathsf D \widehat\Theta_n}
	{\varepsilon^2} \to 0 \quad \text{при } n \to\infty,
\]
что и доказывает утверждение.

\end{proof}



\section{Выборочная функция распределения}
\begin{definition}
	Назовём \emph{индикатором} функцию
\[
	I(x) = \begin{cases}1, &x>0,\\
	0, &x\leqslant 0.\end{cases}
\]
\end{definition}
Видно, что индикатор непрерывен слева.

\begin{definition}
	Назовём \emph{выборочной функцией распределения} функцию
	\[
		\widehat F_n (x) = \frac{1}{n} \sum_{i=1}^n I(x-X_i).
	\]
\end{definition}

\noindent Назовём некоторые из \textbf{свойств выборочной функции распределения}.
\begin{enumerate}
	\item \textsc{Несмещенность}.
\[
	\mathsf M \widehat F_n (x) = \mathsf M \frac{1}{n} \sum_{i=1}^n I(x-X_i) =
	\frac{1}{n} \sum_{i=1}^n P(X_i < x) = F(x).
\]

\item \textsc{Сильная состоятельность}. Исходя из усиленного закона больших
	чисел, имеем 
	\[
		\hat F_n(x) = \frac{1}{n} \sum_{i=1}^n I(x-X_i) \toPN \M I(x - X_i) = P(X_i
		< x) = F(x),
	\]
	что и требовалось доказать.

\item \textsc{Теорема Гливенко -- Кантелли}.
	\begin{theorem}[Гливенко -- Кантелли] 
		\[
			\sup_x |\widehat F_n (x) - F(x)| \toPN 0.
		\]
	\end{theorem}
\begin{proof}
	б/д
	%TODO: (proof)

\end{proof}

\item \textsc{Неравенство Дворецкого -- Кифера -- Волфовица}.

\[
	P(\sup_x |\widehat F_n(x) - F(x)| > \varepsilon) \leqslant 2 e^{-2n
	\varepsilon^2}.
\]
\begin{proof}
	б/д
	%TODO: (proof)

\end{proof}
\begin{corollary*} 
	Возьмём для некоторого $ \alpha > 0 $
\[
		\varepsilon = \sqrt{ \frac{\ln 2/\alpha }{2n}}.
	\]
	Тогда
	\[
		P\left(\sup_x |\widehat F_n (x) - F(x)| \leqslant \varepsilon\right)
		\geqslant 1 - 2 e^{-2n \varepsilon^2} = 1-\alpha.
	\]
\end{corollary*}
Таким образом, с вероятностью $ 1 - \alpha $ имеем
\[
		\widehat F_n (x) - \varepsilon \leqslant F(x) \leqslant \widehat F_n(x) +
		\varepsilon,
\]
или более точно,
\begin{gather*}
	L(x) \leqslant F(x) \leqslant R(x), \quad \text{где}\\
		L(x) = \max \{ \widehat F_n(x) - \varepsilon,\, 0 \}, \qquad R(x) =
		\min \{ \widehat F_n(x) + \varepsilon, \,1 \}.
	\end{gather*}

\item \textsc{Общие свойства}:
	\begin{enumerate}
	\item Пределы  выборочной функции распределения при $ x \to \pm\infty $ равны
		соответственно
		\[
		\widehat F_n(+\infty) = 1, \qquad \widehat F_n(-\infty) = 0.
	\]
	\item Выборосная функция функция распределения $\widehat F_n(x)$ не убывает.
	\item Из первых двух свойств легко получаем 
	\[
			0 \leqslant \widehat F_n(x) \leqslant 1.
	\]
	
	\end{enumerate}
\end{enumerate}

\begin{theorem} Если выборка $x_1, \dots, x_n$ получена из закона распределения
	$F(x)$, то $\widehat F_n(x)$ есть дискретная случайная величина с распределением
\[
	P(\widehat F_n(x) = k/n) = C_n^k F(x)^k (1-F(x))^{n-k} 
\]
\end{theorem}
\begin{proof} 
	Из определения выборочной функции распределения
\[
	\widehat F_n \left(x\right) = \frac{1}{n} \sum_{i=1}^n I\left(x-X_i\right)
\]
сразу виден закон распределения Бернулли с вероятностью успеха $ p = F(x) $. 

Действительно, при каждом $ x \in \mathbb R $ индикаторы $ I(x - X_k) $ ---
независимые случайные величины с распределением 
\begin{align*}
	P(I(x-X_k) = 1) &= F(x) = p, \\
		P(I(x-X_k) = 0) &= 1 - F(x) = q.
\end{align*}

%TODO: дописать!
\end{proof}

\begin{corollary}[несмещённость]
	\[
		\mathsf M \widehat F_n (x) = \frac{np}{n} = p = F(x).
	\]
\end{corollary}
\begin{corollary}
	Если $ X_1, X_2, \ldots $ --- выборка неограниченного объёма, то 
	\[
			\widehat F_n(x) \toP F(x).
	\]
\end{corollary}
	\begin{proof}
	\begin{multline*}
		\mathsf D \widehat F_n (x) = \frac{npq}{n^2} = \frac{F(x) (1 - F(x))}{n}
		\Leftrightarrow\\\Leftrightarrow
 \mathsf M (\widehat F_n(x) - F(x))^2 \to 0\Leftrightarrow\\
\Leftrightarrow \widehat F_n (x) \to F(x) \quad \text{(среднеквадратично при $
n\to\infty $)}.
\end{multline*}
Отсюда и $ \widehat F_n(x) \toP F(x) $.

\end{proof}
\setcounter{corollary}{0}


\section{Порядковая статистика}
\begin{definition}
	Выборка, упорядоченная по возрастанию
	\[
		X_{(1)} \leqslant \dots \leqslant X_{(n)},
	\]
	 называется \emph{вариационным рядом}.
\end{definition}
\begin{definition}
	Член  вариационного ряда $ X_{(k)} $ называют \emph{$ k $-ой порядковой статистикой}.
\end{definition}

\begin{definition}
Назовём \emph{размахом выборки} число
\[
	\omega = X_{(n)} - X_{(1)}.
\]
\end{definition}

\begin{theorem}
	Если независимая выборка взята из генеральной совокупности с
функцией распределения $F(x)$, то функции распределения крайних членов
вариационного ряда и их совместная функция распределения имеют вид 
\begin{align*}
	\text{1. }F_{X_{(n)}}(x) &= [F(x)]^n,\\
	\text{2. }F_{X_{(1)}}(x) &= 1 - [1 - F(x)]^n,\\
	\text{3. }F_{X_{(1)}, X_{(n)}} (x, y) &= [F(y)]^n - [F(y)-F(x)]^n, \quad x < y.
\end{align*}
\end{theorem}
\begin{proof} Проведём доказательсто по всем пунктам.
	\begin{enumerate}
		\item Из независимости случайных величин и равенства $ X_{(n)} = \max\limits_i X_i $ вытекает соотношение
	\[
		F_{X_{(n)}}(x) = P \left( X_{(n)} < x \right) = P \left( \bigcap_{k=1}^n (X_k <
		x)\right) = \prod_{k=1}^n P(X_k < x) = [F(x)]^n.
	\]
\item Аналогично для $ X_{(1)} $ получим 
\[
	F_{X_{(1)}}(x) = 1 - P(X_{(1)} \geqslant x) = 1 - \prod_{k=1}^n P(X_k \geqslant
	x) = 1 - [1- F(x)]^n.
\]
\item В последнем случае 
\begin{multline*}
	F_{X_{(1)}, X_{(n)}}(x,y) = P(X_{(1)} < x, \, X_{(n)} < y) = \\ = P(X_{(n)} < y ) -
	P\left(X_{(1)} \geqslant x, \, X_{(n)} < y \right) = \\ =
	[F(y)]^n - P \left( \bigcap_{k=1}^n (x \leqslant X_k < y ) \right)  = \\ =
	[F(y)]^n - \prod_{k=1}^n P(x \leqslant X_k < y) = \\ =
	[F(y)]^n - [F(y) - F(x)]^n, \quad x < y.
\end{multline*}
\end{enumerate}
\end{proof}
\begin{corollary*}
	Если выборка взята из абсолютно непрерывного закона $F(x)$ с 
плотностью $p(x)$, то плотности распределения крайних членов вариационного ряда и их 
совместная плотность имеют вид 
\begin{align*}
	p_{X_{(n)}}(x) &= n[F(x)]^{n-1}p(x),\\
	p_{X_{(1)}}(x) &= n[1-F(x)]^{n-1}p(x),\\
	p_{X_{(1)}, X_{(n)}}(x, y) &= n(n-1)[F(y)-F(x)]^{n-2}p(x)p(y), \quad x < y.
\end{align*} %TODO: как получить последнее? (вписать формулу)
\end{corollary*}

\begin{theorem}
	Если независимая выборка взята из генеральной совокупности с
плотностью распределения $p(x)$, то плотность распределения $k$-ой порядковой
статистики имеет вид 
\[
	p_{X_{(k)}}(x) = n C^{k-1}_{n-1} \left[ F(x) \right]^{k-1} [1 - F(x)]^{n-k}
	p(x).
\]
\end{theorem}
\begin{proof}
	%TODO: поменять доказательство с инженерного на математическое (см. книгу
	%<<Порядковые статистики>>)
Зафиксируем $ x \in \mathbb R $ и выберем столь малое $ \Delta x $, что в
промежуток $ [x, x+ \Delta x) $ может попасть только один элемент выборки (закон
абсолютно непрерывный!). Тогда согласно полиномиальной схеме событие 
$(x \leqslant X_{(k)} < x + \Delta x)$ означает, что какие-то $ k - 1 $ элементов
выборки попали в промежуток $ (-\infty, x) $, 
один элемент в промежуток $[x, x + \Delta x)$, а остальные в $[x + \Delta x,
+\infty)$. Поскольку вероятности 
попаданий в эти множества равны $ F(x) $, $ p(x)\Delta x $ и $(1 - F(x + \Delta
x))$ соответственно, то  
\begin{multline*}
	F_{X_{(k)}}(x+\Delta x) - F_{X_{(k)}} (x) = P(x \leqslant X_{(k)} < x + \Delta
	x) = \\ =
	\frac{n!}{(k-1)!1!(n-k)!} [F(x)]^{k-1} p(x) \Delta x [1 - F(x + \Delta
	x)]^{n-k} = \\ =
	n C^{k-1}_{n-1} [F(x)]^{k-1} [1 - F(x + \Delta x)]^{n-k} p(x)\Delta x.
\end{multline*}
Завершим доказательство делением на $ \Delta x $ и переходом к пределу при $
\Delta x \to 0 $.

\end{proof}



\section{Примеры}
\begin{ex}
	Пусть дана выборка объёма $ n $ из показательного закона с параметром $ \alpha
	$. Тогда
	\[
		p_{X_{(1)}}(x) = n(1 - (1 - e^{-\alpha x}))^{n-1} \alpha e^{-\alpha
		x} = n \alpha e^{-n\alpha x}.
	\]
\end{ex}
\begin{ex}
	Найдём математическое ожидание и дисперсию $ X_{(k)} $, если выборка получена
	из равномерного распределения на $ [0, a] $. 

	Математическое ожидание минимального члена вариационного ряда равно
	\begin{multline*}
		\mathsf M X_{(1)} = \frac{n}{a }\int\limits_{0}^{a} x\cdot \left( 1 - \frac{x}{a}
		\right)^{n-1}\,dx \to \left[ \begin{aligned} u &= 1 - x/a \\ dx &= -a\,du
		\end{aligned}\right] \to an \int\limits_{0}^{1} \left( 1 - u \right) \cdot
		u^{n-1}\,du =\\=
		\frac{an}{n+1} \left. \left( 1 - \frac{x}{a} \right)^{n+1} \right|^a_0 - a
			\left.\left( 1 - \frac{x}{a} \right)^n \right|^a_0 = - \frac{an}{n+1} + a
				= \frac{a}{n+1}.
	\end{multline*}
	
	Математическое ожидание максимального члена вариационного ряда равно 
	\[
	\mathsf M X_{(n)} = \frac{n}{a}\int\limits_{0}^{a} x
	\left(\frac{x}{a}\right)^{n-1}\,dx = \frac{an}{n+1} \left.\left( \frac{x}{a}
	\right)^{n+1} \right|^a_0 = \frac{an}{n+1}.
	\]
	Отсюда видно, что $ X_{(n)} $ --- асимптотически несмещённая оценка параметра
	$ a $.

	Наконец, 
	\begin{multline*}
		\mathsf M X_{(k)} = \frac{n}{a}\int\limits_{0}^{a} x\cdot C^{k-1}_{n-1}
		\left(\frac{x}{a}\right)^{k-1} \left( 1 - \frac{x}{a} \right)^{n-k}\,dx = \\
		= an C^{k-1}_{n-1} \int\limits_{0}^{1}
		u^k(1-u)^{n-k}\,du = 
		an C^{k-1}_{n-1} B(k+1, n-k+1)  = \\ = a \frac{(n-1)!n}{(k-1)! (n-k)!}
		\frac{\Gamma(k+1) \Gamma(n-k+1)}{\Gamma(n+2)} = \frac{ak}{n+1},
	\end{multline*}
	где $ u = x/a $.
\end{ex}

  \chapter{Лекция 4 - 2023-09-27 - Методы получения точечных оценок: моментов, максимального
правдоподобия. Свойства точечных оценок.}
%%\section{Вероятности для вариационного ряда}
%%\begin{theorem}
%%	$X_1, X_2, \dots, X_n$ - независимыя выборка из з.р. $F(X)$

%%	тогда $F_{X_{(n)}} = (F(x))^n$, $F_{X_{(1)}} = 1 - (1-F(x))^n$.

%%	если имеется плотность $p(x)$, то $p_{X_{(n)}} = n (F(x))^{n-1} p(x)$, $p_{X_{(1)}} = n (1-F(x))^{n-1} p(x)$, $p_{X_{(1)}, X_{(n)}} = n (n-1) (F(y) - F(x))^{n-2} p(x) p(y)$.
%%\end{theorem}

%%\begin{proof}
%%	\begin{multline}
%%		F_{X_{(n)}} = P(X_{(n)} < x) = P\left(\bigcap_{k=1}^n (X_k < x)\right) = \prod_{k=1}^n P(X_k < x) = (F(x))^n\\
%%		F_{X_{(1)}} (x) = P(X_{(1)} < x) = 1 - P(X_{(1)} \geqslant x) = \dots = 1 - (1-F(x+0))^n = 1 - (1-F(x))^n \\
%%		F_{X_{(1)}, X_{(n)}} (x, y) = P(X_{(1)} < x, X_{(n)} < y) = P(X_{(n)} < y) - P(X_{(n)}, X_{(1)} \geqslant x) = \\
%%		F_{X_{(n)}} (y) - P(\bigcap_{k=1}^n (x \leqslant X_k < y)) = (F(y))^n - (F(y) - F(x))^n
%%		% \\
%%		% TODO: доказательство PX(1) X(n) (x, y)
%%	\end{multline}
%%\end{proof}

%%\begin{ex}
%%	$X_1, X_2, \dots, X_n ~ E(\alpha)$
%%	$F_{x_i} (x) = 1- e^{-\alpha x}, x\geqslant 0.$
%%	$F_{X_{(1)}} (x) = 1 - (1-1+e^{-\alpha x})^n = 1 - e^{-\alpha x} \Rightarrow X_{(1)} \sim E(n\alpha)$
%%\end{ex}

%%\begin{theorem}
%%	$X_1, X_2, \dots, X_n$ - независимая выборка из з.р. с плотностью $p(x)$.
%%	тогда $p_{X_{(k)}} = n C_{n-1}^{k-1} (F(x))^{k-1} (1-F(x))^{n-k} p(x)$.
%%\end{theorem}

%%\begin{proof}
%%	\[
%%		P(x\leqslant X_{(k)} < x+\Delta x) = \dfrac{n!}{(k-1)! (n-k)!} (F(x))^{k-1} p(x) \Delta x (1-F(x+\Delta x))^{n-k}
%%	\]
%%\end{proof}

\section{Методы построения точечных оценок параметров}
\subsection{Мотивация}
Пусть
\[
	X_1, X_2, \dots, X_n \sim F(x, \theta_1, \theta_2, \dots, \theta_n),
\]
то есть случайная величина выборки распределена по некоторому
\textsl{неизвестному} закону, зависящему от \textsl{неизвестных} параметров
$\bar \theta = (\theta_1, \theta_2, \ldots, \theta_n) $.

\textsc{Задача}. Оценить значения неизвестных параметров.


\subsection{Метод моментов}
\begin{definition}
Назовём величины
\begin{align*}
\hat{\mu}_k &= \frac{1}{n} \sum X_i^k, \\
\mu_k &= \int\limits_{-\infty}^{+\infty} x^k dF(x, \bar{\theta}).
\end{align*}
соответственно \emph{эмпирическим} и \emph{теоретическим} моментом $ k $-го
порядка.
\end{definition}

По закону больших чисел $\hat{\mu}_k \toPN \mu_k$. Составим тогда систему
уравнений
\[
\begin{cases}
  \hat{\mu}_1 = \mu_1(x, \bar{\theta}), \\
  \hat{\mu}_2 = \mu_2(x, \bar{\theta}), \\
  \ldots = \ldots, \\
  \hat{\mu}_r = \mu_r(x, \bar{\theta}).
\end{cases}
\]
Её решением будет некоторая оценка параметра $ \bar \theta $.

\begin{ex}
  Пусть выборка $X_1, \dots, X_n \sim E(\alpha)$ распределена по
	экспоненциальному закону.

  Тогда
	\[
		\hat \mu_1 = \bar{X} = \frac{1}{\alpha} = \mu_1 \Rightarrow \alpha =
		\frac{1}{\bar{X}},
	\]
	причём данная оценка является асимптотически несмещённой. % у всех почему-то
	% "асимптотически"
  \begin{proof}
    $X_k \sim \alpha e^{-\alpha x} \equiv \gamma_{\alpha, 1}$
		Тогда
		\[
			\xi = \sum_{k=1}^n X_k \sim \gamma_{\alpha, n}.
		\]

		Заметим, что $ \hat \alpha_n = 1/\bar X = n/\xi $, а значит,
		\[
			\mathsf M \hat \alpha_n = \mathsf M \frac{n}{\xi} = \int\limits_0^{+\infty} \frac{n}{x}
			\frac{\alpha^n x^{n-1} e^{-\alpha x}}{\Gamma(n)} \, dx = \frac{n
			\alpha}{\Gamma(n)} \int\limits_0^{+\infty} (\alpha x)^{n-2} e^{-\alpha x}
			\, dx = \frac{n\alpha \Gamma(n-1}{\Gamma(n)} = \frac{n\alpha}{n-1},
		\]
		откуда сразу вытекает асимптотическая несмещённость полученной оценки.

  \end{proof}
\end{ex}


\subsection{Оценки максимального правдоподобия}
Пусть выборка распределена одним из двух возможных способов
\[
	X_1, X_2, \dots, X_n \sim \begin{cases} p(x, \bar\theta)
		&\text{(абсолютно непрерывный случай)}, \\
		P_{\bar \theta}(\xi = X_k) &\text{(дискретный случай)}.
\]
\begin{definition}
	Назовём \emph{функцией правдоподобия} функцию, равную
	\[
		\mathscr{L} (X_1, X_2, \dots, X_n, \hat\theta) = 
		\prod_{k=1}^n p(X_k, \hat \theta)
	\]
в непрерывном случае и  
\[
	\mathscr{L} (X_1, X_2, \dots, X_n, \hat\theta) = P_{\bar\theta}(\xi_1 =
		X_1, \xi_2 = X_2, \dots, \xi_n = X_n) = \prod_{k=1}^n P_{\bar\theta}(\xi_k = X_k)
\]
в дискретном случае.
\end{definition}

  $\mathscr{L}$ исследуется на максимум по параметру $\theta$. При этом зачастую
	удобнее рассматривать не саму функцию $ \mathscr L $, а её логарифм $ \ln
	\mathscr L $, в связи с чем
	заметим, что
	\[
		\frac{\partial \mathscr{L}}{\partial \theta} = 0 \Leftrightarrow
		\frac{\partial \ln \mathscr L}{\partial \theta} = \frac{1}{\mathscr{L}}
		\frac{\partial \mathscr{L}}{\partial \theta} = 0,
	\]
что ясно и без привлечения аппарата дифференцирования (функция $ \ln x $
монотонно возрастающая).
\begin{ex}
	Пусть $ X_1, X_2, \dots, X_n \sim \operatorname{Pois}(\lambda) $.

	Тогда
	\[
		P(\xi=x_k) = \frac{\lambda^{X_k}}{(X_k)!} e^{-\lambda},
	\]
а значит,
\[
	\mathscr{L}(X_1, \dots, X_n, \lambda) = \prod_{k=1}^n \frac{\lambda^{X_k}}{(X_k)!}
	e^{-\lambda} = \frac{\lambda^{\sum X_k}\cdot
	e^{-n\lambda}}{\prod_{k=1}^n (X_k)!}
\]

Вычислим теперь
\[
	\ln \mathscr{L} = \sum_{k=1}^n X_k \ln\lambda - n \lambda - \ln \prod_{k=1}^n (X_k)!,
\]
откуда
\[
	\frac{\partial \ln \mathscr{L}}{\partial \lambda} = \frac{\sum_{k=1}^n X_k}{\lambda}
	- n = 0 \Rightarrow \bar\lambda_n = \frac{1}{n} \sum_{k=1}^n X_k.
\]
\end{ex}

\begin{ex}
  Пусть $X_1, X_2, \dots, X_n \sim E(\alpha)$.

	Имеем
	\begin{align*}
		\mathscr{L}(X_1, \dots, X_n, \alpha) &= \prod_{k=1}^n \alpha e^{-\alpha X_k}
		= \alpha^n e^{-\alpha \sum x_k},\\
		\ln \mathscr{L} &= n \ln\alpha - \alpha \sum_{k=1}^n X_k, \\
		\frac{\partial \ln\mathscr{L}}{\partial \alpha} &= \frac{n}{\alpha} -
		\sum_{k=1}^n
		X_k = 0,\\
		\alpha &= \frac{1}{\bar X}.
	\end{align*}
	Таким образом, \textsl{в данном случае} \boxed{\text{ОММ} = \text{ОМП}.}
\end{ex}


\subsection{Сравнение оценок}
\begin{definition}
	Пусть $\hat \theta_n$ и $\hhat\theta_n$ --- две \textsl{несмещенные} оценки параметра $\theta$.
	Если
	\[
		\mathsf D \hat\theta_n < \mathsf D \hhat\theta_n,
	\]
	то говорят, что $\hat\theta_n$
	\emph{более эффективна}, чем $\hhat\theta_n$.
\end{definition}

\begin{ex}
  Пусть $X_1, X_2, \dots, X_n \sim R[\theta-\frac{1}{2}, \theta+\frac{1}{2}]$ и
\[
	\hat\theta_n = \frac{1}{2} (X_{(1)} + X_{(n)}), \qquad
	\hhat\theta_n = \bar X,\\
\]

	Тогда
	\begin{gather*}
	\D\hat\theta_n = \frac{1}{4} \left[ \M X^2_{(1)} + \M X_{(n)}^2 + 2\M
	X_{(1)}X_{(n)} \right] - \theta^2 = \frac{1}{8 (n+1) (n+2)} \sim \frac{c}{n^2},\\
	\D\hhat\theta_n = \frac{\sigma^2}{n} = \frac{1}{12n}.
\end{gather*}
\end{ex}

\begin{theorem}[неравенство Рао -- Крамера]
  Пусть $X_1, X_2, \dots, X_n$ --- выборка из закона распределения с плотностью
	$p(x, \theta)$ и интеграл $\int_{\mathbb R^n} p(x, \theta) \, d\bar x = 1$ допускает дифференцирование
	по $\theta$\footnote{Регулярное семейство}. 
	%TODO: что значит регулярное семейство?
  
	Тогда для любой несмещенной оценки $\hat\theta_n = \varphi(X_1, X_2, \ldots,
	X_n)$ имеет место неравенство
  \[
		\D\hat\theta_n \geqslant \frac{1}{I_n(\theta)} = \frac{1}{n I_1(\theta)},
	\]
  где функцию
	\[
		I_n(\theta) = \M\left( \dfrac{\partial \ln \mathscr{L}}{\partial
		\theta} \right)^2
	\]
	называют \emph{информацией Фишера}, а функция
	\[
		I_1(\theta) = M\left( \dfrac{\partial\ln p(x, \theta)}{\partial\theta}
		\right)^2 = \int\limits_{-\infty}^{+\infty} \left( \frac{\partial \ln p(x,
		\theta)}{\partial \theta} \right)^2 p(x, \theta) \,dx
	\]
	есть информация Фишера в одном наблюдении.
\end{theorem}
\begin{remark*}
	Для дискретной модели следует заменить плотность $ p(x,\theta) $ на
	вероятность $ P(X = x_k) $.
\end{remark*}
\begin{proof}
	Продифференцируем по параметру очевидные интегральные соотношения
	\begin{multline*}
		1 = \int\limits_{\mathbb R^n} p(\bar x, \theta) \, d\bar x \Rightarrow 0 =
		\int\limits_{\mathbb R^n} \frac{\partial p(\bar x, \theta)}{\partial\theta} \,
		d\bar x = \\ = \int\limits_{\mathbb R^n} \frac{\partial \ln p(\bar x,
		\theta)}{\partial\theta} p(\bar x, \theta) \, d\bar x = \M \left( \frac{\partial
		\ln p(X_1, X_2, \ldots, X_n, \theta)}{\partial \theta} \right).
	\end{multline*}
	Тогда
	\[
		\M \hat\theta_n = \theta = \int\limits_{\mathbb R^n} \varphi(\bar x)
		p(\bar x, \theta) \, d\bar x \Rightarrow 1 = \int\limits_{\mathbb R^n}
		\varphi(\bar x) \frac{\partial \ln
		p(\bar x, \theta)}{\partial\theta} p(\bar x, \theta) \, d\bar x.
	\]

	Отсюда следует, что
	\[
		1 = \int\limits_{\mathbb R^n} (\varphi(\bar x)-\theta) \frac{\partial \ln
		p(\bar x, \theta)}{\partial
		\theta} p(\bar x, \theta) \, d\bar x,
	\]
а значит, с учётом неравенства Коши -- Буняковского
		\begin{gather*}
		1 \leqslant \int\limits_{\mathbb R^n} (\varphi(\bar x) - \theta)^2 p(\bar x,
		\theta) \, d\bar x \cdot
		\int\limits_{\mathbb R^n} \left(\frac{\partial\ln p(\bar x,
		\theta)}{\partial\theta}\right)^2 p(\bar x, \theta)
		\, d\bar x = \D \hat \theta_n \cdot I_n(\theta),\\
		I_n(\theta) = \M \left(\frac{\partial\ln \mathscr L(X_1, X_2, \dots, X_n,
		\theta)}{\partial\theta}\right)^2 = \D \frac{\partial \ln \mathscr L}{\partial\theta}
		= \sum_{k=1}^n \D\frac{\partial \ln p(X_k, \theta)}{\partial\theta} = n
		I_1(\theta),
	\end{gather*}
	поскольку
	\[
		\ln \mathscr L(X, \theta) = \sum_{k=1}^n \ln p(X_k, \theta).
	\]
\end{proof}

\begin{ex}
  Пусть $X_1, \dots, X_n \sim R[\theta-1/2, \theta+1/2]$.

	Тогда
	\[
		p(x, \theta) = I(\theta-1/2 \leqslant x \leqslant \theta+1/2).
	\]
При этом интеграл
\[
	1 = \int\limits_{-\infty}^{+\infty} I(x) \, dx
\]
не допускает дифференцирования по параметру.
\end{ex}

\begin{ex}
	Пусть  $X_1, X_2, \dots, X_n \sim E(\alpha)$.

	Тогда
	\begin{align*}
		p(x, \alpha) &= \alpha e^{-\alpha x},\\
		I_1(\alpha) &= \M \left(\frac{\partial\ln p(X_1, \alpha)}{\partial\alpha}\right)^2,\\
		\ln p(X_1, \alpha) &= \ln \alpha - \alpha X_1,\\
		\frac{\partial \ln p}{\partial\alpha} &= \frac{1}{\alpha} - X_1,\\
		I_1(\alpha) &= \M (\frac{1}{\alpha} - X_1)^2 = \D X_1 = \dfrac{1}{\alpha^2}.
\end{align*}
\textsc{Вывод}. Для любой несмещенной оценки $\hat\alpha_n$
\[
	\D \hat\alpha_n \geqslant \frac{1}{n/\alpha^2} = \frac{\alpha^2}{n}.
\]
\end{ex}



  \section{Ортогональная случайная мера}
Комбинация случайной величины и некоторой меры. Пусть дано вероятностное
пространство $ (\Omega, \mathscr{F}, \mathsf{P}) $, а также $ (E, \mathscr{E})
$, где $ \mathscr{E} $ --- $ \sigma $-алгебра подмножеств. Построим
\emph{ортогональную случайную меру} $ z(\omega, \Delta) $, $ \Delta \in \mathscr
E$, $ z\in\mathbb C $, $ z $ --- случайная величина.

Назовём некоторые \textbf{свойства ортогональной случайной меры}.
\begin{enumerate}
  \item Для любого $ \Delta $ справедливо $
    MZ(\omega,\Delta)\overline{Z(\omega,\Delta)} <\infty $.
  \item Для любого $ \Delta $ справедливо $MZ(\omega,\Delta) = 0$.
  \item Для $\Delta_1 \cap \Delta_2 = \varnothing$ справедливо
  \[
      Z(\omega, \Delta_1 + \Delta_2) = Z(\omega,\Delta_1) + Z(\omega, \Delta_2).
  \]
  \item $ \Delta_i \cap \Delta_j = \varnothing $, $ i\neq j $
  \[
    \Delta = \bigcup_k \Delta_k  M \biggl|Z(\omega, \Delta) - \sum_{k=1}^n
    Z(\omega,\Delta_k) \biggr|^2 \to 0
  \]
  при $ n\to\infty $.
\item \textsc{Ортогональность.} Для $ \Delta_1 \cap \Delta_2 = \varnothing $
  справедливо
\[
  M\big[Z(\omega,\Delta_1) \cdot \overline{Z(\omega,\Delta_2)}\big] = 0.
\]
При этом 
\[
  M Z(\omega, \Delta)\overline{Z(\omega, \Delta)} = m(\Delta) \text{ ---
  \emph{структурная функция меры}}.
\]
\end{enumerate}

\begin{ex}
  Пуассоновская случайная мера. $ \nu(\omega,\Delta) $ --- число событий в
  пуассоновском потоке на $ \Delta $, $ E = [0, \infty) $, $
  \mathscr{E} $ --- борелевские подмножества. Для $ \Delta \in
  \mathscr{E} $ $ \pi(\Delta) < \infty $. При этом  
  \[
    P(\nu(\omega,\Delta)=k) = \frac{(\pi(\Delta))^k}{k!}e^{-\pi(\Delta)}, \quad
    k = 0,1, \ldots.
  \]
  
  Проверим свойства.
  \begin{enumerate}
    \item $ M\nu^2(\omega,\Delta) = \text{существ.} $ \checkmark
    \item $ M\nu(\omega,\Delta) = \pi(\Delta) \neq 0 $. \cross
    \item[3, 4.] автоматически. \checkmark
    \item[5.] $ M\nu(\omega,\Delta_1)\overline{\nu(\omega,\Delta_2)} =
      M\nu(\omega,\Delta_1)\nu(\omega, \Delta_2) = 0 $ (независимые величины).
      \checkmark
  \end{enumerate}
  
   %TODO дописать
\end{ex}
 
\begin{ex}
  Винеровская случайная мера. Пусть $ W_t $ --- стационарный винеровский
  процесс. 
  \[
      W(\omega, \Delta) = W_b - W_a
  \]
 \[
   E = [0, +\infty) \quad \Delta = [a, b]
 \]
 $ \varepsilon $ 
 На $ \varepislon $ продолжаем по теореме Каритоодори. 
 \[
     MW(\omega, \Delta) \overline(\omega,\Delta) = M(W_b - W_a)^2 = m(\Delta) =
     \sigma^2(b-a) = (b-a).
 \]
  
\[
    MW(\omega, \Delta) = M(W_b - W_a) = 0
\]
 
\[
  MW(\omega,\Delta_2) W(\omega, \Delta_2) = 0 (\text{незав., поскольку $
  \Delta_1 \cap \Delta_2 = \varnothing $}).
\]
\end{ex} 

Всё понятие было бы абстрактным, если бы не следующий пример.
\begin{ex}
  $ \xi_n = \sum_{k=1}^N Z_k e^{i\lambda_k n}, $ $ \lambda_i \in (-\pi, \pi) $.
  $ MZ_i = 0 $, $ MZ_iZ_i = \sigma^2 $, $ MZ_iZ_j = 0 $, откуда $ MZ(\omega,\Delta) = 0
  $.
   
 \[
   Z(\omega, \Delta) = \sum_{k=1}^N z_k I(\lambda_k \in\Delta).
 \]
 
\[
  MZ(\omega, \Delta_2)\overline{Z(\omega,\Delta_2)} = M\left(\sum_{k=1}^N
  Z_kI(\lambda_k \in\Delta_2)\right) \left( \sum_{j=1}^n Z_j I(\lambda_j \in
  \Delta_2) \right)  = 0.
\]
 
\begin{multline*}
  MZ(\omega, \Delta)\overline{Z(\omega,\Delta)} = M\sum_{k=1}^NZ_k
  I(\lambda_k\in\Delta)\sum_{j=1}^N\overline{Z}_j I(\lambda_j \in \Delta) =\\=
  \sum_k \sum_j I(\lambda_k\in\Delta) I(\lambda_j\in\Delta) MZ_k\overline{Z_j} =
  \sum_{k=1}^n \sigma^2_k I(\lambda_k \in \Delta) = G(\Delta)
\end{multline*}
 
 
\[
    G(\Delta) = \int dG(\lambda),
\]
где $ G(\lambda) = \sum_{k=1}^n \sigma_k^2 I(\lambda - \lambda_k) $.

 
\begin{align*}
  K_\xi(n) &= \int\limits_{-\pi}^{\pi}e^{i\lambda n}\,dG(\lambda), \\
  \xi_n &= \int\limits_{-\pi}^{\pi}e^{i\lambda n} Z(d\lambda)= e^{i\lambda_1 n}
  Z_1 + e^{i\lambda_N} %TODO: дописать
\end{align*}
\end{ex}
%TODO: рис ко всем примерам


\subsection{Интеграл от неслучайной функции по ОСМ}
$ \mathscr H $ --- гильбертово пространство комплексные с. в. $ M\xi\bar{\xi} < \infty $ 
\[
  (\xi, \eta) = M\dot{\xi} \dot{\bar{\eta}} = M\xi\bar{\eta} - M\xi\cdot
  M\bar{\eta},
\]
 
$\mathscr{L}_2$ --- пространство комплексных (неслуч) функций на $ E $. 
\[
  \int_E f(\lambda)\overline{f(\lambda)} m(d\lambda) < \infty,
\]
 
\[
  (f, g) = \int_E f(\lambda)\overline{g(\lambda)}md(\lambda)
\]
 
\[
    \int_E f(\lambda)Z(d\lambda)
\]
1. $ f(\lambda) = \int_{k=1}^n f_k I(\lambda \in \Delta_k) $,
$ \Delta_k \cap \Delta_j = \varnothing $, $ k\neq j $.
 
\[
  \int_E f(\lambda)Z(d\lambda) = \sum_{k=1}^n f_k Z(\omega,\Delta_k)
\]

Свойства.
\begin{enumerate}
  \item $ MI(f) = 0 $,  
  \[
    MI(f) = \sum_{k=1}^n f_k MZ(\omega, \Delta_k) = 0.
  \]
\item $ I(\alpha f +\beta g) = \alpha I(f) + \betaa I(g) $
\item
  \begin{multline*}
    \cov(I(f), I(g)) = MI(f) \overline{I(g)} = \\ =
    M \left[ \sum_{k=1}^n f_k
      Z(\omega, \Delta_k) \overline{\sum_{j=1}^n g_j Z\left(\omega,\Delta_j\right)}
    \right] = \sum_{k=1}^n f_k \bar{g}_k MZ(\omega, \Delta_k)
    \overline{Z(\omega,\Delta_k)},
  \end{multline*}
\[
  g = \sum_{k=1}^n g_k I(\lambda \in \Delta_k).
\]
 
\[
  \int_E f(\lambda)\overline{g(\lambda)} m(d\lambda) = \sum_{k=1}^n f_k\bar{g}_k
  m(d\lambda).
\]
\end{enumerate}

\paragraph{2.} $ f $ --- произвольная неслучайная функция. Пусть $ f_n $ ---
последовательность простых функций $ \| f - f_n \|_{\mathscr{L}_2} \to 0 $. 
\begin{multline*}
  \|I(f) - I(g)\|_{\mathscr{H}} = M \left[\sum_{k=1}^N(f_k - g_k)Z(\omega,
  \Delta_k)\cdot \sum_{j=1}^N(f_j - g_j)Z(\omega,\Delta_j)\right] = \\ =
  \sum_{k=1}^N
  |f_k - g_k|^2m(\Delta_k) = \int_E |f-g|^2 m(d\lambda) =
  \|f-g\|_{\mathscr{L}_2}.
\end{multline*}
Отсюда $ \|I(f_n) - I(f_m)\|_\mathscr{H} \to 0 $, откуда %TODO: дописать

\begin{theorem}
  Если $ \{\xi_n\} $ --- ССП, то существует ортогональная случайная мера $
  Z_\xi(\omega, \Delta) $, такая что $ \xi_n =
  \int\limits_{-\pi}^{\pi}e^{i\lambda n} \,dG_\xi(\lambda) $, где $ \Delta \in
  \mathscr{E} $, $ E = [-\pi, \pi] $, $ \mathscr{E} $ --- борелевские множества,
 \[
     G_\xi(\Delta) = \int_\Delta d G_\xi(\lambda).
 \]
 --- структурная функция. То есть спектральная функция автоков функции является
 структурной функцией меры $ Z_\xi $.
\end{theorem}

\begin{theorem}
  Если $ \mathscr{H} $ --- пространство из линейной комбинации сечения ССП $
  \{\xi_n\} $ и их ср.кв.пределов  
  \[
    \eta \in \mathscr{H} \implies \exist \Psi(\lambda)\colon \eta =
    \int\limits_{-\pi}^{\pi}\Psi(\lambda)Z_\xi(d\lambda). 
  \]
\end{theorem}
\begin{remark*}
   
 \[
   \int\limits_{-\pi}^{\pi}(e^{i\lambda n} + e^{2i\lambda n})Z_\xi(d\lambda) =
   \xi_{-n} + \xi_n.
 \]
\end{remark*} 

\begin{definition}
  $ \{\Dzeta_n\} $ называется \emph{стационарным линейным преобразованием} $
  \{\xi_n\} $ в спектральной области, если $ \Dzeta_n =
  \int\limits_{-\pi}^{\pi}e^{i\lambda n}\Phi(\lambda)Z_\xi(d\lambda) $.

  $ \Phi(\lambda) $ --- частотная характеристика линейного преобразования, 
  \[
      \int\limits_{-\pi}^{\pi}|\Phi(\lambda)|^2 dG(\lambda) < \infty.
  \]
\end{definition}

\begin{theorem}
  %TODO: дописать

   
 \[
   \Dzeta_n = \int\limits_{-\pi}^{\pi}e^{i\lambda n}\Phi(\lambda)
   Z_\xi(d\lambda),
 \]
 
\end{theorem}
\begin{proof}
  \begin{multline*}
    K_\Dzeta(n+m, m) = \cov(\Dzeta_{n+m}, \Dzeta_m) = \cov \left(
      \int\limits_{-\pi}^{\pi}e^{i\lambda(n+m)}\Phi(\lambda)Z_\xi(d\lambda),
    \int\limits_{-\pi}^{\pi}e^{i\lambda m}\Phi(\lambda) Z_\xi(d\lambda)\right) =
    \\ =
    \int\limits_{-\pi}^{\pi}e^{i\lambda(n+m)} \Phi(\lambda)e^{i\lambda m}
      \Phi(\lambda)\,dG_\xi(\lambda) = \int\limits_{-\pi}^{\pi}e^{i\lambda n}
      |\Phi(\lambda)|^2\,dG_\xi(\lambda)
  \end{multline*}
  не зависит от $ m $. 
  \[
    \Dzeta_n = \int\limits_{-\pi}^{\pi}e^{i\lambda n} Z_\dzeta(d\lambda) =
    \int\limits_{-\pi}^{\pi}e^{i\lambda n}\Phi(\lambda)Z_\xi(d\lambda).
  \]
  
\[
  K_\dzeta(n) = \int\limits_{-\pi}^{\pi}e^{i\lambda n}
  |\Phi(\lambda)|^2\,dG_\xi(\lambda) = \int\limits_{-\pi}^{\pi}e^{i\lambda
  n}\,dG_\xi(\lambda).
\]
\end{proof}

\begin{example}
  Пусть $ \{\varepsilon_n\} $ --- стационарный белый шум. Тогда  
  \[
    \Phi(\lambda) = \sum_{m=-\infty}^\infty e^{-i\lambda m} h(m),
  \]
   
 \[
   h(m) = \frac{1}{2\pi}\int\limits_{-\pi}^{\pi}e^{i\lambda
   m}\Phi(\lambda)\,d\lambda
 \]
  
\[
  \Dzeta_n = \int\limits_{-\pi}^{\pi}e^{i\lambda n}\Phi(\lambda)Z_\xi(d\lambda)
  = \sum_{m=-\infty}^\infty h_m
  \int\limits_{-\pi}^{\pi}e^{i\lambda(n-m)}Z_\xi(d\lambda) =
  \sum_{m=-\infty}^{\infty}h_m\varepsilon_{n-m}
\]
(скользящее среднее). 
 
\[
  g_\varepsilon(\lambda) = \frac{1}{2\pi} \to g_\dzeta(\lambda) =
  |\Phi(\lambda)|^2 \frac{1}{2\pi}
\]
\end{example}

\begin{example}
  $\ARMA(p, q)$.  $ \xi_n - \sum_{k=1}^p b_k \xi_{n-k} = \sum_{k=1}^q a_k
  \varepsilon_{n-k} $, $ n \in \mathbb Z $.

   
 \[
   \xi_n - \sum_{k=1}^p \xi_{n-k} b_k = \int\limits_{-\pi}^{\pi}lr(e^{i\lambda
   n} - \sum_{k=1}^p b_k e^{i\lambda(n-k)})Z_\xi(d\lambda)
 \]
 --- левая часть.
\end{example}

Прогнозирование достаточно лёгкое. У фильтрации сложное практическое применение,
и она решается другим способом.



  \chapter{Лекция 6 - 2023-10-11}

\section{Доверительные интервалы для параметров нормального закона}

$X_1, \dots, X_n \sim N(a, \sigma)$

\begin{theorem}
  Если $X_1, \dots, X_n \sim N(a, \sigma)$, то статистика $\bar X$ и $\dfrac{(n-1) S^2}{\sigma^2} = \dfrac{1}{\sigma^2} \sum\limits_{k=1}^n (X_k-\bar X)^2$ независимы и $\bar X \sim N(a, \dfrac{\sigma}{\sqrt{n}}$, $\dfrac{(n-1) S^2}{\sigma^2} \sim \chi^2(n-1)$
\end{theorem}

\begin{theorem}[Лемма]
  $n = 1, \eta = a \xi, p_\eta(y) = \dfrac{1}{|a|} p_\xi(y/a)$.
  Если $p_{\bar \xi} (\bar x)$ - плотность СВ $\bar\xi$, СВ $\bar\eta = A\xi$, $\exists A^{-1}$
  $p_{\bar\eta} (\bar y) = |det A|^{-1} p_{\bar\xi} (A^{-1} \bar y)$
\end{theorem}
\begin{proof}
  $$p(\bar\xi \in B) = p(A\xi \in AB) = p(\eta \in AB)$$
  $$ p(\xi \in B) = \int_B p_\xi (\bar x) \, d\bar x
  = \left|\, \begin{aligned}
    \bar x &= A^{-1} \bar y \\
    \bar y &= A \bar x \\
    d\bar x &= |det(A^{-1})| d\bar y
  \end{aligned} \,\right| 
  = \int_{AB} p_\xi(A^{-1} \bar y) |det A_{-1}| d\bar y 
  \Rightarrow p_{\bar \eta} (\bar y)
  = p_{\bar\xi} (A^{-1} \bar y) |det A^{-1}|$$
\end{proof}


\begin{proof}[доказательство первой теоремы]
  $X_1, \dots, X_n \rightarrow X_1', \dots, X_n'$
  $$\bar X' = \dfrac{1}{n} \sum\limits_{k=1}^n X_k' = \dfrac{1}{n} \sum \dfrac{(X_k - a)}{\sigma} = \dfrac{\bar X - a}{\sigma}$$
  $$\vec{X}' = (X_1', \dots, X_n')^T$$
  $$p_{\bar X'} (x) = \dfrac{1}{(\sqrt{2\pi})^n} e^{-\dfrac{1}{2} \bar x^T \bar x}$$
  $$\bar Y = C \bar X', C - \text{ортог}$$
  $$Y_1 = \dfrac{1}{\sqrt{n}} \sum X_k' = \sqrt{n} \bar X' = \dfrac{\bar X - a}{\sigma} \sqrt{n}$$
  $$Y_1^2 = n (\bar X')^2$$
  $$p_{\bar Y} (y) = |det C|^{-1} p_{\bar X} (C^{-1} \bar y) = \dfrac{1}{(\sqrt{2\pi})^n} e^{-\dfrac{1}{2} (C^{-1} \bar y)^T C^{-1} \bar y} = \dfrac{1}{(\sqrt{2\pi})^2} e^{-\dfrac{1}{2} \bar y^T \bar y}$$
  
  \begin{multline*}
    ? \bar X и \dfrac{(n-1) S^2}{\sigma^2}; \\
    \dfrac{(n-1) S^2}{\sigma^2} \dfrac{1}{\sigma^2} \sum (X_k-\bar X)^2 = \dfrac{1}{\sigma^2} \sum ((X_k - a) - (\bar X - a))^2 =\\
    \sum \dfrac{(X_k-a)^2}{\sigma^2}-2 \dfrac{\bar X - a}{\sigma^2} \sum (X_k-a) + \dfrac{n}{\sigma^2} (\bar X - a)^2 = \\
    \sum \dfrac{(X_k-a)^2}{\sigma^2} - n \left(\dfrac{\bar X-a}{\sigma}\right)^2  = \sum (X_k')^2 - n (\bar X')^2 = \sum Y_k^2 - Y_1^2 = \\ 
    \sum\limits_{k=2}^n Y_k^2 \sim \chi^2(n-1)
  \end{multline*}
  $$\bar X = \sigma (\bar X' + a) = \sigma \left(\sqrt{\dfrac{Y_1}{n}} + a\right)$$
\end{proof}

\section{Доверительные интервалы}

Для $\sigma$ при неизвестном a:
\begin{align*}
  \dfrac{(n-1) S^2}{ \sigma^2 } &\sim \chi^2(n-1) \\
  \chi^2_{\alpha/2} (n-1) < \dfrac{(n-1) S^2}{\sigma^2}& < \chi^2_{1 - \alpha/2} (n-1) \\
  \sqrt{\dfrac{(n-1) S^2}{\chi^2_{1 - \alpha/2}(n-1)}} < \sigma& < \sqrt{\dfrac{(n-1) S^2}{\chi^2_{\alpha/2} (n-1)}}
\end{align*}

2. Дов. инт для а при неизвестном $\sigma$
\begin{align*}
  \sqrt{n} \dfrac{\bar X - a}{S} &= \sqrt{n} \dfrac{\dfrac{\bar X - a}{\sigma}}{\dfrac{S}{\sigma}} = \sqrt{n} \dfrac{\dfrac{\bar X - a}{\sigma}}{\sqrt{\dfrac{1}{n-1} \dfrac{(n-1) S^2}{\sigma^2}}} \sim t(n-1) \\
  S &= \sqrt{S^2} \\
  -t_{1 - \alpha/2} (n-1) &= t_{\alpha/2} (n-1) < \sqrt{n} \dfrac{\bar X - a}{\sigma}& < t_{1-\alpha/2} (n-1) \\
  \bar X - \dfrac{t_{1-\alpha/2}(n-1) S}{\sqrt{n}} < a& < \bar X + \dfrac{t_{1-\alpha/2}(n-1) S}{\sqrt{n}}
\end{align*}

3. Доверительный интервал для разности мат ожиданий при известной дисперсии
$X_1, \dots, X_n \sim N(a, \sigma_x)$, $Y_1, \dots, Y_m \sim N(b, \sigma_y)$, $\sigma_x, \sigma_y$ - независимы.
? $a-b$

\begin{multline*}
  \dfrac{\bar X - \bar Y - (a-b)}{\sqrt{\dfrac{\sigma_x^2}{n} + \dfrac{\sigma_y^2}{n}}} \sim N(0, 1) \\
  M(\bar X - \bar Y - (a-b)) = (a-b) - (a-b) = 0 \\
  D(\dfrac{\bar X - \bar Y - (a-b)}{\sqrt{\dfrac{\sigma_x^2}{n} + \dfrac{\sigma_y^2}{n}}} = \dfrac{D \bar X + D \bar Y}{\sigma_x^2 / n + \sigma_y^2 / m} = \dfrac{\sigma_x^2 / n + \sigma_y^2 / m}{\sigma_x^2 / n + \sigma_y^2 / m} = 1 \\
  - u_{1-\alpha/2} < \dfrac{\bar X - \bar Y - (a-b)}{\sqrt{\sigma_x^2 / n + \sigma_y^2 / m}} < u_{1-\alpha/2} \\
  \bar X - \bar Y - u_{1-\alpha/2} \sqrt{\sigma_x^2 / n + \sigma_y^2 / m}< a-b < \bar X - \bar Y + u_{1-\alpha/2} \sqrt{\sigma_x^2 / n + \sigma_y^2 / m} 
\end{multline*}

4. Для разности матожиланий при неизв диспресия
$X_1, \dots, X_n \sim N(a, \sigma)$, $Y_1, \dots, Y_m \sim N(b, \sigma)$, $\sigma$ - неизвестн.
\begin{multline*}
  \dfrac{\bar X - \bar Y - (a-b)}{\sigma \sqrt{1/n + 1/m}} \sim N(0, 1) \text{и не зависит от $\dfrac{(n-1)S_x^2}{\sigma^2} + \dfrac{(m_1) S_y^2}{\sigma^2}$} \\
  \dfrac{\bar X - \bar Y - (a-b)}{\sqrt{1/n + 1/m} \sqrt{\dfrac{1}{n+m-2} \dfrac{(n-1) S_x^2 + (m-1)S_y^2}{\sigma^2}}} \sim t(n+m-2)\\
\dfrac{\bar X - \bar Y - (a-b)}{\sqrt{1/n + 1/m}} \sim N(0, 1) \\
\dfrac{(n-1) S_x^2 + (m-1)S_y^2}{\sigma^2} \sim \chi^2(n+m-2) \\
(a-b) \in (\bar X - \bar Y - \delta, \bar X - \bar Y + \delta), где \\
\delta = t_{1-\alpha/2} (n+m-2) \sqrt{\dfrac{n+m}{nm(n+m-2)}} \sqrt{(n-1)S_x^2 + (m-1) S_y^2}
\end{multline*}

5. доверительный интервал для отношения дисперсий
$X_1, \dots, X_n \sim N(a, \sigma_x)$, $Y_1, \dots, Y_m \sim N(b, \sigma_y)$

\begin{multline*}
  \dfrac{\sigma_y^2}{\sigma_x^2} \dfrac{S_x^2}{S_y^2} = \dfrac{(n-1) S_x^2 / \sigma_x^2}{(m-1) S_y^2 / \sigma_y^2} \dfrac{m-1}{n-1} \sim F(n-1, m-1) \\
(m-1) S_y^2 / \sigma_y^2 \sim \chi^2(m-1), (n-1) S_x^2 / \sigma_x^2 \sim \chi^2(n-1) \\
f_{\alpha/2} (n-1, m-1) < \dfrac{\sigma_y^2 s_x^2}{\sigma_x^2 s_y^2} < f_{1-\alpha/2} (n-1, m-1) \\
\dfrac{S_x^2}{S_y^2 f_{1-\alpha/2} (n-1, m-1)} < \dfrac{\sigma_x^2}{\sigma_y^2} < \dfrac{s_x^2}{s_y^2 f_{\alpha/2} (n-1, m-1)}
\end{multline*}

6. Дов интервал для отношения дисперсий при изв a, b

\begin{multline*}
  \dfrac{S_{x0}^2}{\sigma_x^2} = \dfrac{1}{n}\sum \dfrac{(X_k-a)^2}{\sigma_x^2}\sim \chi^2(n) \\
  \dfrac{S_{y0}}{\sigma_y^2} = \dfrac{1}{m} \sum \dfrac{(Y_k - b)^2}{\sigma_y^2} \sim \chi^2 (m)
\end{multline*}

% TODO дописать 6 

  \chapter{Лекция 7 - 2023-10-18}

\[
P(A|D) = \dfrac{P(D|A) P(A)}{P(D)}
\]

\begin{ex}
  Пусть имеется монета с вероятностью выпадения решки x ($x \sim R[0, 1]$).
  $D_{hk} = $ монета подброшена n раз и в k опытах выпала решка.
  
  H_k задают дискретный закон:
  $$P(\xi = x_j | D) = \dfrac{P(D|\xi=x_j) P(\xi=x_j)}{P(D)}$$

  $H_\theta$ - набор 

  $$P(H_\theta | D) = \dfrac{P(D|H_\theta) P(H_\theta)}{P(D)}$$

  Терминология:
  \begin{itemize}
    \item $P(H_\theta |D)$ - апостериорная вероятность ($\dots$)
    \item $P(D|H_\theta)$ - likelihood правдоподобия
  \end{itemize}
\end{ex}

\begin{ex}
  \begin{tabular}{|c|c|c|}
    & успех & неудача \\
    $H_1$ & 1/3 & 2/3 \\
    $H_2$ & 1/2 & 1/2
  \end{tabular}

   % TODO: дописать этот пример
   % в лекциях Облаковой описано всё текстом, а на лекции было много формул

\end{ex}

Если $H_\theta$ - набор абсолютно-непрерывных законов.

\[
  p(x|D) = \dfrac{P(D|x) p(x)}{P(D)} \propto P(D|x) p(x)
\]

$p(x)$ - априорная плотность

$P(D|x)$ - функция правдоподобия

\begin{ex}
  x - вер успеха, $x \sim R[0, 1]$

  $D_{nk}$

  \begin{multline*}
    p(x|D_{nk}) \propto P(D_{nk} | x) p(x) \\
    P(D_{nk} | x) = x^k (1-k)^{n-k} \\
    \sum X_j = k \\
    p(x|D_{nk}) = \dfrac{x^k (1-k)^{n-k}}{B(k+1, n-k+1)} - \text{бета распределение}
  \end{multline*}

  \begin{align*}
    D_{11} &\Rightarrow p(x|D_{11}) = \dfrac{x}{B(2, 1)} = 2x \\
    D_{22} &\Rightarrow p(x|D_{22}) = \dfrac{x^2}{B(3, 1)} = 3x^2 \\
    D_{21} &\Rightarrow p(x|D_{21}) = \dfrac{x (1-x)}{B(2, 2)} = 6x(1-x)
  \end{align*}
\end{ex}

\begin{theorem}[Колмогорова-Блэкуэлла]
  Пусть $X_1, \dots, X_n \sim p(x, \theta)$ и $\bar t = \bar t(X_1, \dots, X_n)$ - достаточная статистика для оценки $\theta$.
  Если $\tilde \theta_n$ - любая несмещенная оценка $\theta$. $\hat \theta_n = M(\tilde \theta_n | \bar t(X_1, \dots, X_n))$ - также несмещенная оценка и $D\hat \theta _n \leqslant D \tilde \theta_n$.
\end{theorem}

Свойства условного МО (как СВ)
$M(\xi|\eta) = M(\xi | \eta = y) |_{y=\eta}$ (то есть рассматриваем как будто y зафиксировано, а потом подставляем вместо y какую-то СВ)
\begin{enumerate}
  \item $MM(\xi | \eta) = M\xi$ 
  \item $M(\xi \cdot g(\eta) | \eta) = g(\eta) \cdot M(\xi)$
  \item в частности $M\left(g(\eta) | \eta\right) = g(\eta)$
  \item $M(\xi - M(\xi | \eta))^2 = \min_g M(\xi - g(\eta))^2$
\end{enumerate}

\begin{proof}
  \begin{multline*}
    M\hat\theta_n = MM(\tilde \theta_n | \bar t(X_1, \dots, X_n)) = M(\tilde \theta_n) = \theta \\
    M\tilde \theta_n 
    = M(\tilde \theta_n - \theta)^2
    = M(\tilde\theta_n - \hat\theta_n + \hat\theta_n - \theta)^2
    = M(\tilde\theta_n - \hat\theta_n)^2 + M(\hat\theta_n-\theta)^2 + 2 M(\tilde\theta_n - \hat\theta_n)(\hat\theta_n - \theta) =\\
    = M(\tilde\theta_n - \hat\theta_n)^2 + D\tilde\theta_n \leqslant 0 \\
  \end{multline*}

  Доказательство того, что последнее слагаемое действительно 0:
  \begin{multline*}
    M((\tilde\theta_n - \hat\theta_n)(\hat\theta_n-\theta) | \bar t(X_1, \dots, X_n))
    = |\text{$(\hat\theta_n - \theta)$ - функция от $\bar t$}| = \\
    = (\hat\theta_n - \theta) \left[ M(\tilde\theta_n|\bar t) - M(\har\theta_n | \bar t) \right]
    = \dots
    = 0
  \end{multline*}
  % TODO возможно нужно получше расписать
\end{proof}

\begin{ex}
  Пусть $X_1, \dots, X_n \sim N(a, 1)$
  $\tilde a_n = X_1$, $M\tilde a_n = MX_1 = a$, $D\tilde a_n = D X_1 = 1$

  $\hat a_n = \bar X$, $M \hat a_n = M\bar X = a$, $D \hat a_n = 1/n$

  Достаточная статистика:
  \begin{multline*}
    \mathcal{L}(X_1, \dots, X_n, a)
    = \dfrac{1}{(\sqrt{2\pi})^n} e^{-\dfrac{1}{2} \sum (X_k - a)^2}
    = \dfrac{1}{(\sqrt{2\pi})^n} e^{-\dfrac{1}{2} (\sum X_k^2 + a \sum X_k - na^2 / 2)} \\
    \bar t = \sum X_k
  \end{multline*}

  \begin{multline*}
    M(X_1 | \sum X_k) = M(X_2 | \sum X_k) = \dots = M(X_n | \sum X_k) \\
    \sum X_k = M(\sum X_k | \sum X_k) = n M(X_1 | \sum X_k) \Rightarrow \\
    \Rightarrow M(X_k) = \dfrac{1}{n} \sum X_k
  \end{multline*}

  \begin{multline*}
    p_{X_1} (x | \bar x = y) = \dfrac{p_{X_1, \bar X} (x, y)}{p_{\bar X} (y)} \\
    (X_1, \bar X) - \text{нормальный вектор} \\
    MX_1 = a, DX_1 = 1, M\bar X = a, D\bar X = \dfrac{1}{n} \\
    \operatorname{cov} (X_1, \bar X) = \dfrac{1}{n} \sum_{k=1}^n \operatorname{cov} (X_1, X_k) = \dfrac{1}{n} (1 + 0 + \dots + 0) = \dfrac{1}{n} \\
    \Sigma = \begin{pmatrix}
      1 & 1/n \\
      1/n & 1/n
    \end{pmatrix},
    |\Sigma| = (n-1) / n^2,
    \Sigma^{-1} = \dfrac{1}{|\Sigma|} \begin{pmatrix}
      1/n & - 1/n \\
      -1/n & 1
    \end{pmatrix} \\
    p_{X_1, \bar X} (x, y)
    = \dfrac{1}{2\pi \sqrt{(n-1) / n^2}} e^{- \dfrac{n^2}{2(n-1)} \left( \dfrac{1}{n}(x-a)^2 - \dfrac{2}{n} (x-a)(y-a) + (y-a)^2 \right) } \\
    p_{X_1} (x | \bar X = y) = \dfrac{N}{N} \sim N() 
    % TODO дописать последнюю строчку
  \end{multline*}
\end{ex}

\section{Эффективные оценки}

Неравенство Рао-Крамера: $D\hat \theta_n \geqslant 1 / (n I_1(\theta))$

\begin{def}
  $e(\hat \theta_n) = \dfrac{1}{n I_1(\theta) D(\hat\theta_n)} \leqslant 1$ - эффективность.
  Оценка называется эффективной, если $e(\hat\theta_n) = 1$
\end{def}

\begin{ex}
  Пусть $X_1, \dots, X_n \sim N(a, \sigma_0)$, $\sigma_0$ - известно.
  
  $\hat a_n = \bar X$, $D\hat\theta_n = \dfrac{\sigma_0^2}{n}$

  \begin{multline*}
    \mathcal{L} (X_1, \dots, X_n, a)
    = \dfrac{1}{(\sqrt{2\pi})^n \sigma_0^n} e^{-\dfrac{1}{2\sigma_0} \sum (X_k-a)^2 } \\
    \ln \mathcal{L} = -\dfrac{n}{2} \ln 2n - n \ln \sigma_0 - \dfrac{1}{2\sigma_0^2} \\
    \dfrac{\partial \ln \mathcal{L}}{\partial a} = - \dfrac{1}{\sigma_0^2}\sum (X_k-a) \\
    \dfrac{\partial \ln \mathcal{L}_1}{\partial a} = - \dfrac{1}{\sigma_0^2} \\
    I_1(a) = M( \dfrac{\partial \ln \mathcal{L}_1}{\partial a}  )^2
    = M \dfrac{1}{\sigma_0^4} (X_1-a)^2
    = \dfrac{1}{\sigma_0^4} \sigma_0^2 = \dfrac{1}{\sigma_0^2} \\
    e(\bar X) = \dfrac{1}{n I_1 D\bar X} = \dots = 1
  \end{multline*}
  $\Rightarrow$ $\bar X$ - эффективная оценка.
\end{ex}

\begin{theorem}[о связи эффективных оценок и достаточных статистик]
  Если $\hat\theta_n = \bar t(X_1, \dots, X_n)$ - несмещенная и эффективная оценка $\theta$, построенная по выборке $X_1, \dots, X_n \sim p(x, \theta)$. $p(x, \theta)$ - удовлетворяет условиям регулярности, то $\bar t(X_1, \dots, X_n)$ - достаточная статистика.
\end{theorem}
\begin{proof}
  $1 = \int_{R^n} p(\bar X, \theta) \, d\bar X$ - можно дифф по параметру.
  $$0 = \int_{R^n} \dfrac{\partial \ln \mathcal{L}}{\partial \theta} p(\bar X, \theta) \, d\bar X$$
  
  Из несмещенности $\theta = \int_{R^n} t(X_1, \dots, X_n) p(\bar X, \theta) \, d\bar x$.
  
  дифф по параметру: $1 = \int_{R_n} t(X_1, \dots, X_n) \dfrac{\partial \ln \mathcal{L}}{\partial \theta} p(\bar X, \theta) \, d\bar x$

  $t(X_1, \dots, X_n) - \theta = \xi, \dfrac{\partial \ln \mathcal{L}}{\partial \theta} = \eta$, $M\xi = M\eta = 0, D\eta = I_n(\theta) = n I_1(\theta), D\xi = D\hat\theta_n$ 

  $1 = \int_{R_n} (t-\theta) \dfrac{\partial \ln \mathcal{L}}{\partial \theta} p(\bar X, \theta) \, d\bar x = cov(\xi, \eta)$

   % TODO поработать над доказательством

   $D\xi D\eta = 1$

  $\dots \Rightarrow \rho_{\xi\eta} = \pm 1, \xi = \alpha\eta \rightarrow D\xi = \alpha^2 D\eta $
  
  $\dots \alpha^2 = \dots = 1 / (n^2 I_1^2(\theta))$

  $t(X_1, \dots, X_n) = \alpha \dfrac{\partial \ln \mathcal{L}}{\partial \theta} = \pm 1 / (n I_1(\theta)) \dfrac{\partial \ln \mathcal{L}}{\partial \theta}$

  $\Rightarrow \dfrac{\partial \ln \mathcal{L}}{\partial \theta} = \pm n I_1(\theta) (t-\theta)$

  $\ln \mathcal{L} = \int n I_1 (\theta) (t-x) d \theta + C(X_1, \dots, X_n)$

  $\dots \Rightarrow$ t - достаточная статистика.
  \begin{multline}
    
  \end{multline}
\end{proof}

\end{document}
