\section{Лекция 1 - 2023-09-06 - Основные распределения МС}
\subsection{Гамма-распределение}
\[
  \gamma_{\alpha, \lambda} (x) = \begin{cases}
    \dfrac{\alpha^\lambda}{\Gamma(\lambda)} x^{\lambda-1} e^{-\alpha x}, x>0 \\
    0, x \leqslant 0
  \end{cases}, \lambda > 0, \alpha > 0.
\]

\subsubsection{Гамма-функция \texorpdfstring{$\Gamma(\lambda)$}{Gamma(lambda)} }
\begin{definition}
  $$\Gamma(\lambda) = \int_0^{+\infty} x^{\lambda-1} e^{-x} \, dx, \lambda > 0.$$
\end{definition}

\begin{itemize}
  \item $\Gamma(1) = \int_0^{+\infty} e^{-x} \, dx = 1$.
  \item $\Gamma(\frac{1}{2}) = \int_0^{+\infty} x^{-1/2} e^{-x} \, dx = 2 \int_0^{+\infty} e^{-x} \, d\sqrt{x} = 2 \int_0^{+\infty} e^{-y^2} \, dy = \sqrt{\pi}$.
  \item $\Gamma(\lambda+1) = \int_0^{+\infty} x^\lambda e^{-x} \, dx = 
    \left. - x^\lambda e^{-x} \right|_0^{+\infty} 
    + \int_0^{+\infty} \lambda x^{\lambda-1} e^{-x} dx 
    = \lambda \Gamma(\lambda)$.

  В частности, $\Gamma(n+1) = n!$. 
\end{itemize}

\begin{remark}
  \[
    \lambda = 1 \Rightarrow \Gamma_{\alpha, 1} = \begin{cases}
      \alpha e^{-\alpha x}, x>0 \\
      0, x\leqslant 0
    \end{cases}
  \]
\end{remark}

\subsection{Бета распределение}

\begin{definition}
  $\beta_{m, n} = \begin{cases}
    \dfrac{x^{m-1}  (1-x)^{n-1}}{B(m, n)}, 0 < x < 1 \\
    0, x \notin (0, 1)
  \end{cases}, m, n > 0$
\end{definition}

\begin{definition}
  $B(m, n) = \int\limits_0^1 w^{m-1} (1-w)^{n-1} dw, m, n>0$.
\end{definition}

\begin{remark}
  Частные случаи:
  \begin{itemize}
    \item $\beta_{1, 1} = \begin{cases}
        \frac{1}{B(1, 1)} = 1, 0<x<1 \\
        0, x\notin (0, 1)
      \end{cases}$ - равномерное распределение на $(0, 1)$.

    \item $\beta_{\frac{1}{2}, \frac{1}{2}} (x) = \begin{cases}
        \dfrac{1}{\pi \sqrt{x(1-x)}}, 0<x<1 \\
        0, x \notin (0, 1)
      \end{cases}$ - закон арксинуса
  \end{itemize}
\end{remark}

\subsubsection{Свойства бета-функции}

\begin{itemize}
  \item $B(m, n) = \int\limits_0^\infty \dfrac{u^{m-1}}{(1+u)^{m+n}} \, du$
    \begin{proof}
      \begin{multline*}        
        \int\limits_0^1 w^{m-1} (1-w)^{n-1} \, dw = \left[
                \begin{aligned}
                  w &= \frac{u}{1+u} \\
                  1-w &= \frac{1}{1+u} \\
                  dw &= \frac{1}{(1+u)^2} du
                \end{aligned}
              \right] = \\
              = \int\limits_0^\infty \dfrac{u^{m-1}}{(1+u)^{m-1} (1+u)^{n-1} (1+u)^2} \, du = \int_0^\infty \dfrac{u^{m-1}}{(1+u)^{m+n}} \, du.
      \end{multline*}
    \end{proof}

  \item $B(m, n) = \dfrac{\Gamma(m) \Gamma(n)}{\Gamma(m+n)}$
    \begin{proof}
      \begin{multline*}
        \Gamma(\lambda) = \int\limits_0^{+\infty} x^{\lambda-1} e^{-x} \, dx =
        \left[ \begin{aligned} x &= (1+u)v \\ dx &= (1+u) dv \end{aligned} \right] =
        \int\limits_0^{+\infty} (1+u)^\lambda v^{\lambda-1} e^{-(1+u)v} \, dv \\
        \Leftrightarrow
        \dfrac{\Gamma(m+n)}{(1+u)^{m+n}} = \int\limits_0^{+\infty} v^{m+n-1} e^{-(1+u)v} \, dv \Leftrightarrow %\\
        % \Leftrightarrow
        \dfrac{\Gamma(m+n) u^{m-1}}{(1+u)^{m+n}} = \int\limits_0^{+\infty} v^{m+n-1} u^{m-1} e^{-(1+u)v} \, dv\\
        \text{интегрируя по u: }
        B(m, n) \Gamma(m+n) = \int\limits_0^{+\infty} \int\limits_0^{+\infty} v^{m+n-1} u^{m-1} e^{-(1+u) v} \, dv \, du = \\
        = \int\limits_0^{+\infty} \int\limits_0^{+\infty} v^{m+n-1} u^{m-1} e^{-v} e^{-uv} \, dv \, du =
        \int\limits_0^{+\infty} v^{n-1} e^{-v} \, dv \int\limits_0^{+\infty} (uv)^{m-1} e^{-uv} \, d(uv) = \\
        = \Gamma(n) \Gamma(m)
      \end{multline*}
    \end{proof}

  \item $B(m+1, n+1) = \dfrac{1}{(m+k+1) C_{m+k}^m}$ - следует из предыдущего.
\end{itemize}

\subsection{Хи-квадрат распределение с n степенями свободы}

\begin{definition}
  \[
    p_{\chi^2(n)} (x) = \begin{cases}
       \dfrac{x^{\frac{n}{2} - 1}}{2^\frac{n}{2} \Gamma(\frac{n}{2})} e^{-\frac{n}{2}}, x>0 \\
       0, x\leqslant 0
    \end{cases}
  \]
\end{definition}

\subsubsection{Свойства хи-квадрат распределения}

\begin{itemize}
  \item $p_{\chi^2(1)} (x) = \begin{cases}
      \frac{1}{\sqrt{2\pi y}} e^{-\frac{y}{2}}, y>0 \\
      0, y\leqslant 0
    \end{cases}$ - совпадает с распределение квадрата стандартной нормально распределенной случайной величины.
    
    \begin{proof}
    
    \end{proof}

  \item Пусть $\xi_i$ независимы и распределены по стандартному нормальному закону. Тогда случайная величина $\eta = \sum_{i=1}^n \xi_i^2$ распределена по закону хи-квадрат с n степенями свободы.
  
    \begin{proof}
      % TODO proof 
    \end{proof}

  \item $p_{\chi^2(n)} = \gamma_{\frac{1}{2}, \frac{n}{2}} (x)$.
\end{itemize}

\subsection{Распределение Фишера-Снедекора F(m,n)}

\begin{definition}
  \[
    p_{F(m, n)} (x) = \begin{cases}
      \dfrac{m^\frac{m}{2} n^\frac{n}{2} x^{\frac{m}{2}-1}}{(n+mx)^\frac{m+n}{2} B(\frac{m}{2}, \frac{n}{2}}, x>0 \\
      0, x\leqslant 0
    \end{cases}
  \]
\end{definition}

% TODO left part of lection
