\chapter{Лекция 2 - 2023-09-13 - Виды сходимости случайных величин. Предельные теоремы
теории вероятностей. Закон больших чисел}

Пусть на $(\Omega, \cal A, P)$ задана случайная величина $\xi$ и последовательность случайных величин $\xi_1, \xi_2, \dots$.

\section{Сходимость почти наверное}

\begin{definition}
  $\xi_n \toPN \xi$, если $P\left(\lim\limits_{n\to\infty} \xi_n = \xi \right) = 1$.
\end{definition}

\begin{ex}
  Пусть $\Omega = [0, 1]$, $\cal A$ - $\sigma$-алгебра измеримых подмножеств, $P(A) = mes A$.

  Определим $\xi_n(\omega) = \begin{cases} 1, 0\leqslant \omega < \frac{1}{n} \\ 0, \text{иначе} \end{cases}$
  
  Тогда $\forall \omega \neq 0, \xi_n(\omega) \to 0$, то есть $P\left(\lim\limits_{n\to\infty} \xi_n=0\right) = 1$.
\end{ex}

\begin{definition}
  $\xi_n \toPN \xi$, если $\forall \varepsilon > 0: \lim\limits_{n\to\infty} P\left( \omega: \sup\limits_{m\geqslant n} |\xi_n-\xi| > \varepsilon \right) = 0$
\end{definition}

\begin{theorem}
  Эти определения эквивалентны.
\end{theorem}

\begin{proof}
  % TODO: proof
\end{proof}


\section{Сходимость по вероятности}

\begin{definition}
  $\xi_n \toP \xi$, если $\forall \varepsilon > 0, \lim\limits_{n\to\infty} P\left(\omega: |\xi_n-\xi|>\varepsilon\right) = 0$.
\end{definition}

\begin{theorem}
  $\xi_n \toPN \xi \Rightarrow \xi_n \toP \xi$.
\end{theorem}

\begin{proof}
  Очевидно.
\end{proof}

\begin{remark}
  Обратное, вообще говоря, не верно.
\end{remark}

\begin{ex}
  % TODO example
\end{ex}

\begin{theorem}
  Если последовательность $\{ \xi_n \}$ монотонно возрастает (убывает) и $\xi_n \toP \xi$, то $\xi_n \toPN \xi$.
\end{theorem}
\begin{proof}
  % TODO proof
\end{proof}

\begin{theorem}
  Если $\xi_n \toP \xi$, то из последовательности $\{ \xi_n \}$ можно выбрать подпроследовательность $\{ \xi_{n_k} \}$ такую, что $\xi_{n_k} \toPN \xi$.
\end{theorem}
\begin{proof}
  без доказательства.
\end{proof}

\section{Сходимость по распределению (слабая)}

\begin{definition}
  $\xi_n \toD \xi$, или $F_{\xi_n} (x) \Rightarrow F_\xi (x)$, если для любой непрерывной ограниченной функции $g(x)$ выполняется:
  $$\int\limits_{-\infty}^{+\infty} g(x) \, dF_{\xi_n}(x) \xrightarrow[]{n\to\infty} \int\limits_{-\infty}^{+\infty} g(x) \, dF_\xi(x)$$
\end{definition}

\begin{definition}
  $F_{\xi_n} (x) \Rightarrow F_\xi(x)$, если $F_{\xi_n}(x) \to F_\xi(x)$ в каждой точке непрерывности $F_\xi(x)$.
\end{definition}

\begin{theorem}
  $P(\xi_n < x) \xrightarrow[]{n\to\infty} P(\xi<x)$.
\end{theorem}

\section{Сходимость в среднем порядка r}

% TODO

\section{Предельные теоремы}

% TODO
