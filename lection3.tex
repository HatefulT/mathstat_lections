\chapter{Лекция 3 - 2023-09-20}
\section{Мотивация}
Пусть $ X = X_1, X_2, \ldots, X_n $ есть выборка (реализация случайной величины
$ \xi $, причём $ \{\xi_n\} $ независимы) из \textsl{известного}
распределения $ F(x, \theta) $, зависящего от \textsl{неизвестного} параметра $
\theta$.

\textsc{Задача}. Оценить значение параметра $ \theta $.

\begin{definition}
   Назовём функцию от выборки
	 \[
		 \widehat \Theta_n = f\left(x_1, x_2,\dots, x_n\right)
	 \]
	 \emph{оценкой параметра $\Theta$}, или \emph{статистикой}.
\end{definition}

Перечислим некоторые \textbf{свойства оценок}.
\begin{definition}
Оценку $ \widehat \Theta_n $ параметра назовём несмещённой, если 
\[
		\mathsf M \widehat \Theta_n = \theta.
\]
\end{definition}
При этом математическим ожиданием здесь считаем %ничего не понял
\begin{gather*}
  \M f(\bar{x}) = \int\limits_{R_n} f(x_1, x_2, \dots, x_n)
	\prod\limits_{j=1}^{n} p_{x_i} (x_i) \, dx_1 dx_2 \dots dx_n, \\
	f(x_1, \dots, x_n) \toP 0, \qquad n \to \infty.
\end{gather*}

\begin{definition}
	Оценка $ \widehat \Theta_n $ называется \emph{асимптотически несмещённой},
	если  
	\[
		\mathsf M \widehat \Theta_n \to \theta \quad \text{при } n\to\infty.
	\]
\end{definition}

\begin{definition}
	Оценка $\widehat \Theta_n$ называется \emph{состоятельной}, если  
	\[
		\widehat \Theta_n \to \theta \quad \text{при } n\to\infty.
	\]
\end{definition}

\begin{definition}
Наконец, если  
\[
	\widehat \Theta_n \toPN \theta \quad \text{при } n\to\infty,
\]
то оценку $ \widehat \Theta_n $ называют \emph{сильно состоятельной}.
\end{definition}



%Пусть $x_1, \dots, x_n$ - выборка.
%$M X_i = a$.
%$$\hat{a_n} = ?$$

\section{Выборочное среднее}

\begin{definition}
	\emph{Выборочным средним} называют величину
	\[
		\hat a_n = \bar{X} = \frac{1}{n} \sum\limits_{i=1}^{n} X_i.
	\]
\end{definition}
Будем рассматривать $ \hat a_n $ как некоторую оценку математического ожидания
$ \mathsf M X_i = a $ рассматриваемой случайной величины. Обозначим кроме того
$ \mathsf D X_i = \sigma^2 $.

Перечислим и докажем \textbf{свойства выборочного среднего}.
\begin{enumerate}
	\item \textsc{Несмещённость}.
		\[
			\M \bar{X} = \M\left(\frac{1}{n} \sum\limits_{i=1}^{n} x_i\right) =
			\frac{na}{n} = a.
		\]
	\item \textsc{Сильная состоятельность}. 
		Нужно доказать, что
	\[
		\frac{1}{n} \sum_{i=1}^n X_i \toPN \M X_i = a.
	\]
	Это условие согласно усиленному закону больших чисел выполняется тогда и только тогда,
	когда существует и конечно $ \M
	X_i = a < \infty$.
	\item \textsc{Дисперсия}. Дисперсия выборочного среднего стремится к нулю при увеличении $ n $.
		Действительно,  
		\[
			\mathsf D\bar X = \frac{1}{n^2} \sum_{i=1}^n \mathsf D X_i =
			\frac{\sigma^2}{n}.
		\]
		
\end{enumerate}

\section{Выборочная дисперсия (исправленная)}
\begin{definition} 
	\emph{Выборочной дисперсией} будем называть величину
\[ 
	S^2 = \frac{1}{n-1} \sum\limits_{i=1}^{n} \left( X_i-\bar{X}\right)^2.
\]
\end{definition}

Перечислим \textbf{свойства выборочной дисперсии}.
\begin{enumerate}
	\item Математическое ожидание выборочной дисперсии $ \mathsf M S^2 = \sigma^2
		$ равно истинной дисперсии. 
	\begin{multline*}
  \mathsf M S^2 = \frac{1}{n-1} \mathsf M\left[\sum_{i=1}^{n} \left(X_i - a - \left(\bar{X} - a\right)^2\right)^2 \right] = \\
  = \frac{1}{n-1} \sum \mathsf M (X_i-a)^2 - 2 \mathsf M\left[ \left(\bar{X} - a\right) \sum
	(X_i - a) \right] + \mathsf M (\bar{X} -a)^2 = \\
  = \frac{1}{n-1} \left[ n \sigma^2 - 2 n \mathsf M(\bar{X}-a)^2 + n \mathsf M (\bar{X}-a)^2 \right] = \\
  = \frac{1}{n-1} \left[ n \sigma^2 - n \frac{\sigma^2}{n} \right] = \sigma^2.
\end{multline*}
\item \textsc{Состоятельность}. 
\[
	S_n^2 = \frac{1}{n-1} \sum_{i=1}^n (X_i - \bar{X})^2 = \frac{n}{n-1} \left[
\frac{1}{n} \sum_{i=1}^n X_i^2 - \bar{X}^2 \right] \toPN \sigma^2 \quad
\text{при } n \to \infty.
\]
Действительно,
\[
	\frac{n}{n-1} \to 1, \qquad \bar X^2 \to a^2, 
\]
и, наконец,
\[
	\frac{1}{n} \sum_{i=1}^n X^2_i \toPN \sigma^2 + a^2
\]
в силу усиленного закона больших чисел.
\item \textsc{Дисперсия}.
\[
\mathsf D S^2 = \dfrac{\mu_4}{n} - \frac{\sigma^4 (n-3)}{n (n-1)} = O \left(\frac{1}{n}
\right),
\]
где $\mu_4 = \mathsf M X_i^4$ --- четвертый момент.
\end{enumerate}

\begin{theorem}[достаточное условие состоятельности]

Пусть $\widehat\Theta_n$ --- асимптотически несмещённая оценка $\Theta$ и
$\mathsf D \bar\Theta_n \to 0$ при $ n \to \infty$.

Тогда $\widehat\Theta_n$ --- состоятельная оценка $\Theta$.
\end{theorem}
\begin{proof}
	Запишем условие состоятельности оценки $ \widehat\Theta_n $ в следующем виде:
\[
  | \widehat\Theta_n - \Theta| = |\widehat \Theta_n - \mathsf M \widehat\Theta_n
	+ \mathsf M \widehat\Theta_n
	- \Theta| \leqslant
	|\widehat\Theta_n - \mathsf M\widehat\Theta_n| + | \mathsf M\widehat\Theta_n -
	\Theta| \toP 0,
\]
где второе слагаемое стремится к нулю ввиду ассимптотической несмещённости $
\widehat\Theta_n $. Для сколь угодно малого $ \varepsilon > 0 $ имеем кроме того
\[
  P( | \widehat\Theta_n - \mathsf M \widehat\Theta_n| > \varepsilon) \leqslant
	\frac{|\widehat\Theta_n - \mathsf M
	\widehat\Theta_n| ^2} {\varepsilon^2} =  \frac{\mathsf D \widehat\Theta_n}
	{\varepsilon^2} \to 0 \quad \text{при } n \to\infty,
\]
что и доказывает утверждение.

\end{proof}



\section{Выборочная функция распределения}
\begin{definition}
	Назовём \emph{индикатором} функцию
\[
	I(x) = \begin{cases}1, &x>0,\\
	0, &x\leqslant 0.\end{cases}
\]
\end{definition}
Видно, что индикатор непрерывен слева.

\begin{definition}
	Назовём \emph{выборочной функцией распределения} функцию
	\[
		\widehat F_n (x) = \frac{1}{n} \sum_{i=1}^n I(x-X_i).
	\]
\end{definition}

\noindent Назовём некоторые из \textbf{свойств выборочной функции распределения}.
\begin{enumerate}
	\item \textsc{Несмещенность}.
\[
	\mathsf M \widehat F_n (x) = \mathsf M \frac{1}{n} \sum_{i=1}^n I(x-X_i) =
	\frac{1}{n} \sum_{i=1}^n P(X_i < x) = F(x).
\]

\item \textsc{Сильная состоятельность}. Исходя из усиленного закона больших
	чисел, имеем 
	\[
		\hat F_n(x) = \frac{1}{n} \sum_{i=1}^n I(x-X_i) \toPN \M I(x - X_i) = P(X_i
		< x) = F(x),
	\]
	что и требовалось доказать.

\item \textsc{Теорема Гливенко -- Кантелли}.
	\begin{theorem}[Гливенко -- Кантелли] 
		\[
			\sup_x |\widehat F_n (x) - F(x)| \toPN 0.
		\]
	\end{theorem}
\begin{proof}
	б/д
	%TODO: (proof)

\end{proof}

\item \textsc{Неравенство Дворецкого -- Кифера -- Волфовица}.

\[
	P(\sup_x |\widehat F_n(x) - F(x)| > \varepsilon) \leqslant 2 e^{-2n
	\varepsilon^2}.
\]
\begin{proof}
	б/д
	%TODO: (proof)

\end{proof}
\begin{corollary*} 
	Возьмём для некоторого $ \alpha > 0 $
\[
		\varepsilon = \sqrt{ \frac{\ln 2/\alpha }{2n}}.
	\]
	Тогда
	\[
		P\left(\sup_x |\widehat F_n (x) - F(x)| \leqslant \varepsilon\right)
		\geqslant 1 - 2 e^{-2n \varepsilon^2} = 1-\alpha.
	\]
\end{corollary*}
Таким образом, с вероятностью $ 1 - \alpha $ имеем
\[
		\widehat F_n (x) - \varepsilon \leqslant F(x) \leqslant \widehat F_n(x) +
		\varepsilon,
\]
или более точно,
\begin{gather*}
	L(x) \leqslant F(x) \leqslant R(x), \quad \text{где}\\
		L(x) = \max \{ \widehat F_n(x) - \varepsilon,\, 0 \}, \qquad R(x) =
		\min \{ \widehat F_n(x) + \varepsilon, \,1 \}.
	\end{gather*}

\item \textsc{Общие свойства}:
	\begin{enumerate}
	\item Пределы  выборочной функции распределения при $ x \to \pm\infty $ равны
		соответственно
		\[
		\widehat F_n(+\infty) = 1, \qquad \widehat F_n(-\infty) = 0.
	\]
	\item Выборосная функция функция распределения $\widehat F_n(x)$ не убывает.
	\item Из первых двух свойств легко получаем 
	\[
			0 \leqslant \widehat F_n(x) \leqslant 1.
	\]
	
	\end{enumerate}
\end{enumerate}

\begin{theorem} Если выборка $x_1, \dots, x_n$ получена из закона распределения
	$F(x)$, то $\widehat F_n(x)$ есть дискретная случайная величина с распределением
\[
	P(\widehat F_n(x) = k/n) = C_n^k F(x)^k (1-F(x))^{n-k} 
\]
\end{theorem}
\begin{proof} 
	Из определения выборочной функции распределения
\[
	\widehat F_n \left(x\right) = \frac{1}{n} \sum_{i=1}^n I\left(x-X_i\right)
\]
сразу виден закон распределения Бернулли с вероятностью успеха $ p = F(x) $. 

Действительно, при каждом $ x \in \mathbb R $ индикаторы $ I(x - X_k) $ ---
независимые случайные величины с распределением 
\begin{align*}
	P(I(x-X_k) = 1) &= F(x) = p, \\
		P(I(x-X_k) = 0) &= 1 - F(x) = q.
\end{align*}

\end{proof}

\begin{corollary}[несмещённость]
	\[
		\mathsf M \widehat F_n (x) = \frac{np}{n} = p = F(x).
	\]
\end{corollary}
\begin{corollary}
	Если $ X_1, X_2, \ldots $ --- выборка неограниченного объёма, то 
	\[
			\widehat F_n(x) \toP F(x).
	\]
\end{corollary}
	\begin{proof}
	\begin{multline*}
		\mathsf D \widehat F_n (x) = \frac{npq}{n^2} = \frac{F(x) (1 - F(x))}{n}
		\Leftrightarrow\\\Leftrightarrow
 \mathsf M (\widehat F_n(x) - F(x))^2 \to 0\Leftrightarrow\\
\Leftrightarrow \widehat F_n (x) \to F(x) \quad \text{(среднеквадратично при $
n\to\infty $)}.
\end{multline*}
Отсюда и $ \widehat F_n(x) \toP F(x) $.

\end{proof}
\setcounter{corollary}{0}


\section{Порядковая статистика}
\begin{definition}
	Выборка, упорядоченная по возрастанию
	\[
		X_{(1)} \leqslant \dots \leqslant X_{(n)},
	\]
	 называется \emph{вариационным рядом}.
\end{definition}
\begin{definition}
	Член  вариационного ряда $ X_{(k)} $ называют \emph{$ k $-ой порядковой статистикой}.
\end{definition}

\begin{definition}
Назовём \emph{размахом выборки} число
\[
	\omega = X_{(n)} - X_{(1)}.
\]
\end{definition}

\begin{theorem}
	Если независимая выборка взята из генеральной совокупности с
функцией распределения $F(x)$, то функции распределения крайних членов
вариационного ряда и их совместная функция распределения имеют вид 
\begin{align*}
	\text{1. }F_{X_{(n)}}(x) &= [F(x)]^n,\\
	\text{2. }F_{X_{(1)}}(x) &= 1 - [1 - F(x)]^n,\\
	\text{3. }F_{X_{(1)}, X_{(n)}} (x, y) &= [F(y)]^n - [F(y)-F(x)]^n, \quad x < y.
\end{align*}
\end{theorem}
\begin{proof} Проведём доказательсто по всем пунктам.
	\begin{enumerate}
		\item Из независимости случайных величин и равенства $ X_{(n)} = \max\limits_i X_i $ вытекает соотношение
	\[
		F_{X_{(n)}}(x) = P \left( X_{(n)} < x \right) = P \left( \bigcap_{k=1}^n (X_k <
		x)\right) = \prod_{k=1}^n P(X_k < x) = [F(x)]^n.
	\]
\item Аналогично для $ X_{(1)} $ получим 
\[
	F_{X_{(1)}}(x) = 1 - P(X_{(1)} \geqslant x) = 1 - \prod_{k=1}^n P(X_k \geqslant
	x) = 1 - [1- F(x)]^n.
\]
\item В последнем случае 
\begin{multline*}
	F_{X_{(1)}, X_{(n)}}(x,y) = P(X_{(1)} < x, \, X_{(n)} < y) = \\ = P(X_{(n)} < y ) -
	P\left(X_{(1)} \geqslant x, \, X_{(n)} < y \right) = \\ =
	[F(y)]^n - P \left( \bigcap_{k=1}^n (x \leqslant X_k < y ) \right)  = \\ =
	[F(y)]^n - \prod_{k=1}^n P(x \leqslant X_k < y) = \\ =
	[F(y)]^n - [F(y) - F(x)]^n, \quad x < y.
\end{multline*}
\end{enumerate}
\end{proof}
\begin{corollary*}
	Если выборка взята из абсолютно непрерывного закона $F(x)$ с 
плотностью $p(x)$, то плотности распределения крайних членов вариационного ряда и их 
совместная плотность имеют вид 
\begin{align*}
	p_{X_{(n)}}(x) &= n[F(x)]^{n-1}p(x),\\
	p_{X_{(1)}}(x) &= n[1-F(x)]^{n-1}p(x),\\
	p_{X_{(1)}, X_{(n)}}(x, y) &= n(n-1)[F(y)-F(x)]^{n-2}p(x)p(y), \quad x < y.
\end{align*} %TODO: как получить последнее? (вписать формулу)
\end{corollary*}

\begin{theorem}
	Если независимая выборка взята из генеральной совокупности с
плотностью распределения $p(x)$, то плотность распределения $k$-ой порядковой
статистики имеет вид 
\[
	p_{X_{(k)}}(x) = n C^{k-1}_{n-1} \left[ F(x) \right]^{k-1} [1 - F(x)]^{n-k}
	p(x).
\]
\end{theorem}
\begin{proof}
	%TODO: поменять доказательство с инженерного на математическое (см. книгу
	%<<Порядковые статистики>>)
Зафиксируем $ x \in \mathbb R $ и выберем столь малое $ \Delta x $, что в
промежуток $ [x, x+ \Delta x) $ может попасть только один элемент выборки (закон
абсолютно непрерывный!). Тогда согласно полиномиальной схеме событие 
$(x \leqslant X_{(k)} < x + \Delta x)$ означает, что какие-то $ k - 1 $ элементов
выборки попали в промежуток $ (-\infty, x) $, 
один элемент в промежуток $[x, x + \Delta x)$, а остальные в $[x + \Delta x,
+\infty)$. Поскольку вероятности 
попаданий в эти множества равны $ F(x) $, $ p(x)\Delta x $ и $(1 - F(x + \Delta
x))$ соответственно, то  
\begin{multline*}
	F_{X_{(k)}}(x+\Delta x) - F_{X_{(k)}} (x) = P(x \leqslant X_{(k)} < x + \Delta
	x) = \\ =
	\frac{n!}{(k-1)!1!(n-k)!} [F(x)]^{k-1} p(x) \Delta x [1 - F(x + \Delta
	x)]^{n-k} = \\ =
	n C^{k-1}_{n-1} [F(x)]^{k-1} [1 - F(x + \Delta x)]^{n-k} p(x)\Delta x.
\end{multline*}
Завершим доказательство делением на $ \Delta x $ и переходом к пределу при $
\Delta x \to 0 $.

\end{proof}



\section{Примеры}
\begin{ex}
	Пусть дана выборка объёма $ n $ из показательного закона с параметром $ \alpha
	$. Тогда
	\[
		p_{X_{(1)}}(x) = n(1 - (1 - e^{-\alpha x}))^{n-1} \alpha e^{-\alpha
		x} = n \alpha e^{-n\alpha x}.
	\]
\end{ex}
\begin{ex}
	Найдём математическое ожидание и дисперсию $ X_{(k)} $, если выборка получена
	из равномерного распределения на $ [0, a] $. 

	Математическое ожидание минимального члена вариационного ряда равно
	\begin{multline*}
		\mathsf M X_{(1)} = \frac{n}{a }\int\limits_{0}^{a} x\cdot \left( 1 - \frac{x}{a}
		\right)^{n-1}\,dx \to \left[ \begin{aligned} u &= 1 - x/a \\ dx &= -a\,du
		\end{aligned}\right] \to an \int\limits_{0}^{1} \left( 1 - u \right) \cdot
		u^{n-1}\,du =\\=
		\frac{an}{n+1} \left. \left( 1 - \frac{x}{a} \right)^{n+1} \right|^a_0 - a
			\left.\left( 1 - \frac{x}{a} \right)^n \right|^a_0 = - \frac{an}{n+1} + a
				= \frac{a}{n+1}.
	\end{multline*}
	
	Математическое ожидание максимального члена вариационного ряда равно 
	\[
	\mathsf M X_{(n)} = \frac{n}{a}\int\limits_{0}^{a} x
	\left(\frac{x}{a}\right)^{n-1}\,dx = \frac{an}{n+1} \left.\left( \frac{x}{a}
	\right)^{n+1} \right|^a_0 = \frac{an}{n+1}.
	\]
	Отсюда видно, что $ X_{(n)} $ --- асимптотически несмещённая оценка параметра
	$ a $.

	Наконец, 
	\begin{multline*}
		\mathsf M X_{(k)} = \frac{n}{a}\int\limits_{0}^{a} x\cdot C^{k-1}_{n-1}
		\left(\frac{x}{a}\right)^{k-1} \left( 1 - \frac{x}{a} \right)^{n-k}\,dx = \\
		= an C^{k-1}_{n-1} \int\limits_{0}^{1}
		u^k(1-u)^{n-k}\,du = 
		an C^{k-1}_{n-1} B(k+1, n-k+1)  = \\ = a \frac{(n-1)!n}{(k-1)! (n-k)!}
		\frac{\Gamma(k+1) \Gamma(n-k+1)}{\Gamma(n+2)} = \frac{ak}{n+1},
	\end{multline*}
	где $ u = x/a $.
\end{ex}
