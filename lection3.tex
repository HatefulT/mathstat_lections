\chapter{Лекция 3 - 2023-09-20}
  
\begin{definition}
   $\widehat{\Theta_n} = f\left(x_1, x_2,\dots, x_n\right)$ - оценка параметра $\Theta$ (функция от выборки, статистика).
\end{definition}

%\begin{definition}
%  Оценка параметра $\widehat{D_n} = f(x_1, x_2, \dots, x_n)$.
%\end{definition}

\section{Свойства оценок}

\begin{definition}
  Оценка $\widehat{D_n}$ называется несмещенной, если $M[\widehat{D_n}]$.
\end{definition}

\[
  M f(\bar{x}) = \int\limits_{R_n} f(x_1, x_2, \dots, x_n) \prod\limits_{j=1}^{n} p_{x_i} (x_i) \, dx_1 dx_2 \dots dx_n.
\]

$$f(x_1, \dots, x_n) \toP 0, n \to \infty$$

Если $f(x_1, \dots, x_n) \toPN 0$, то $\hat{D_n}$ - силнльно состоятельная.

%Пусть $x_1, \dots, x_n$ - выборка.
%$M X_i = a$.
%$$\hat{a_n} = ?$$

\section{Выборочное среднее}

\begin{definition}
  Выборочное среднее $\hat{a_n} = \bar{X} = \frac{1}{n} \sum\limits_{i=1}^{n} x_i$.
\end{definition}

\subsection{Свойства выборочного среднего}

\begin{itemize}
  \item $M \bar{X} = M(\frac{1}{n} \sum\limits_{i=1}^{n} x_i) = \dots$. Несмещенная
  \item сильно состоятельная.
\end{itemize}

\section{Выборочная дисперсия}

$$ S^2 = \frac{1}{n-1} \sum\limits_{i=1}^{n} \left(x_I-\bar{x}\right)^2 $$

\begin{multline}
  M S^2 = \frac{1}{n-1} M(\sum_{i=1}^{n} (X_i - a - (\bar{X} - a)^2)^2 ) = \\
  = \frac{1}{n-1} \sum M (X_i-a)^2 - 2 M[ (\bar{X} - a) \sum (x_i - a) ] + M (\bar{x} -a)^2 = \\
  = \frac{1}{n-1} \left[ n \sigma^2 - 2 n M(\bar{X}-a)^2 + n M (\bar{x}-a)^2 \right] = \\
  = \frac{1}{n-1} \left[ n \sigma^2 - n \frac{\sigma^2}{n} \right] = \sigma^2.
\end{multline}

2. Состоятельность 

\begin{multline*}
  S_n^2 = \frac{1}{n-1} \sum (X_i - \bar{X})^2 = \frac{n}{n-1} \left[ \frac{1}{n} \sum X_i^2 - \bar{X}^2 \right] = \\
  \text{В силу сильного ЗБЧ} \\
  \to \sigma^2
\end{multline*}

3. Дисперсия

 
$$D S^2 = \dfrac{\mu_4}{n} - \dfrac{\sigma^4 (n-3)}{n (n-1)} = O \left(\frac{1}{n} \right)$$
$\mu_4 = M X_i^4$ - четвертый момент

4. Теорема достаточное условие состоятельности

Пусть $\hat{\Theta_n}$ асимптотически несмешщенная оценка $\Theta$ и $D \bar{\Theta_n} \to 0, n \to \infty$, тогда $\hat{\Theta_n}$ - состоятельная оценка $\Theta$.

Доказательство
\begin{multline*}
  \forall \epsilon > 0: | \hat{\Theta_n} - \Theta| = |\Theta_n - M \Theta_n + M \Theta_n - \Theta| <= |\Theta_n - M\Theta_n| + | \Theta - M \Theta_n | \\
  P( | \Theta_n - M \Theta_n| > \epsilon) \leqslant \dfrac{|\Theta_n - M \Theta_n| ^2} {\epsilon^2} =  \dfrac{D \Theta_n} {\epsilon^2} \to 0, n \to \infty
\end{multline*}

\section{Выборочная функция распределения}

\begin{definition}
\[
  I(x) = 1, x>0, I(x) = 0 = x<=0.
\]
\end{definition}

Индикатор непрерывен слева.

\begin{definition}
  Выборочная функция распределения $\hat{F_n} (x) = \frac{1}{n} \sum I(x-X_i)$
\end{definition}

\subsection{Свойства выборочной функции распределения}

1. Несмещенность
\begin{multline*}
  M \hat{F_n} (x) = M \dfrac{1}{n} \sum I(x-X_i)
\end{multline*}

2. сильно состоянельна

3.  Теорема гаовенко-канослли

\begin{multline*}
  sup_x |\hat{F_n} (x) - F(x)| \toPN 0.
\end{multline*}

4. Неравенство Дворецного-Кифери-Волфовица

$$P(sup_x |\hat{F_n}(x) - F(x)| > \epsilon) \leqslant 2 e^{-2n \epsilon^2}$$

Следствие. $P\left(sup_x |\hat{F_n} (x) - F(x)| <= \epsilon\right) >= 1 - 2 e^{-2n \epsilon^2} = 1-\alpha$.
% $\epsilon = \sqrt{ \dfrac{ln \frac{2}{\alpha} }}{2n}$.

$$\hat F_n$$

5. $$0 \leqslant \hat F_n (x) \leqslant 1$$

$$\lim_{x\to -\infty} \hat{F_n} (x) = 0$$

$$F_n(+\infty) = 1$$

$F_n(x)$ неубывает

Теорема. Если выборка $x_1, \dots, x_n$ получена из закона распределения $F(x)$, то $F_n(x)$ - .. СВ

\begin{multline*}
  P(F_n(x) = k/n) = C_n^k F(x)^k (1-F(x))^{n-k}
\end{multline*}

схема бернулли

Доказательство. 
\[
  \hat{F_n} \left(x\right) = \frac{1}{n} \sum I\left(x-X_i\right)
\]

Следствие
1. $M \hat F_n (x) = \frac{1}{n} np = p = F(x)$
2. $D \hat F_n (x) = \frac{1}{n^2} npq = \dfrac{F(X) (1 - F(x))}{n}$
$\Leftrightarrow M (\hat F_n(x) - F(x))^2 \to 0$
$\Leftrightarrow \hat F_n (x) \to^\text{ср.кв.} F(x)$

4. Порядковая статистика

$x_1, x_2, \dots, x_n$

\begin{definition}
  Выборка упорядоченная по возрастанию называется вариационным рядом $X_{(1)} \leqslant \dots X_{(n)}$.
\end{definition}

$min (X) = X_{(1)}$ - минимальный член вариационного ряда.

$\omega = X_{(n)} - X_{(1)}$ - размах выборки
