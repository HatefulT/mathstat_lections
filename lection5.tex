\chapter{Лекция 5 - 2023-10-04 - Достаточные статистики. Критерий факторизации}
\section{Достаточные статистики}
Пусть $ X_1, X_2, \ldots, X_n $ --- выборка из распределения, зависящего от
параметра $ \bar \theta $ (реализация случайного вектора $ \bar \xi = (\xi_1,
\xi_2, \ldots, \xi_n) $ с независимыми компонентами).

\begin{definition}
  Вектор-функция $\bar t = \bar t(X_1, X_2, \dots, X_n)$ называется \emph{достаточной
	статистикой} для оценки параметра $\bar{\theta}$, если условная функция
	распределения
	\[
		F_{\bar \xi} (X_1, X_2, \dots, X_n\, | \,\bar{t} (X_1, \dots, X_n)
		= \bar t)
	\]
	не зависит от $\bar\theta$ при любых значениях $\bar t$. 
\end{definition}

\begin{remark*}
Статистика $ \bar t $ является достаточной для оценки параметра $ \theta $, если
\begin{enumerate}
	\item \textsc{Дискретный случай}. Условные вероятности 
	\[
			P(\xi_1 = X_1, \xi_2 = X_2, \ldots, \xi_n = X_n\, \mid \, \bar t(\bar \xi) = \bar
			t)
	\]
	не зависят от $ \theta $ при любых значениях $ \bar t $.
\item \textsc{Непрерывный случай}. Условные плотности  
\[
	p_{\bar \xi} (X_1, X_2, \ldots, X_n\, |\,\bar t(\bar \xi) = \bar t)
\]
не зависят от $ \theta $ при любых значениях $ \bar t $.
\end{enumerate}

\end{remark*}

\begin{ex}
  Пусть $X_1, X_2, \dots, X_n$ распределены по Бернулли с вероятностью успеха $p$
	(последовательность 0 и 1).

	Докажем, что статистика $\sum_{k=1}^n X_k$ достаточна для оценки параметра $p$.
Найдём условные вероятности
  \begin{multline*}
		P\left(\xi_1 = X_1, \dots, \xi_n = X_n\,\Big|\, \sum_{k=1}^n \xi_k =
		t\right) = \\
		= \frac{P(\xi_1 = X_1, \dots \xi_n = X_n, \sum \xi_k = t)}{P\left(\sum_{k=1}^n \xi_k = t\right)} = \\
    = \begin{cases}
			p = 0, &t \neq \sum_{k=1}^n X_k, \\
      \frac{ p^t (1-p)^{n-t} }{ C_n^t p^t (1-p)^{n-t} } = \frac{1}{C_n^t}, &t
			= \sum_{k=1}^n X_k.
    \end{cases}
  \end{multline*}
	Действительно, знаменатель по теореме Бернулли равен $ C^t_n p^t(1-p)^{n-1} $.
	Числитель равен нулю, если $ t \ne \sum_{k=1}^n $ и $ p^t(1-p)^{n-t} $ в ином
	случае.

	Таким образом, условная вероятность в любом случае не зависит от $ p $.
\end{ex}

\begin{theorem}[критерий факторизации]
  Если $X_1, X_2, \dots, X_n \sim F(x, \bar\theta)$, то $\bar t = \bar t (X_1,
	\dots, X_n)$ является достаточной статистикой для оценки $\bar\theta$ тогда и
	только тогда, когда 
  \[
    \mathscr{L}(X_1, X_2, \dots, X_n, \bar\theta) = h(X_1, \dots, X_n)
		g(\bar\theta, \bar t (X_1, X_2, \dots, X_n)),
  \]
  где $h$ не зависит от $\theta$, g - не зависит от $X_1, X_2, \dots, X_n$
	(только через $\bar t$)
\end{theorem}

\begin{proof}
	Проведём доказательство в обе стороны.
	\begin{enumerate}
		\item $\boxed{\Leftarrow}$
  \begin{multline*}
    P(\xi_1 = X-1, \dots, \xi_n = X_n | \bar t (\xi_1, \dots, \xi_n) = \bar t)
    = \dfrac{P(\xi_1=X_1, \dots, \xi_n = X_n, \bar t(X_1, \dots, X_n) = \bar
		t_0)}{P(\bar t(\xi_1, \dots, \xi_n) = \bar t_0)} = \\
    = \begin{cases}
      0, &\bar t(X_1, \dots, X_n) \neq \bar t_0 \\
      \dfrac{p(\xi_1 = X_1, \dots, \xi_n = X_n)}{\sum_{\bar t(Z_1, \dots, Z_n) =
	\bar t_0} P(\xi_1 = Z_1, \dots, \xi_n = Z_n)}, &\bar t(X_1, \dots, X_n) = \bar t_0
    \end{cases} \stackrel{\bar t(X_1, \dots, X_n) = \bar t_0}{=} \\
    \stackrel{\bar t(X_1, \dots, X_n) = \bar t_0}{=} \dfrac{h(X_1, \dots, X_n)
		g(\bar \theta, \bar t(X_1, \dots, X_n))}{\sum h(Z_1, \dots, Z_n)
		g(\bar\theta, \bar t(Z_1, \dots)} = \dfrac{h(X_1, \dots)}{\sum h(Z_1,
		\dots)}.
  \end{multline*}

\item $\boxed{\Rightarrow}$
  Доказываем, что если $P(\xi_1 = X_1, \dots, \xi_n = X_n | \bar t(\xi_1, \dots)
	= \bar t)$ - не зависит от $\bar \theta$, то $\mathscr{L}$ факторизуется.
  \begin{equation*}
    \mathscr{L} (X_1, \dots, X_n, \bar\theta) = P_{\bar\theta} (\xi_1 = X_1,
		\dots, \xi_n = X_n) = P_{\bar\theta} (\xi_1 = X_1, \dots, \xi_n = X_n | \bar
		t(\bar\xi) = \bar t) \cdot P(\bar t(\bar\xi) = \bar t)
  \end{equation*}
  Первый множитель не зависит от $\bar\theta$, второй зависит от $X_1, \dots,
	X_n$ только через $\bar t(X_1, \dots)$
\end{proof}

\begin{remark}
  \[
    \mathscr{L} = h(X_1, \dots, X_n) g(\bar\theta, \bar t(X_1, \dots, X_n)
  \]
  \[
    \ln \mathscr{L} = \ln h(X_1, \dots, X_n) + \ln g(\dots)
  \]
  \[
    \dfrac{\partial \mathscr{L} (\bar\theta, \bar t(X_1, \dots, X_n))}{\partial \bar\theta}  
  \]

  $\Rightarrow$ ОМП функция от $\bar t(X_1, \dots, X_n)$

  % TODO не понял о чём это замечание, скорее всего ошибка
\end{remark}

\begin{ex}
  Пусть $X_1, \dots X_n \sim Pois(\lambda)$, 

  \[
    P(X_1, \dots, X_n, \lambda) = P_\lambda (\xi_1=X_1, \dots, \xi_n=X_n) = \prod P(\xi_k = X_k) = \prod \dfrac{\lambda^{X_k}}{(X_k)!)} e^{-\lambda} = \dfrac{1}{\prod (X_k)!} \lambda^{\sum X_k} e^{-n\lambda}
  \]

  $\Rightarrow$ $\sum X_k$ - достаточная статистика для оценки $\lambda$ ($\bar x$ - тоже)
\end{ex}

\begin{ex}
  $X_1, \dots, X_n \sim N(a, \sigma)$

  \[
    \mathscr{L} (X_1, \dots, X_n, a, \sigma) = \prod \dfrac{1}{\sqrt{2\pi} \sigma} e^{-\dfrac{(X_k-a)^2}{2\sigma^2}} = \dfrac{1}{\sqrt{2\pi}^n} \dfrac{1}{\sigma^n} e^{-\dfrac{1}{2\sigma^2} \sum (X_k-a)^2} = \dfrac{1}{\sqrt{2\pi}^n} \dfrac{1}{\sigma^n} e^{-\dfrac{1}{2\sigma^2} (\sum X_k^2 - 2a \sum X_k + na^2)}
  \]

  $(\sum X_k^2, \sum X_k)$ - достаточная статистика для $(a, \sigma)$

  $\sum (X_k-a)^2 = \sum (X_k - \bar X + \bar X - a)^2 = \sum (X_k-\bar X)^2 + (\bar X - a)^2 n$
  $(\bar X, S^2 = \dfrac{1}{n-1} \sum (X_k-\bar X)^2)$ - достаточная статистика для $\sigma$ при известном $a$.
\end{ex}

\begin{ex}
  Пусть $X_1, \dots, X_n \sim Par(\alpha, \theta)$,
  $p(x, \theta, \alpha) = \dfrac{\alpha}{\theta} \left(\dfrac{\theta}{x}\right)^{\alpha+1}, x\geqslant 0$

  \[
    \mathscr{L} (X_1, \dots, X_n, \alpha, \theta) = \prod \dfrac{\alpha}{\theta} (\dfrac{\theta}{x})^{\alpha+1} I(X_k \geqslant 0) = \dfrac{\alpha^n \theta^{\alpha n}}{(\prod X_k)^{\alpha+1}} I(X_{(1)}\geqslant 0) \cdot 1
  \]

  $(X_{(1)}, \prod X_k)$ - достаточная статистика
\end{ex}

\section{Интервальные оценки}

Пусть $X_1, \dots, X_n$ - выборка из распределения с плотностью $p(x, \theta)$; $1-\alpha$ - уровень доверия

\begin{definition}
  Если $P(\theta \in (\underline{t} = t(X_1, \dots, X_n), \bar t(X_1, \dots, X_n)) = 1-\alpha$, то $(\underline{t}, \bar t)$ - доверительный интервал.
\end{definition}

\subsection{Принцип построения доверительных интервалов}

\begin{enumerate}
  \item Находим статистику $\eta = \varphi(X_1, \dots, X_n, \theta)$, закон распределения $F_\eta (x)$ которой известен точно или приближенно и не зависит от $\theta$.
  \item Находим квантили $t_{\alpha/2}$ и $t_{1-\alpha/2}$, т.е.
    $$ F_\eta(t_{\alpha/2})  = \alpha/2, F_\eta(t_{1-\alpha/2}) = 1 - \alpha/2$$

    \[
      P(t_{\alpha/2} < \varphi(X_1, \dots, X_n, \theta) < t_{1-{\alpha/2}}) = F_\eta (t_{1 - \alpha/2}) - F_\eta (t_{\alpha/2}) = 1- \alpha
    \]
  \item Разрешение неравенства относительно $\theta$
    $$P(\underline{t} < \theta < \bar t) = 1-\alpha$$
\end{enumerate}

\begin{ex}
  Пусть $X_1, \dots, X_n \sim N(a, \sigma_0)$, где $\sigma_0$ - известна.

  Тогда по методу моментов $\hat a = \bar x$

  \[
    \bar x = \dfrac{1}{n} \sum X_k \sim N(a, \dfrac{\sigma_0^2}{\sqrt{n}})
  \]

  $\dfrac{\bar x - a}{\sigma_0} \sqrt{n} \sim N(0, 1)$ - распределение не зависит от параметра a

  \begin{align*}
    -u_{1-\alpha/2} = u{\alpha/2} < &\dfrac{\bar x - a}{\sigma_0} \sqrt{n} < u_{1-\alpha/2} \\
    \bar x - \dfrac{u_{1-\alpha/2}}{\sqrt{n}} < &a < \bar x + \dfrac{u_{1-\alpha/2}}{\sqrt{n}}
  \end{align*} 
 \end{ex}

\begin{ex}
  Пусть $X_1, \dots, X_n \sim N(a_0, \sigma)$, $a_0$ - известна.

  $\mathscr{L} (X_1, \dots, X_n, \sigma) = \prod\limits_{k=1}^n \dfrac{1}{\sqrt{2\pi} \sigma} e^{-\dfrac{(X_k - a)^2}{2\sigma^2}} = \dfrac{1}{(\sqrt{2\pi})^n} \dfrac{1}{\sigma^n} e^{-\dfrac{1}{2\sigma^2} \sum_{k=1}^n (X_k-a)^2}$

  Следовательно, $\sum (X_k-a)^2$ - достаточная статистика

  $\sum\limits_{k=1}^n \dfrac{(X_k-a)^2}{\sigma^2} \sim \chi^2 (n)$ ()

  \begin{gather*} 
    \chi^2_{\alpha/2} (n) < \sum \dfrac{(X_k - a_0)^2}{\sigma^2} < \chi^2_{1 - \alpha/2} (n) \\
    \dfrac{\sum (X_k-a_0)^2}{\chi^2_{1 - \alpha/2} (n)} < \sigma^2 < \dfrac{\sum (X_k - a_0)^2}{\chi^2_{\alpha/2}(n)} 
  \end{gather*}

  $S_0^2 = \frac{1}{n} \sum (X_k-a_0)^2$ % TODO чтото тут надо дописать
\end{ex}

\begin{ex}
  Произведено $n=36$ измерений, $\bar X = 9.3 \text{кОм},  \sigma_0 = 2 \text{кОм}$. Найти доверительный интервал $1- \alpha = 0.95$ для оценки параметра a.

\[
  0.95 = P\left(\bar X - \dfrac{u_{0.975} \cdot 2}{\sqrt{36}} < a < \bar X + \dfrac{u_{0.975} \cdot 2}{\sqrt{36}}\right) \Leftrightarrow a \in (8.65, 9.95)
\]
\end{ex}

\begin{ex}
  $X_1, \dots, X_n \sim E(\lambda)$

  Достаточная статистика: $\hat \lambda = \dfrac{1}{\bar X}$

  $$2\lambda \sum X_k$$

  $X_k \sim \lambda e^{-\lambda x} \Rightarrow 2\lambda X_k \sim \dfrac{1}{2} e^{-\dfrac{x}{2}} \equiv \chi^2 (2)$

  $$\lambda < \dfrac{\chi^2_{1-\alpha/2} (2n)}{2 \sum X_k} = \dfrac{\lambda^2_{1-\alpha/2} (2n)}{2n \bar X}$$
\end{ex}

  Приближенные доверительные интервалы:
  $X_1, \dots, X_n \sim F(x, \theta), M_\theta X_i = a(\theta), D_\theta X_i = d(\theta)$, n - велико. 
  $$\dfrac{\sum X_i - n a(\theta)}{\sqrt{n d(\theta)}} \approx N(0, 1)$$

  $$- u_{1 - \alpha/2} < \dfrac{\sum X_i - n a(\theta)}{\sqrt{n d(\theta)}}< u_{1-\alpha/2}$$

\begin{ex}
  $X_1, \dots, X_n \sim B(k, p)$, k - известно.

  $M X_k = kp, D X_k = k p (1-p)$

  $\dfrac{\sum X_k - n k p}{\sqrt{nkp(1-p)}} \in (-u_{1-\alpha/2}, u_{1-\alpha/2})$

  p под корнем в знаменателе заменяем оценкой $\hat p = \dfrac{\bar X}{k}$ хуй

  $-u_{1-\alpha/2} \sqrt{nk \dfrac{\bar X}{k} (1 - \dfrac{\bar X}{k})} < \sum X_k - nkp < u_{1-\alpha/2} \sqrt{nk \dfrac{\bar X}{k} (1 - \dfrac{\bar X}{k})}$

  $\dfrac{\bar X}{k} - \dfrac{u_{1-\alpha/2} \sqrt{\dfrac{\bar X}{k} (1 - \dfrac{\bar X}{k})}}{\sqrt{nk}} < p < \dfrac{\bar X}{k} + \dfrac{u_{1-\alpha/2} \sqrt{\dfrac{\bar X}{k} (1 - \dfrac{\bar X}{k})}}{\sqrt{nk}}$
\end{ex}
