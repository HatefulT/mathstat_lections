\chapter{Лекция 5 - 2023-10-04 - Достаточные статистики. Критерий факторизации}
\section{Достаточные статистики}
Пусть $ X_1, X_2, \ldots, X_n $ --- выборка из распределения, зависящего от
параметра $ \bar \theta $ (реализация случайного вектора $ \bar \xi = (\xi_1,
\xi_2, \ldots, \xi_n) $ с независимыми компонентами).

\begin{definition}
  Вектор-функция $\bar t = \bar t(X_1, X_2, \dots, X_n)$ называется \emph{достаточной
	статистикой} для оценки параметра $\bar{\theta}$, если условная функция
	распределения
	\[
		F_{\bar \xi} (X_1, X_2, \dots, X_n\, | \,\bar{t} (\xi_1, \dots, \xi_n)
		= \bar t)
	\]
	не зависит от $\bar\theta$ при любых значениях $\bar t$. 
\end{definition}

\begin{remark*}
Статистика $ \bar t $ является достаточной для оценки параметра $ \theta $, если
\begin{enumerate}
	\item \textsc{Дискретный случай}. Условные вероятности 
	\[
			P(\xi_1 = X_1, \xi_2 = X_2, \ldots, \xi_n = X_n\, \mid \, \bar t(\bar \xi) = \bar
			t)
	\]
	не зависят от $ \theta $ при любых значениях $ \bar t $.
\item \textsc{Непрерывный случай}. Условные плотности  
\[
	p_{\bar \xi} (X_1, X_2, \ldots, X_n\, |\,\bar t(\bar \xi) = \bar t)
\]
не зависят от $ \theta $ при любых значениях $ \bar t $.
\end{enumerate}

\end{remark*}

\begin{ex}
  Пусть $X_1, X_2, \dots, X_n$ распределены по Бернулли с вероятностью успеха $p$
	(последовательность 0 и 1).

	Докажем, что статистика $\sum_{k=1}^n X_k$ достаточна для оценки параметра $p$.
Найдём условные вероятности
  \begin{multline*}
		P\left(\xi_1 = X_1, \dots, \xi_n = X_n\,\Big|\, \sum_{k=1}^n \xi_k =
		t\right) = \\
		= \frac{P(\xi_1 = X_1, \dots \xi_n = X_n, \sum \xi_k = t)}{P\left(\sum_{k=1}^n \xi_k = t\right)} = \\
    = \begin{cases}
			p = 0, &t \neq \sum_{k=1}^n X_k, \\
			\frac{ p^t (1-p)^{n-t} }{ C_n^t p^t (1-p)^{n-t} } = \frac{1}{C_n^t}, &t
			= \sum_{k=1}^n X_k.
    \end{cases}
  \end{multline*}
	Действительно, знаменатель по теореме Бернулли равен $ C^t_n p^t(1-p)^{n-1} $.
	Числитель равен нулю, если $ t \ne \sum_{k=1}^n $ и $ p^t(1-p)^{n-t} $ в ином
	случае.

	Таким образом, условная вероятность в любом случае не зависит от $ p $.
\end{ex}

\begin{theorem}[критерий факторизации]
  Если $X_1, X_2, \dots, X_n \sim F(x, \bar\theta)$, то $\bar t = \bar t (X_1,
	\dots, X_n)$ является достаточной статистикой для оценки $\bar\theta$ тогда и
	только тогда, когда 
  \[
    \mathscr{L}(X_1, X_2, \dots, X_n, \bar\theta) = h(X_1, \dots, X_n)
		g(\bar\theta, \bar t (X_1, X_2, \dots, X_n)),
  \]
  где $h$ не зависит от $\theta$, $g$ не зависит от $X_1, X_2, \dots, X_n$
	(разве что через $\bar t$)
\end{theorem}

\begin{proof}
	Проведём доказательство в обе стороны.
	\begin{enumerate}
		\item $\boxed{\Leftarrow}$ Пусть
			\[
				\mathscr L(X_1, X_2, \ldots, X_n, \theta) = g\left(\bar \theta, \bar t(X_1,
				X_2, \ldots, X_n) \right)h(X_1, \ldots, X_n).
			\]

			Имеем
  \begin{multline*}
    P(\xi_1 = X_1, \dots, \xi_n = X_n \, | \, \bar t (\xi_1, \dots, \xi_n) = \bar t)
    = \\ = \frac{P(\xi_1=X_1, \dots, \xi_n = X_n, \bar t(X_1, \dots, X_n) = \bar
		t_0)}{P(\bar t(\xi_1, \dots, \xi_n) = \bar t_0)}.
  \end{multline*}
При $ \bar t(X_1, X_2, \ldots, X_n) \neq \bar t_0$ полученное выражение,
очевидно, равно нулю. Рассмотрим случай $ \bar t(X_1, X_2, \ldots, X_n) = \bar
t_0 $. Тогда числитель будет равен функции правдоподобия $ \mathscr L(X_1,
\ldots , X_n, \bar \theta) $, а знаменатель 
\[
		P(\bar t(\xi_1, \ldots, \xi_n) = \bar t) = \sum_{\bar Z \colon \bar t(\bar
		Z) = \bar t_0} P(\xi_1 = Z_1, \ldots, \xi_n = Z_n) = \sum_{\bar Z\colon \bar
		t(\bar Z) = \bar t_0}\mathscr L(Z_1, \ldots, Z_n, \bar \theta). 
\]

Следовательно, для рассматриваемого случая имеем
\begin{multline*}
    P(\xi_1 = X_1, \dots, \xi_n = X_n \, | \, \bar t (\xi_1, \dots, \xi_n) =\\=
		\frac{P(\xi_1 = X_1, \ldots, \xi_n = X_n)}{\sum P(\xi_1 = Z_1, \ldots, \xi_n = Z_n)} =\\= \frac{h(X_1, \dots, X_n)
		g(\bar \theta, \bar t(X_1, \dots, X_n))}{\sum h(Z_1, \dots, Z_n)
		g(\bar\theta, \bar t(Z_1, \dots, Z_n)} = \frac{h(X_1, \dots, X_n)}{\sum h(Z_1,
		\dots, Z_n)},
\end{multline*}
где суммирование, как и прежде, производится по таким $ \bar Z $, что $ \bar
t(\bar Z) = \bar t_0 $. Результат не зависит от $ \bar \theta $, что и
требовалось доказать.
\item $\boxed{\Rightarrow}$
  Докажем, обратно, что если вероятность
	\[
		P(\xi_1 = X_1, \dots, \xi_n = X_n \,| \,\bar
		t(\xi_1, \dots, \xi_n)
		= \bar t)
	\]
	не зависит от $\bar \theta$, то $\mathscr{L}$ факторизуется. Действительно, в
	случае $ \bar t_0 = \bar t(X_1, \ldots, X_n) $
  \begin{multline*}
    \mathscr{L} (X_1, \dots, X_n, \bar\theta) = P_{\bar\theta} (\xi_1 = X_1,
		\dots, \xi_n = X_n) =\\= P_{\bar\theta} (\xi_1 = X_1, \dots, \xi_n = X_n \,
		|\, \bar
		t(\bar\xi) = \bar t) \cdot P(\bar t(\bar\xi) = \bar t).
  \end{multline*}
  Первый множитель не зависит от $\bar\theta$, второй зависит от $X_1, \dots,
	X_n$ только через $\bar t(X_1, \dots, X_n)$.
\end{enumerate}
\end{proof}

\begin{remark}
  \begin{align*}
		\mathscr{L} &= h(X_1, \dots, X_n) g(\bar\theta, \bar t(X_1, \dots, X_n),\\
		\ln \mathscr{L} &= \ln h + \ln g,\\
		\frac{\partial \ln g}{\partial \bar \theta} &= \frac{\partial \ln \mathscr{L} (\bar\theta, \bar t(X_1, \dots,
	X_n))}{\partial \bar\theta}.
  \end{align*}
  $\Rightarrow$ ОМП функция от $\bar t(X_1, \dots, X_n)$
  % что значит последняя строка?
  % TODO не понял о чём это замечание, скорее всего ошибка
\end{remark}

\begin{ex}
	Пусть $X_1, \dots X_n \sim \operatorname{Pois}(\lambda)$, 
  \[
    P(X_1, \dots, X_n, \lambda) = P_\lambda (\xi_1=X_1, \dots, \xi_n=X_n) =
		\prod_{k=1}^n P(\xi_k = X_k) = \prod_{k=1}^n \frac{\lambda^{X_k}}{(X_k)!}
		e^{-\lambda} =
		\frac{\lambda^{\sum X_k}}{\prod_{k=1}^n (X_k)!}  e^{-n\lambda},
  \]
	а значит, $\sum_{k=1}^n X_k$ --- достаточная статистика для оценки $\lambda$
	(как и $\bar X$).
\end{ex}

\begin{ex}
  Пусть $X_1, \dots, X_n \sim N(a, \sigma)$.
  \begin{multline*}
		\mathscr{L} (X_1, \dots, X_n, a, \sigma) = \prod_{k=1}^n \frac{1}{\sqrt{2\pi}
		\sigma} \exp\left(-\frac{(X_k-a)^2}{2\sigma^2}\right) =\\= 
		\left(\frac{1}{\sqrt{2\pi}\sigma} \right)^n
		\exp\left(-\frac{1}{2\sigma^2} \sum_{k=1}^n (X_k-a)^2\right) = \\ =
		\left(\frac{1}{\sqrt{2\pi}\sigma}\right)^n \exp\left(-\frac{1}{2\sigma^2}
			\left(\sum_{k=1}^n
		X_k^2 - 2a \sum_{k=1}^n X_k + na^2\right)\right).
  \end{multline*}
  Здесь достаточной статистикой для $ (a, \sigma) $ будет $(\sum X_k^2, \sum
	X_k)$.

  В то же время
	\[
		\sum_{k=1}^n (X_k-a)^2 = \sum_{k=1}^n (X_k - \bar X + \bar X - a)^2 =
		\sum_{k=1}^n (X_k-\bar X)^2 + (\bar X - a)^2 n,
	\]
	поэтому пара
	\[
		\left(\bar X,\, S^2 = \frac{1}{n-1} \sum_{k=1}^n (X_k-\bar X)^2\right)
	\]
	является достаточной статистикой для $\sigma$ при известном $a$.
\end{ex}

\begin{ex}
	Пусть $X_1, \dots, X_n \sim \operatorname{Par}(\alpha, \theta)$. Плотность
	равна
	\[
		p(x, \theta, \alpha) = \frac{\alpha}{\theta}
		\left(\frac{\theta}{x}\right)^{\alpha+1}, \quad x\geqslant 0.
	\]

	Найдём функцию правдоподобия и вычленим достаточную статистику:
  \[
		\mathscr{L} (X_1, \dots, X_n, \alpha, \theta) = \prod_{k=1}^n \frac{\alpha}{\theta}
		\left(\frac{\theta}{x}\right)^{\alpha+1} I(X_k \geqslant 0) = \frac{\alpha^n
		\theta^{\alpha n}}{(\prod X_k)^{\alpha+1}} I(X_{(1)}\geqslant 0) \cdot 1.
  \]
Следовательно, 
  $(X_{(1)}, \prod X_k)$ --- достаточная статистика для закона Паретто.
\end{ex}

\section{Интервальные оценки (доверительные интервалы)}
Пусть $X_1, \dots, X_n$ --- выборка из распределения с плотностью $p(x,
\theta)$, а $1-\alpha$ --- \emph{уровень доверия}.

\begin{definition}
	\emph{Доверительным интервалом} для одномерного параметра $\theta$ с 
доверительной вероятностью $1  - \alpha$ называется любой интервал $(\theta_1, \theta_2)$, 
содержащий истинное значение параметра с вероятностью $1 - \alpha$.  
\end{definition}


\subsection{Принцип построения доверительных интервалов}
\begin{enumerate}
  \item Находим статистику $\eta = \varphi(X_1, \dots, X_n, \theta)$, закон
		распределения  которой $F_\eta (x)$ известен точно или приближенно и не
		зависит от $\theta$.
  \item Находим квантили $t_{\alpha/2}$ и $t_{1-\alpha/2}$, при этом
    \[ 
			F_\eta(t_{\alpha/2})  = \alpha/2,\quad F_\eta(t_{1-\alpha/2}) = 1 -
			\alpha/2.
		\]
		Тогда
    \[
      P(t_{\alpha/2} < \varphi(X_1, \dots, X_n, \theta) < t_{1-{\alpha/2}}) =
			F_\eta (t_{1 - \alpha/2}) - F_\eta (t_{\alpha/2}) = 1- \alpha
    \]
  \item Неравенство  
  \[
		t_{\alpha/2} < \varphi(X_1, \ldots, X_n, \theta) < t_{1-\alpha/2}
  \]
		разрешается относительно $\theta$: 
		\[
			\ubar \varphi(X_1, \ldots, X_n, \alpha) < \theta < \bar\varphi(X_1,
			\ldots, X_n, \alpha).
		\]
		Обозначив $ \theta_1 = \ubar\varphi(X_1, \ldots, X_n, \alpha) $, $ \theta_2
		= \bar\varphi(X_1, \ldots, X_n, \alpha)$, получаем доверительный интервал
		уровня $ 1 - \alpha $.
\end{enumerate}

\begin{ex}[для неизвестного $ a $ при известном $ \sigma $]
  Пусть $X_1, \dots, X_n \sim N(a, \sigma_0)$, где $\sigma_0$ известна.

  Тогда по методу моментов $\hat a = \bar X$. Отсюда
  \[
		\bar X = \frac{1}{n} \sum_{k=1}^n X_k \sim N\left(a,
		\frac{\sigma_0}{\sqrt{n}}\right).
  \]

	Следовательно, распределение
	\[
		\frac{\bar X - a}{\sigma_0} \sqrt{n} \sim N(0, 1)
	\]
не зависит от параметра $a$.

Получили
  \[
		-u_{1-\alpha/2} = u_{\alpha/2} &< \frac{\bar X - a}{\sigma_0} \sqrt{n} <
		u_{1-\alpha/2}, 
	\]
	или
	\[
		\bar X - \frac{u_{1-\alpha/2}}{\sqrt{n}} &< a < \bar X +
		\frac{u_{1-\alpha/2}}{\sqrt{n}}.
	\]
 \end{ex}

 \begin{ex}[для неизвестного $ \sigma $ при известном $ a $]
  Рассмотрим противоположный случай. Пусть $X_1, \dots, X_n \sim N(a_0,
	\sigma)$,
	и параметр $a_0$ известен.

	В этом случае
	\[
		\mathscr{L} (X_1, \dots, X_n, \sigma) = \prod_{k=1}^n \frac{1}{\sqrt{2\pi}
		\sigma} \exp\left(-\frac{(X_k - a)^2}{2\sigma^2}\right) =
		\left(\frac{1}{\sqrt{2\pi}\sigma}\right)^n \exp\left(-\frac{1}{2\sigma^2}
		\sum_{k=1}^n (X_k-a)^2\right).
	\]
	Следовательно, $\sum_{k=1}^n (X_k-a)^2$ (вместе с $ S_0^2 = 1/n \sum_{k=1}^n(X_k - a_0)^2 $) будет достаточной статистикой.

	При этом, как легко заметить,
	\[
		\sum\limits_{k=1}^n \frac{(X_k-a)^2}{\sigma^2} = \frac{S^2_0 n}{\sigma^2}
		\sim \chi^2 (n).
	\]

	Найдём квантили и разрешим неравенство:
  \begin{gather*} 
		\chi^2_{\alpha/2} (n) < \sum_{k=1}^n \frac{(X_k - a_0)^2}{\sigma^2} <
		\chi^2_{1 - \alpha/2} (n), \\
		\frac{\sum_{k=1}^n (X_k-a_0)^2}{\chi^2_{1 - \alpha/2} (n)} < \sigma^2 <
		\frac{\sum_{k=1}^n (X_k - a_0)^2}{\chi^2_{\alpha/2}(n)}.
  \end{gather*}
\end{ex}

\begin{ex}
  Произведено $n=36$ измерений, $\bar X = 9.3 \text{кОм},  \sigma_0 = 2
	\text{кОм}$. Найти доверительный интервал $1- \alpha = 0.95$ для оценки
	параметра $a$.

\[
  0.95 = P\left(\bar X - \frac{u_{0.975} \cdot 2}{\sqrt{36}} < a < \bar X + \frac{u_{0.975} \cdot 2}{\sqrt{36}}\right) \Leftrightarrow a \in (8.65, 9.95)
\]
\end{ex}

\begin{ex}[на показательный закон]
	Пусть
  $X_1, \dots, X_n \sim E(\lambda)$.
  Достаточной статистикой  будет $\hat \lambda = \frac{1}{\bar X}$. Докажем
	сначала, что
  \[
		2\lambda \sum_{k=1}^n X_k \sim \chi^2(2n).
	\]

	Действительно, имеем
	\[
		X_k \sim \lambda e^{-\lambda x} \Rightarrow \eta = 2\lambda X_k \sim \frac{1}{2}
		e^{-\frac{x}{2}} \equiv \chi^2 (2).
	\]

	Найдём квантили и разрешим неравенство: 
	\begin{gather*}
		\chi^2_{\alpha/2}(2n) < 2\lambda\sum_{k=1}^n X_k <
		\chi^2_{1-\alpha/2}(2n),\\
	\frac{\chi^2_{\alpha/2}(2n)}{2 n\bar X } < \lambda <
	\frac{\chi^2_{1-\alpha/2}(2n)}{2 n \bar X}.
	\end{gather*}
\end{ex}

\subsection{Приближенные доверительные интервалы}
  Пусть $X_1, \dots, X_n \sim F(x, \theta), \M_\theta X_i = a(\theta), \D_\theta
	X_i = d(\theta)$, и $n$ велико. 

	Тогда согласно центральной предельной теореме статистика
\[
	\frac{\sum_{i=1}^n X_i - n a(\theta)}{\sqrt{n d(\theta)}} \approx N(0, 1)
\]
в пределе распределена по стандартному нормальному закону, что даёт приближённое
равенство
\[
	P\left(u_{1 - \alpha/2} < \frac{\sum_{i=1}^n X_i - n a(\theta)}{\sqrt{n d(\theta)}}<
	u_{1-\alpha/2}\right) \approx 1 - \alpha.
\]

\begin{ex}[на биномиальный закон, $ k $ известно]
  Пусть $X_1, \dots, X_n \sim B(k, p)$, но $k$ известно, а $ p $ неизвестно.

	Известно, что 
	\[
		\M X_k = kp, \quad \D X_k = k p (1-p).
	\]
В этом случае
\[
	\frac{\sum_{k=1}^n X_k - n k p}{\sqrt{nkp(1-p)}} \in (-u_{1-\alpha/2}, u_{1-\alpha/2})
\]
с вероятностью, стремящейся к $ 1 - \alpha $. Чтобы не решать квадратные
неравенства, заменим ещё $ p $ под корнем в
знаменателе оценкой $\hat p = \frac{\bar X}{k}$ (потеря точности невелика). 

Найдём квантили и разрешим неравенство:
\begin{gather*}
	-u_{1-\alpha/2}\cdot \sqrt{nk \frac{\bar X}{k} \left(1 - \frac{\bar X}{k}\right)} <
	\sum_{k=1}^n X_k -
	nkp < u_{1-\alpha/2} \cdot \sqrt{nk \frac{\bar X}{k} \left(1 - \frac{\bar X}{k}\right)}, \\
	\frac{\bar X}{k} - \frac{u_{1-\alpha/2}\cdot \sqrt{\frac{\bar X}{k} \left(1 - \frac{\bar
	X}{k}\right)}}{\sqrt{nk}} < p < \frac{\bar X}{k} + \frac{u_{1-\alpha/2}
\cdot \sqrt{\frac{\bar X}{k} \left(1 - \frac{\bar X}{k}\right)}}{\sqrt{nk}}.
\end{gather*}
\end{ex}
