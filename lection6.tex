\section{Лекция 6 - 2023-10-11 - Доверительные интервалы для параметров
нормального закона} \label{sec:lec6}
\subsection{Совместный закон распределения выборочного среднего и выборочной
дисперсии для нормально распределенной генеральной совокупности.}

Пусть $X_1, \dots, X_n \sim N(a, \sigma)$.

\begin{theorem}
	\label{the:1}
  Если $X_1, \dots, X_n \sim N(a, \sigma)$, то статистики
\[
	\bar X,\quad \frac{(n-1) S^2}{\sigma^2} = \frac{1}{\sigma^2} \sum\limits_{k=1}^n (X_k-\bar X)^2
\]
	независимы и
	\[
		\bar X \sim N\left(a, \frac{\sigma}{\sqrt{n}}\right),\quad \frac{(n-1)
		S^2}{\sigma^2} \sim \chi^2(n-1).
	\]
\end{theorem}

  Пусть $n = 1$, $\eta = a \xi$, $p_\eta(y) = p_\xi(y/a)/|a| $.
\begin{lemma}
  По аналогии, если случайный вектор $ \bar \xi $ имеет плотность $p_{\bar \xi}
	(\bar x)$ и $\bar\eta = A\xi$, где $ A $ --- некоторая невырожденная матрица, то 
	плотность
	\[
		p_{\bar\eta} (\bar y) = |\det A|^{-1} p_{\bar\xi} (A^{-1} \bar y).
	\]
\end{lemma}
\begin{proof}
	Пусть $ B $ --- произвольное борелевское подмножество $ \mathbb R^n $. Заметим
	для начала, что
  \[
		p(\bar\xi \in B) = p(A\xi \in AB) = p(\eta \in AB) = \int\limits_{AB}
		p_{\bar\eta}(\bar y)\,d\bar y.
	\]
	С другой стороны,
  \[ 
		p(\bar \xi \in B) = \int\limits_B p_\xi (\bar x) \, d\bar x
  \to \left[\, \begin{aligned}
    \bar x &= A^{-1} \bar y \\
    \bar y &= A \bar x \\
    d\bar x &= |\det(A^{-1})| \,d\bar y
\end{aligned} \,\right] \to 
  \int\limits_{AB} p_\xi(A^{-1} \bar y) |\det A^{-1}|\, d\bar y ,
\]
откуда  
\[
		p_{\bar \eta} (\bar y)
  = p_{\bar\xi} (A^{-1} \bar y) |\det A^{-1}|.
\]

\end{proof}


\begin{proof}[Доказательство теоремы \ref{the:1}]
  Рассмотрим центрированную и нормированную выборку
	\[
		X'_k = \frac{X_k -
		a}{\sigma}.
	\]

	Компоненты вектора $ \vec X_n' $ независимы и распределены по стандартному
	нормальному закону. Значит,  
	\begin{gather*}
		\bar X' = \frac{1}{n} \sum\limits_{k=1}^n X_k' = \frac{1}{n} \sum \frac{(X_k
		- a)}{\sigma} = \frac{\bar X - a}{\sigma}, \\
  p_{\vec X'_n} (\bar x) = \frac{1}{(\sqrt{2\pi})^n} \exp\left(-\frac{1}{2} \bar
		x^T \bar
	x\right).
\end{gather*}

Произведём ортогональное преобразование 
\[
		\vec Y_n = C \vec X'_n,
\]
причём положим 
\[
	Y_1 = \frac{1}{\sqrt{n}} \sum_{k=1}^n X_k' = \sqrt{n} \bar X' = \frac{\bar X -
	a}{\sigma} \sqrt{n}, \qquad Y_1^2 = n (\bar X')^2,
\]
то есть матрица перехода $ C $ имеет вид 
\[
	C = \begin{pmatrix}
		\frac{1}{\sqrt n} & \frac{1}{\sqrt n} &\dots & \frac{1}{\sqrt n}\\
		c_{21} & c_{22} & \dots & c_{2n}\\
		\vdots & \vdots & \ddots & \vdots \\
		c_{n1} & c_{n2} & \dots & c_{nn}
	\end{pmatrix}.
\]
Кроме того, согласно свойствам ортогональных матриц
\begin{multline*}
  p_{\bar Y} (y) = |\det C|^{-1} p_{\bar X'} (C^{-1} \bar y) = \\ =
	\frac{1}{(\sqrt{2\pi})^n} \exp\left(-\frac{1}{2} (C^{-1} \bar y)^T C^{-1} \bar
	y\right) =
	\frac{1}{ \left( \sqrt{2\pi} \right)^n} \exp \left( - \frac{1}{2} \bar y^T
	(C^{-1})^T C^{-1} \bar y \right) = \\ =
	\frac{1}{(\sqrt{2\pi})^n} \exp\left(-\frac{1}{2} \bar y^T \bar y\right).
\end{multline*}
Таким образом, вектор $ \bar Y_n $ также имеет независимые компоненты,
распределённые по закону $ \mathscr N(0,1) $.

Представим следующую статистику в следующем виде:  
\begin{multline*}
	\frac{S^2(n-1)}{\sigma^2} = \frac{1}{\sigma^2} \sum_{i=1}^n \left( X_i - \bar
	X\right)^2 = 
	\frac{1}{\sigma^2} \sum_{k=1}^n ((X_k - a) - (\bar X - a))^2 =\\
		\sum_{k=1}^n \frac{(X_k-a)^2}{\sigma^2}-2 \frac{\bar X - a}{\sigma^2}
		\sum_{k=1}^n (X_k-a) +
		\frac{n}{\sigma^2} (\bar X - a)^2 = \\
		\sum_{k=1}^n \frac{(X_k-a)^2}{\sigma^2} - n \left(\frac{\bar X-a}{\sigma}\right)^2 =
		\\ = 
		\sum_{k=1}^n (X_k')^2 - n (\bar X')^2 = \sum_{k=1}^n Y_k^2 - Y_1^2 = 
		\sum_{k=2}^n Y_k^2.
\end{multline*}
Отсюда следует независимость 
\[
		\frac{\bar X - a}{\sigma}\sqrt n = Y_1, \qquad \frac{S^2(n-1)}{\sigma^2} =
		\sum_{k=2}^n Y^2_k,
\]
а также вид распределения 
\[
	\frac{S^2(n-1)}{\sigma^2} = \sum_{k=2}^n Y^2_k \sim \chi^2(n-1).
\]

Заметим, что
  \[
		\bar X = \sigma \cdot (\bar X' + a) = \sigma\cdot \left(\sqrt{\frac{Y_1}{n}} +
	a\right).
\]
\end{proof}


\subsection{Доверительный интервал для дисперсии при неизвестном матожидании}
\begin{ex}[для $\sigma$ при неизвестном $a$]
	Рассмотрим статистику  
	\[
		S^2 = \frac{1}{n-1} \sum_{k=1}^n \left( X_k - \bar X \right)^2.
	\]
	По теореме \ref{the:1}
	\[
			\frac{S^2(n-1)}{\sigma^2} \sim \chi^2(n-1).
	\]
	
	Находим квантили и разрешаем неравенство:
\begin{gather*}
  \chi^2_{\alpha/2} (n-1) < \frac{(n-1) S^2}{\sigma^2} < \chi^2_{1 - \alpha/2}
	(n-1), \\
  \sqrt{\frac{(n-1) S^2}{\chi^2_{1 - \alpha/2}(n-1)}} < \sigma <
	\sqrt{\frac{(n-1) S^2}{\chi^2_{\alpha/2} (n-1)}}.
\end{gather*}
\end{ex}

\subsection{Доверительный интервал для матожидания при неизвестной дисперсии}
\begin{ex}[для $a$ при неизвестном $\sigma$]
	По теореме \ref{the:1} статистики 
	\begin{align*}
		\frac{\bar X - a}{\sigma}\sqrt n &\sim \mathscr N(0, 1),\\
		\frac{S^2(n-1)}{\sigma^2} &\sim \chi^2(n-1)
	\end{align*}
	независимы. Поэтому отношение  
	\[
  \sqrt{n} \frac{\bar X - a}{S} &= \sqrt{n} \frac{\frac{\bar X -
	a}{\sigma}}{\frac{S}{\sigma}} = \sqrt{n} \frac{\frac{\bar X -
a}{\sigma}}{\sqrt{\frac{1}{n-1} \frac{(n-1) S^2}{\sigma^2}}} \sim t(n-1),
	\]
	где $ S = \sqrt{S^2} $, распределено по закону Стьюдента (см. лекцию
  \ref{sec:lec1}.	

Находим квантили и разрешаем неравенство:
\begin{gather*}
  -t_{1 - \alpha/2} (n-1) = t_{\alpha/2} (n-1) < \sqrt{n} \frac{\bar X -
	a}{S} < t_{1-\alpha/2} (n-1), \\
  \bar X - \frac{t_{1-\alpha/2}(n-1) S}{\sqrt{n}} < a < \bar X +
	\frac{t_{1-\alpha/2}(n-1) S}{\sqrt{n}}.
\end{gather*}
\end{ex}

\subsection{Доверительный интервал для разницы матожиданий при известных дисперсиях}
\begin{ex}[для $\Delta a$ при известных $ \sigma_1 $, $ \sigma_2 $]
Пусть $X_1, \dots, X_n \sim N(a, \sigma_x)$, $Y_1, \dots, Y_m \sim N(b,
\sigma_y)$, а $\sigma_x, \sigma_y$ независимы.

Рассмотрим статистику $ \bar X - \bar Y $, распределение которой нормально с
параметрами  
\[
		\M (\bar X - \bar Y) = a - b, \qquad \D(\bar X - \bar Y) =
		\frac{\sigma_x^2}{n} + \frac{\sigma_y^2}{m}.
\]
Стандартизируем распределение:
\[
  \frac{\bar X - \bar Y - (a-b)}{\sqrt{\sigma_x^2/n +
	\sigma_y^2/n}} \sim \mathscr N(0, 1).
\]
Действительно,
\begin{align*}
	\M\left[\bar X - \bar Y - \left(a-b\right)\right] &= (a-b) - (a-b) = 0, \\
  \D\left[\frac{\bar X - \bar Y - (a-b)}{\sqrt{\sigma_x^2/n +
\sigma_y^2/n}}\right] &= \frac{D \bar X + D \bar Y}{\sigma_x^2 / n +
\sigma_y^2 / m} = \frac{\sigma_x^2 / n + \sigma_y^2 / m}{\sigma_x^2 / n +
\sigma_y^2 / m} = 1.
\end{align*}

Находим квантили и разрешаем неравенство:
\begin{gather*}
  - u_{1-\alpha/2} < \frac{\bar X - \bar Y - (a-b)}{\sqrt{\sigma_x^2 / n +
	\sigma_y^2 / m}} < u_{1-\alpha/2}, \\
  \bar X - \bar Y - u_{1-\alpha/2} \sqrt{\sigma_x^2 / n + \sigma_y^2 / m}< a-b <
	\bar X - \bar Y + u_{1-\alpha/2} \sqrt{\sigma_x^2 / n + \sigma_y^2 / m}.
\end{gather*}
\end{ex}

\subsection{Доверительный интервал для разницы матожиданий при неизвестных, но равных дисперсиях}
\begin{ex}[для $\Delta a$ при неизвестном $ \sigma_1 = \sigma_2 $]
Пусть $X_1, \dots, X_n \sim N(a, \sigma)$, $Y_1, \dots, Y_m \sim N(b, \sigma)$,
но $\sigma$ неизвестен.

Как мы выяснили,
\[
  \frac{\bar X - \bar Y - (a-b)}{\sigma \sqrt{1/n + 1/m}} \sim N(0, 1).
\]
Кроме того, по теореме \ref{the:1} эта статистика независима от статистики
\[
	\frac{(n-1)S_x^2}{\sigma^2} + \frac{(m_1) S_y^2}{\sigma^2} \sim \chi^2(m+n-2).
\]
После очевидных преобразований 
\begin{multline*}
	\frac{\bar X - \bar Y - (a-b)}{\sqrt{(n-1)S^2_X+(m-1)S^2_Y}} \sqrt{nm}
	\sqrt{\frac{n+m -2}{n+m}}= \\ =
	\frac{\bar X - \bar Y - (a-b)}{\sqrt{\frac1n + \frac1m}
	\sqrt{\frac{1}{n+m-2} \frac{(n-1) S_x^2 + (m-1)S_y^2}{\sigma^2}}} \sim
	t(n+m-2)
\end{multline*}
получаем распределение Стьюдента.

Находим квантили и разрешаем неравенство:
\begin{gather*}
	t_{\alpha/2}(n+m-2) \leqslant \frac{\bar X - \bar Y - (a-b)}{\sqrt{
	\frac{1}{n}+ \frac{1}{m}} \sqrt{ \frac{1}{n+m - 2}} \left( \sum_{k=1}^n \left(
X_k - \bar X\right)^2 + \sum_{k=1}^m \left( Y_k - \bar Y \right)^2 \right) }
\leqslant t_{1-\alpha/2} (n+m-2),\\
	(a-b) \in (\bar X - \bar Y - \delta, \bar X - \bar Y + \delta), \quad
	\text{где}\\
\delta = t_{1-\alpha/2} (n+m-2) \sqrt{\frac{n+m}{nm(n+m-2)}} \sqrt{(n-1)S_x^2 +
(m-1) S_y^2}.
\end{gather*}
\end{ex}

\subsection{Доверительный интервал для отношения дисперсий при неизвестных матожиданиях}
\begin{ex}[для отношения $ \sigma_1/\sigma_2 $ при неизвестных $ a_1 $, $ a_2 $]
Пусть $X_1, \dots, X_n \sim \mathscr N(a, \sigma_x)$, $Y_1, \dots, Y_m
\sim\mathscr N(b, \sigma_y)$.

Рассмотрим статистику  
\[
		\frac{S^2_X}{S^2_Y}\cdot \frac{\sigma^2_y}{\sigma^2_x} =
		\frac{\frac{(n-1)S^2_X}{\sigma^2_x}}{\frac{(m-1)S^2_Y}{\sigma^2_y}} \cdot
		\frac{m-1}{n-1} \sim F(n-1,\,m-1),
\]
поскольку
\[
(m-1) S_Y^2 / \sigma_y^2 \sim \chi^2(m-1),\qquad (n-1) S_X^2 / \sigma_x^2 \sim
\chi^2(n-1).
\]

Находим квантили и разрешаем неравенство:
\begin{gather*}
f_{\alpha/2} (n-1, m-1) < \frac{\sigma_y^2 s_x^2}{\sigma_x^2 s_y^2} <
f_{1-\alpha/2} (n-1, m-1), \\
\frac{S_x^2}{S_y^2 f_{1-\alpha/2} (n-1, m-1)} < \frac{\sigma_x^2}{\sigma_y^2} <
\frac{s_x^2}{s_y^2 f_{\alpha/2} (n-1, m-1)}.
\end{gather*}
\end{ex}


\subsection{Доверительный интервал для отношения дисперсий при известных матожиданиях}
\begin{ex}[для отношения $ \sigma_1/\sigma_2 $ при известных $ a_1 $, $ a_2 $]
	В случае известных средних значений, используя статистики
\begin{gather*}
  \frac{S_{0X}^2}{\sigma_x^2} = \frac{1}{n}\sum \frac{(X_k-a)^2}{\sigma_x^2}\sim
	\chi^2(n), \\
  \frac{S_{0Y}}{\sigma_y^2} = \frac{1}{m} \sum \frac{(Y_k - b)^2}{\sigma_y^2}
	\sim \chi^2 (m),
\end{gather*}
получаем аналогично доверительный интервал
\[
	\frac{S_{0X}^2}{S_{0Y}^2 f_{1-\alpha/2}(n,m)} < \frac{\sigma_x^2}{\sigma_y^2}
	< \frac{S_{0X}^2}{S_{0Y}^2 f_{\alpha/2}(n,m)}.
\]

% TODO дописать 6 
\end{ex}
