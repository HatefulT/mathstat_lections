\section{Лекция 14 - 2023-12-06 - Корреляционные отношения}

$(\xi, \eta)$, регрессия $\eta = f(\xi) + \varepsilon$
в гауссовском случае оказывается линейной:
\[
  f(\xi) = M(\eta | \xi) = (\text{в гауссовском случае}) = M\eta = \frac{\sigma_{12}}{\sigma_{11}} (\xi - M\xi)
\]

\begin{definition}
  Корреляционным отношением СВ $\eta$ к СВ $\xi$ называется 
  \[
    \nu_{\xi \eta} = \sqrt{\frac{Df(\xi)}{D\eta}} = \sqrt{1 - \frac{\bar\sigma^2_n}{D\eta}}, 
  \]
  где $\eta = f(\xi) + \varepsilon \Rightarrow D\eta = Df(\xi) + \bar \sigma_\eta^2$.
\end{definition}

\subsection{Свойства корреляционного отношения}
\begin{enumerate}
  \item $\nu_{\xi\eta} = 0 \Leftrightarrow \bar\sigma^2_\eta = D\eta \Leftrightarrow Df(\xi) = 0 \Leftrightarrow f(\xi) = Mf(\xi) = M\eta$
  
  \item $\nu_{\xi\eta}^2 = 1 \Leftrightarrow \bar\sigma_\eta^2 = 0 \Leftrightarrow \eta = f(\xi)$,
    то есть существует функциональная зависимость.

  \item $(\xi, \eta)$ - нормальный вектор, то $\nu_{\xi\eta}^2 = r_{\xi\eta}^2$.
    \begin{proof}
      \begin{multline*}
        Df(\xi) = D\left(M\eta + \dfrac{\sigma_{12}}{\sigma_{11}} (\xi - M\xi)\right)
        = \dfrac{\sigma_{12}^2}{\sigma_{11}^2} D(\xi - M\xi)
        = \dfrac{\sigma_{12}^2}{\sigma_{11}^2} \sigma_{11}
        = \dfrac{\sigma_{12}^2}{\sigma_{11}}
        = \dfrac{\sigma_{12}^2}{\sigma_{11} \sigma_{22}} \sigma_{22}
        = \sigma_{22} r_{\xi\eta}^2
        \Leftrightarrow \\
        \Leftrightarrow
        \nu_{\xi\eta}^2 = \frac{Df(\xi)}{D\eta} = \frac{\sigma_{22} r_{\xi\eta}^2}{\sigma_{22}} = r_{\xi\eta}^2
      \end{multline*}
    \end{proof}

  \item В общем случае $ 0 \leqslant r_{\xi\eta}^2 \leqslant \nu_{\xi\eta}^2 \leqslant 1$.
    \begin{proof}
      Пусть $\zeta = x \cdot \xi - f(\xi)$, $x\in\mathbb{R}$.
      \[
        0 \leqslant D\xi = D(x \cdot \xi - f(\xi)) = x^2 D\xi - 2 x \cov (\xi, f(\xi)) + D f(\xi)
      \]
      Больше нуля означает, что дискриминант меньше нуля:
      \[
        \frac{\text{Дискриминант}}{4} = \cov(\xi, f(\xi))^2 - D\xi Df(\xi) \leqslant 0
      \]
      \[
        \cov(\xi, f(\xi)) = \cov(\xi, \eta - \xi)
        = \cov(\xi, \eta) - \cov(\xi, \xi) = \cov(\xi, \eta)
      \]

      \[
        0 \geqslant \sigma_{12}^2 - \sigma_{11} \nu_{\xi\eta}^2 \sigma_{22} = \sigma_{12}^2 - D\xi Df(\xi) \Rightarrow r_{\xi\eta}^2 = \frac{\sigma_{12}^2}{\sigma_{11} \sigma_{22}} = \nu_{\xi\eta}^2
      \]
    \end{proof}
\end{enumerate}

Таким образом, если $0 \leqslant r_{\xi\eta}^2 < \nu_{\eta \xi}^2 = 1$, то имеется нелинейная
функциональная зависимость; если $0 \leqslant r_{\xi\eta}^2 = \nu_{\eta\xi}^2$ --- регрессия
линейная.

\subsubsection{Оценка корреляционного отношения}
Проверяем гипотезы:
\begin{align*}
  &H_0\text{: } \nu_{\xi\eta}^2 = 0, \\
  &H_1\text{: } \nu_{\xi\eta}^2 \neq 0.
\end{align*}

Интуитивный смысл этой проверки --- проверка того, является ли регрессия константой, т.е.
\[
  Y(X_k) = \operatorname{const} + \varepsilon.
\]

Планирование эксперимента:
Выберем $m$ точек $X_i, i=\overline{1, m}$. Каждое число подаём на вход несколько ($n_i$) раз и
получаем точки:
\[
  X_i \rightarrow y_{i,1}, y_{i,2}, \dots, y_{i, n_i}, i = \overline{1, m}.
\]
Оценка регрессии:
\[
  \bar Y_i = \frac{1}{n_i} \sum_{j=1}^{n_i} Y_{ij} = \hat f(X_i).
\]


\[
  \hat\sigma_f^2 = \frac{1}{n} \sum_{i=1}^m n_i (\bar Y_i - \bar Y)^2
  \]

\[
  \hat\sigma_\eta^2 = \frac{1}{n} \sum_{i=1}^m \sum_{j=1}^{n_i} (Y_{ij} - \bar Y)^2
  \text{-- оценка дисперсии}
\]

\[
  \hat f = \frac{\sigma_\eta^2}{\sigma_\eta^2 - \sigma_f^2} \cdot \frac{n-m}{m-1}
  \sim F(m-1, n-m)
\]

\begin{align*}
  \hat f &\geqslant F_{1-\alpha} (m-1, n-m) \Rightarrow \text{$H_0$ отклоняем} \\
  \hat f &< F_{1-\alpha} (m-1, n-m) \Rightarrow \text{$H_0$ принимаем}
\end{align*}

\subsection{Линейная регрессионная модель}

Выборка $(\vec{X}_1, \vec{Y}_1), (\vec{X}_2, \vec{Y}_2), \dots, (\vec{X}_n, \vec{Y}_n)$, 
(p+1) - мерная: $\vec{X}_j = (X_j^{(1)}, X_j^{(2)}, \dots, X_j^{(n)})$ - факторы (входные переменные).
\[
  Y_k = \sum_{j=0}^{m-1} \beta_j \Psi_j(\vec{X}_k) + \varepsilon_k,
\]
$\Psi_j$ - базисные функции.

\[
  Y = \mathcal{F} \vec{\beta} + \varepsilon
\]

\[
  \mathcal{F} = \begin{pmatrix}
    \Psi_0(\vec{X}_1) & \Psi_1(\vec{X}_1) & \dots & \Psi_{m-1} (\vec{X}_1) \\
    \Psi_0(\vec{X}_1) & \Psi_1(\vec{X}_1) & \dots & \Psi_{m-1} (\vec{X}_1) \\
    \dots & \dots \\
    \Psi_0(\vec{X}_1) & \Psi_1(\vec{X}_1) & \dots & \Psi_{m-1} (\vec{X}_1) \\
  \end{pmatrix}
\]

\begin{ex}
  Простейшая линейная регрессия: $\Psi_0(X_k) = 1$, $\Psi_1(X_k) = X_k$.
  \[
    Y_k = \beta_0 + \beta_1 X_k + \varepsilon_k
  \]
\end{ex}
  Двухфакторная линейная регрессия: $\vec{X}_k = (X_k^{(1)}, X_k^{(2)})$.
  \[
    \Psi_0^(\vec{X}_k) = 1, \Psi_1(\vec{X}_k) = X_k^{(1)}, \Psi_2(\vec{X}_k) = X_k^{(2)}
  \]
  \[
    Y_k = \beta_0 + \beta_1 X_k^{(1)} + \beta_2 X_k^{(2)} + \varepsilon_k
  \]
\begin{ex}
  Простейшая квадратичная регрессия:
  \[
    Y_k = \beta_0 + \beta_1 X_k + \beta_2 X_k^2 + \varepsilon_k
  \]
\end{ex}

\subsection{Ограничения на $\varepsilon_k$}

\begin{enumerate}
  \item $M\varepsilon_k = 0$ для всех $ k = \overline{1, \ldots, n} $,
  \item $M\varepsilon_k \varepsilon_j = 0$ для всех $k \neq j$,
  \item $D \varepsilon_k = \delta^2$ для всех $ k = \overline{1, \ldots, n} $,
  \item $\mathbf Y = \mathcal{F} \vec{\beta} + \varepsilon$.
\end{enumerate}
Из второго ($ \cov(\eta, \xi) = M(\xi\cdot\eta) - M\xi \cdot M\eta $) и третьего свойства получаем, что $\Sigma_{\mothbf{Y}} = \Sigma_{\bm\varepsilon} = \delta^2 E_n$
($E_n$ --- единичная матрица).
Тогда $M\mathbf{Y} = \mathcal{F} \vec{\beta}$.

\begin{theorem}
  Если выполнены условия (*), $\mathcal{F}^T \mathcal{F}$ -- невырожденная матрица, то
  $\hat{\vec{\beta}} = (\mathcal{F}^T \mathcal{F})^{-1} \mathcal{F} Y$ является несмещенной и оптимальной (в классе линейных несмещенных оценок) оценкой $\vec{\beta}$.
\end{theorem}
\begin{remark*}
$\hat{\vec{\beta}}$ выбирается по МНК $\sum \varepsilon_k^2 \to \min$.
\end{remark*}
\begin{proof}
  \begin{enumerate}
    \item \textbf{Несмещённость}:
      \[
        M \hat{\vec{\beta}} = M ( (\mathcal{F}^T \mathcal{F})^{-1} \mathcal{F}^T Y )
        = (\mathcal{F}^T \mathcal{F})^{-1} \mathcal{F}^T MY 
        =(\mathcal{F}^T \mathcal{F})^{-1} \mathcal{F}^T \mathcal{F} \vec{\beta}
        = \vec{\beta}
      \]

    \item Найдём МНК параметра $\vec{\beta}$, для чего минимизируем сумму квадратов ошибок: 
      \[ 
        \Delta = \sum_{k=1}^n \varepsilon_k^2
        = \sum_{k=1}^n (Y_k - \sum_{j=0}^{m-1} \beta_j \Psi_j (\vec{X}_k) )^2
      \] 
      Необходимое условие экстремума:
      \[
        0 = \frac{\partial \Delta}{\partial \beta_j}
        = 2 \sum_{k=1}^n (Y_k - \sum_{j=0}^{m-1} \beta_j \Psi_j(\vec{X}_k) (-\Psi_j (\vec{X}_k)) 
      \]

      В матричной записи:
      \[
        \frac{\partial \Delta}{\partial \beta_j} = 0
        \Leftrightarrow \mathcal{F}^T (Y - \mathcal{F} \vec{\beta}) = 0
        \Leftrightarrow 
        (\mathcal{F}^T \mathcal{F}) \vec{\beta} = \mathcal{F}^T Y
        \Rightarrow
        \hat{\vec{\beta}} = (\mathcal{F}^T \mathcal{F})^{-1} \mathcal{F}^T Y
      \]

    \item \textbf{Оптимальность}:
      Для некоторой другой линейной несмещенной оценки:
      $\tilde{\vec{\beta}} = L Y$,
      \[
        M \tilde{\vec{\beta}} = L MY = L \mathcal{F} \vec{\beta}
        \Rightarrow L \mathcal{F} = E
      \]
      Обозначим: $(\mathcal{F}^T \mathcal{F})^{-1} \mathcal{F}^T = B$
      \[
        \Sigma_{LY} = M(LY - MLY)(LY - MLY)^T = M( L(Y-MY)(Y-MY)^T L^T ) = L \delta^2 E_n L^T = \delta^2 L L^T
      \]

      Следствие: 
      \[
        \Sigma_{\hat{\vec{\beta}}} = \delta^2 (\mathcal{F}^T \mathcal{F})^{-1} \mathcal{F}^T ( (\mathcal{F}^T \mathcal{F})^{-1} \mathcal{F}^T )^T = B B^T = \delta^2 (\mathcal{F}^T \mathcal{F})^{-1} 
      \]

      \begin{multline*}
        LL^T = (L-B+B) (L-B+B)^T = (L-B)(L-B)^T + B (L-B)^T + (L-B) B^T + BB^T = \\
        = (\text{так как $L\mathcal{F} = E$}) = (L-B)(L-B)^T + BB^T
      \end{multline*}

      \[
        \Sigma_{LY} = \delta^2 LL^T = \delta^2 (L-B)(L-B)^T + \delta^2 BB^T = \delta^2 (L-B)(L-B)^T + \Sigma_{BY}
      \]

      У матрицы $\delta^2 (L-B)(L-B)^T$ на диагонали стоят неотрицательные элементы, следовательно:
      \[
        D\tilde{\beta_i} \geqslant D\hat{\beta_i}
      \]
      что означает оптимальность.

  \end{enumerate}
\end{proof}

\begin{ex}
  Пусть $\sum_{k=1}^20 X_k^2 = 5$, $\sum_{k=1}^20 X_k= -5$, и
  \[
    Y_k = \beta_0 + \beta_1 X_k + \varepsilon_k.
  \]

  Найти: $\hat{\beta_0}, \hat{\beta_1}$ - ОНК, $r_{\hat{\beta_0} \hat{\beta_1}}$.

  \[
    \mathcal{F} = \begin{pmatrix}
      1 & X_1 \\
      1 & X_2 \\
      \dots & \dots \\
      1 & X_{20}
    \end{pmatrix}
  \]

  \[
    \mathcal{F} = \begin{pmatrix}
      20 & \sum_{k=1}^{20} X_k \\
      \sum_{k=1}^{20} X_k & \sum_{k=1}^{20} X_k^2
    \end{pmatrix}
  \]

  \[
    \mathcal{F}^T \mathcal{F} = \begin{pmatrix}
      20 & -5 \\
      -5 & 5
    \end{pmatrix}
  \]
  
  \[
    \delta^2 (\mathcal{F}^T \mathcal{F})^{-1} = \delta^2 \frac{1}{75} \begin{pmatrix}
      5 & 5 \\
      5 & 20
    \end{pmatrix}
  \]

  \begin{align*}
    D \hat{\beta}_0 &= \frac{5}{75} \delta^2 \\
    D \hat{\beta}_1 &= \frac{20}{75} \delta^2
  \end{align*}
\end{ex}

\subsection{Оценка дисперсии ошибок}

\begin{definition}
  \emph{Оценкой среднего значения отклика} называется СВ
  \[
    \hat{Y}_k = \sum_{j=0}^{m-1} \hat{\beta}_j \Psi_j(\vec{X}_k) = \mathcal{F} \hat{\vec{\beta}}. 
  \]
\end{definition}

\begin{definition}
  \emph{Остаточной суммой квадратов} называется СВ
  \[
  % Y-\hat{Y} = Y - \mathcal{F} \hat{\vec{\beta}}
  % = \mathcal{F} \vec{\beta} + \varepsilon - \mathcal{F}\hat{\vec{\beta}}
  % = \mathcal{F} (\vec{\beta} - \hat{\vec{\beta}}) + \varepsilon 
    (Y - \mathcal{F} \hat{\vec{\beta}})^T (Y-\mathcal{F}\hat{\vec{\beta}})
  \]
\end{definition}

\begin{theorem}
  Если выполнено (*) и ранг матрицы $\mathcal{F}$ максимален, то несмещенная оценка остаточной
  дисперсии имеет вид:
  \[
    \hat{\delta^2}
    = \frac{1}{n-m} (Y-\mathcal{F}\hat{\vec{\beta}})^T (Y-\mathcal{F}\hat{\vec{\beta}})
    = \dfrac{1}{n-m} \sum \left(Y_k - \sum_{j=0}^{m-1} \hat{\beta}_j \Psi_j (\vec{X_k}) \right)^2.
  \]
\end{theorem}
