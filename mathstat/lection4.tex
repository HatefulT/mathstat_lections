\section{Лекция 4 - 2023-09-27 - Методы получения точечных оценок: моментов, максимального
правдоподобия. Свойства точечных оценок.}
%%\subsection{Вероятности для вариационного ряда}
%%\begin{theorem}
%%	$X_1, X_2, \dots, X_n$ - независимыя выборка из з.р. $F(X)$

%%	тогда $F_{X_{(n)}} = (F(x))^n$, $F_{X_{(1)}} = 1 - (1-F(x))^n$.

%%	если имеется плотность $p(x)$, то $p_{X_{(n)}} = n (F(x))^{n-1} p(x)$, $p_{X_{(1)}} = n (1-F(x))^{n-1} p(x)$, $p_{X_{(1)}, X_{(n)}} = n (n-1) (F(y) - F(x))^{n-2} p(x) p(y)$.
%%\end{theorem}

%%\begin{proof}
%%	\begin{multline}
%%		F_{X_{(n)}} = P(X_{(n)} < x) = P\left(\bigcap_{k=1}^n (X_k < x)\right) = \prod_{k=1}^n P(X_k < x) = (F(x))^n\\
%%		F_{X_{(1)}} (x) = P(X_{(1)} < x) = 1 - P(X_{(1)} \geqslant x) = \dots = 1 - (1-F(x+0))^n = 1 - (1-F(x))^n \\
%%		F_{X_{(1)}, X_{(n)}} (x, y) = P(X_{(1)} < x, X_{(n)} < y) = P(X_{(n)} < y) - P(X_{(n)}, X_{(1)} \geqslant x) = \\
%%		F_{X_{(n)}} (y) - P(\bigcap_{k=1}^n (x \leqslant X_k < y)) = (F(y))^n - (F(y) - F(x))^n
%%		% \\
%%		% TODO: доказательство PX(1) X(n) (x, y)
%%	\end{multline}
%%\end{proof}

%%\begin{ex}
%%	$X_1, X_2, \dots, X_n ~ E(\alpha)$
%%	$F_{x_i} (x) = 1- e^{-\alpha x}, x\geqslant 0.$
%%	$F_{X_{(1)}} (x) = 1 - (1-1+e^{-\alpha x})^n = 1 - e^{-\alpha x} \Rightarrow X_{(1)} \sim E(n\alpha)$
%%\end{ex}

%%\begin{theorem}
%%	$X_1, X_2, \dots, X_n$ - независимая выборка из з.р. с плотностью $p(x)$.
%%	тогда $p_{X_{(k)}} = n C_{n-1}^{k-1} (F(x))^{k-1} (1-F(x))^{n-k} p(x)$.
%%\end{theorem}

%%\begin{proof}
%%	\[
%%		P(x\leqslant X_{(k)} < x+\Delta x) = \dfrac{n!}{(k-1)! (n-k)!} (F(x))^{k-1} p(x) \Delta x (1-F(x+\Delta x))^{n-k}
%%	\]
%%\end{proof}

\subsection{Методы построения точечных оценок параметров}
\subsubsection{Мотивация}
Пусть
\[
	X_1, X_2, \dots, X_n \sim F(x, \theta_1, \theta_2, \dots, \theta_n),
\]
то есть случайная величина выборки распределена по некоторому
\textsl{неизвестному} закону, зависящему от \textsl{неизвестных} параметров
$\bar \theta = (\theta_1, \theta_2, \ldots, \theta_n) $.

\textsc{Задача}. Оценить значения неизвестных параметров.


\subsubsection{Метод моментов}
\begin{definition}
Назовём величины
\begin{align*}
\hat{\mu}_k &= \frac{1}{n} \sum X_i^k, \\
\mu_k &= \int\limits_{-\infty}^{+\infty} x^k dF(x, \bar{\theta}).
\end{align*}
соответственно \emph{эмпирическим} и \emph{теоретическим} моментом $ k $-го
порядка.
\end{definition}

По закону больших чисел $\hat{\mu}_k \toPN \mu_k$. Составим тогда систему
уравнений
\[
\begin{cases}
  \hat{\mu}_1 = \mu_1(x, \bar{\theta}), \\
  \hat{\mu}_2 = \mu_2(x, \bar{\theta}), \\
  \ldots = \ldots, \\
  \hat{\mu}_r = \mu_r(x, \bar{\theta}).
\end{cases}
\]
Её решением будет некоторая оценка параметра $ \bar \theta $.

\begin{ex}
  Пусть выборка $X_1, \dots, X_n \sim E(\alpha)$ распределена по
	экспоненциальному закону.

  Тогда
	\[
		\hat \mu_1 = \bar{X} = \frac{1}{\alpha} = \mu_1 \Rightarrow \alpha =
		\frac{1}{\bar{X}},
	\]
	причём данная оценка является асимптотически несмещённой. % у всех почему-то
	% "асимптотически"
  \begin{proof}
    $X_k \sim \alpha e^{-\alpha x} \equiv \gamma_{\alpha, 1}$
		Тогда
		\[
			\xi = \sum_{k=1}^n X_k \sim \gamma_{\alpha, n}.
		\]

		Заметим, что $ \hat \alpha_n = 1/\bar X = n/\xi $, а значит,
		\[
			\mathsf M \hat \alpha_n = \mathsf M \frac{n}{\xi} = \int\limits_0^{+\infty} \frac{n}{x}
			\frac{\alpha^n x^{n-1} e^{-\alpha x}}{\Gamma(n)} \, dx = \frac{n
			\alpha}{\Gamma(n)} \int\limits_0^{+\infty} (\alpha x)^{n-2} e^{-\alpha x}
			\, dx = \frac{n\alpha \Gamma(n-1}{\Gamma(n)} = \frac{n\alpha}{n-1},
		\]
		откуда сразу вытекает асимптотическая несмещённость полученной оценки.

  \end{proof}
\end{ex}


\subsubsection{Оценки максимального правдоподобия}
Пусть выборка распределена одним из двух возможных способов
\[
	X_1, X_2, \dots, X_n \sim
  \begin{cases}
    p(x, \bar\theta), &\text{(абсолютно непрерывный случай)}, \\
    P_{\bar \theta}(\xi = X_k), &\text{(дискретный случай)} \end{cases}.
\]
\begin{definition}
	Назовём \emph{функцией правдоподобия} функцию, равную
	\[
		\mathscr{L} (X_1, X_2, \dots, X_n, \hat\theta) = 
		\prod_{k=1}^n p(X_k, \hat \theta)
	\]
в непрерывном случае и  
\[
	\mathscr{L} (X_1, X_2, \dots, X_n, \hat\theta) = P_{\bar\theta}(\xi_1 =
		X_1, \xi_2 = X_2, \dots, \xi_n = X_n) = \prod_{k=1}^n P_{\bar\theta}(\xi_k = X_k)
\]
в дискретном случае.
\end{definition}

  $\mathscr{L}$ исследуется на максимум по параметру $\theta$. При этом зачастую
	удобнее рассматривать не саму функцию $ \mathscr L $, а её логарифм $ \ln
	\mathscr L $, в связи с чем
	заметим, что
	\[
		\frac{\partial \mathscr{L}}{\partial \theta} = 0 \Leftrightarrow
		\frac{\partial \ln \mathscr L}{\partial \theta} = \frac{1}{\mathscr{L}}
		\frac{\partial \mathscr{L}}{\partial \theta} = 0,
	\]
что ясно и без привлечения аппарата дифференцирования (функция $ \ln x $
монотонно возрастающая).
\begin{ex}
	Пусть $ X_1, X_2, \dots, X_n \sim \operatorname{Pois}(\lambda) $.

	Тогда
	\[
		P(\xi=x_k) = \frac{\lambda^{X_k}}{(X_k)!} e^{-\lambda},
	\]
а значит,
\[
	\mathscr{L}(X_1, \dots, X_n, \lambda) = \prod_{k=1}^n \frac{\lambda^{X_k}}{(X_k)!}
	e^{-\lambda} = \frac{\lambda^{\sum X_k}\cdot
	e^{-n\lambda}}{\prod_{k=1}^n (X_k)!}
\]

Вычислим теперь
\[
	\ln \mathscr{L} = \sum_{k=1}^n X_k \ln\lambda - n \lambda - \ln \prod_{k=1}^n (X_k)!,
\]
откуда
\[
	\frac{\partial \ln \mathscr{L}}{\partial \lambda} = \frac{\sum_{k=1}^n X_k}{\lambda}
	- n = 0 \Rightarrow \bar\lambda_n = \frac{1}{n} \sum_{k=1}^n X_k.
\]
\end{ex}

\begin{ex}
  Пусть $X_1, X_2, \dots, X_n \sim E(\alpha)$.

	Имеем
	\begin{align*}
		\mathscr{L}(X_1, \dots, X_n, \alpha) &= \prod_{k=1}^n \alpha e^{-\alpha X_k}
		= \alpha^n e^{-\alpha \sum x_k},\\
		\ln \mathscr{L} &= n \ln\alpha - \alpha \sum_{k=1}^n X_k, \\
		\frac{\partial \ln\mathscr{L}}{\partial \alpha} &= \frac{n}{\alpha} -
		\sum_{k=1}^n
		X_k = 0,\\
		\alpha &= \frac{1}{\bar X}.
	\end{align*}
	Таким образом, \textsl{в данном случае} \boxed{\text{ОММ} = \text{ОМП}.}
\end{ex}


\subsubsection{Сравнение оценок}
\begin{definition}
	Пусть $\hat \theta_n$ и $\hhat\theta_n$ --- две \textsl{несмещенные} оценки параметра $\theta$.
	Если
	\[
		\mathsf D \hat\theta_n < \mathsf D \hhat\theta_n,
	\]
	то говорят, что $\hat\theta_n$
	\emph{более эффективна}, чем $\hhat\theta_n$.
\end{definition}

\begin{ex}
  Пусть $X_1, X_2, \dots, X_n \sim R[\theta-\frac{1}{2}, \theta+\frac{1}{2}]$ и
\[
	\hat\theta_n = \frac{1}{2} (X_{(1)} + X_{(n)}), \qquad
	\hhat\theta_n = \bar X,\\
\]

	Тогда
	\begin{gather*}
	\D\hat\theta_n = \frac{1}{4} \left[ \M X^2_{(1)} + \M X_{(n)}^2 + 2\M
	X_{(1)}X_{(n)} \right] - \theta^2 = \frac{1}{8 (n+1) (n+2)} \sim \frac{c}{n^2},\\
	\D\hhat\theta_n = \frac{\sigma^2}{n} = \frac{1}{12n}.
\end{gather*}
\end{ex}

\begin{theorem}[неравенство Рао -- Крамера]
  Пусть $X_1, X_2, \dots, X_n$ --- выборка из закона распределения с плотностью
	$p(x, \theta)$ и интеграл $\int_{\mathbb R^n} p(x, \theta) \, d\bar x = 1$ допускает дифференцирование
	по $\theta$\footnote{Регулярное семейство}. 
	%TODO: что значит регулярное семейство?
  
	Тогда для любой несмещенной оценки $\hat\theta_n = \varphi(X_1, X_2, \ldots,
	X_n)$ имеет место неравенство
  \[
		\D\hat\theta_n \geqslant \frac{1}{I_n(\theta)} = \frac{1}{n I_1(\theta)},
	\]
  где функцию
	\[
		I_n(\theta) = \M\left( \dfrac{\partial \ln \mathscr{L}}{\partial
		\theta} \right)^2
	\]
	называют \emph{информацией Фишера}, а функция
	\[
		I_1(\theta) = M\left( \dfrac{\partial\ln p(x, \theta)}{\partial\theta}
		\right)^2 = \int\limits_{-\infty}^{+\infty} \left( \frac{\partial \ln p(x,
		\theta)}{\partial \theta} \right)^2 p(x, \theta) \,dx
	\]
	есть информация Фишера в одном наблюдении.
\end{theorem}
\begin{remark*}
	Для дискретной модели следует заменить плотность $ p(x,\theta) $ на
	вероятность $ P(X = x_k) $.
\end{remark*}
\begin{proof}
	Продифференцируем по параметру очевидные интегральные соотношения
	\begin{multline*}
		1 = \int\limits_{\mathbb R^n} p(\bar x, \theta) \, d\bar x \Rightarrow 0 =
		\int\limits_{\mathbb R^n} \frac{\partial p(\bar x, \theta)}{\partial\theta} \,
		d\bar x = \\ = \int\limits_{\mathbb R^n} \frac{\partial \ln p(\bar x,
		\theta)}{\partial\theta} p(\bar x, \theta) \, d\bar x = \M \left( \frac{\partial
		\ln p(X_1, X_2, \ldots, X_n, \theta)}{\partial \theta} \right).
	\end{multline*}
	Тогда
	\[
		\M \hat\theta_n = \theta = \int\limits_{\mathbb R^n} \varphi(\bar x)
		p(\bar x, \theta) \, d\bar x \Rightarrow 1 = \int\limits_{\mathbb R^n}
		\varphi(\bar x) \frac{\partial \ln
		p(\bar x, \theta)}{\partial\theta} p(\bar x, \theta) \, d\bar x.
	\]

	Отсюда следует, что
	\[
		1 = \int\limits_{\mathbb R^n} (\varphi(\bar x)-\theta) \frac{\partial \ln
		p(\bar x, \theta)}{\partial
		\theta} p(\bar x, \theta) \, d\bar x,
	\]
а значит, с учётом неравенства Коши -- Буняковского
		\begin{gather*}
		1 \leqslant \int\limits_{\mathbb R^n} (\varphi(\bar x) - \theta)^2 p(\bar x,
		\theta) \, d\bar x \cdot
		\int\limits_{\mathbb R^n} \left(\frac{\partial\ln p(\bar x,
		\theta)}{\partial\theta}\right)^2 p(\bar x, \theta)
		\, d\bar x = \D \hat \theta_n \cdot I_n(\theta),\\
		I_n(\theta) = \M \left(\frac{\partial\ln \mathscr L(X_1, X_2, \dots, X_n,
		\theta)}{\partial\theta}\right)^2 = \D \frac{\partial \ln \mathscr L}{\partial\theta}
		= \sum_{k=1}^n \D\frac{\partial \ln p(X_k, \theta)}{\partial\theta} = n
		I_1(\theta),
	\end{gather*}
	поскольку
	\[
		\ln \mathscr L(X, \theta) = \sum_{k=1}^n \ln p(X_k, \theta).
	\]
\end{proof}

\begin{ex}
  Пусть $X_1, \dots, X_n \sim R[\theta-1/2, \theta+1/2]$.

	Тогда
	\[
		p(x, \theta) = I(\theta-1/2 \leqslant x \leqslant \theta+1/2).
	\]
При этом интеграл
\[
	1 = \int\limits_{-\infty}^{+\infty} I(x) \, dx
\]
не допускает дифференцирования по параметру.
\end{ex}

\begin{ex}
	Пусть  $X_1, X_2, \dots, X_n \sim E(\alpha)$.

	Тогда
	\begin{align*}
		p(x, \alpha) &= \alpha e^{-\alpha x},\\
		I_1(\alpha) &= \M \left(\frac{\partial\ln p(X_1, \alpha)}{\partial\alpha}\right)^2,\\
		\ln p(X_1, \alpha) &= \ln \alpha - \alpha X_1,\\
		\frac{\partial \ln p}{\partial\alpha} &= \frac{1}{\alpha} - X_1,\\
		I_1(\alpha) &= \M (\frac{1}{\alpha} - X_1)^2 = \D X_1 = \dfrac{1}{\alpha^2}.
  \end{align*}
\textsc{Вывод}. Для любой несмещенной оценки $\hat\alpha_n$
\[
	\D \hat\alpha_n \geqslant \frac{1}{n/\alpha^2} = \frac{\alpha^2}{n}.
\]
\end{ex}


