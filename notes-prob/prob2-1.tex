\chapter{Математические основания}
\section{Вероятностное пространство}
\subsection{Мотивация и несколько важных определений}
Рассмотрим эксперимент с подбрасыванием монеты \textsl{конечное} число раз $ n $ --- схему Бернулли $ (\Omega, \mathscr A, \mathsf P) $ , где, напомним,
\begin{align*}
	\Omega &= \{\omega\colon \omega = (a_1, \ldots, a_n),\; a_i = 0, 1\},\\
	\mathscr A &= \{A \colon A \subseteq \Omega\},\\
	\mathsf P(\{\omega\}) &= p^{\sum a_i}q^{n-\sum a_i}.
\end{align*}

Попытаемся обобщить эту вероятностную модель на случай $ n = \infty $.
Пространство $ \Omega $ тогда имеет мощность континуум. Его можно отождествить с
множеством точек $ \omega $ отрезка $ [0, 1) $. Возьмём наиболее простой
случай <<правильной>> монеты $ p = q = 1/2 $. Интуитивно ясно, что все
элементарные исходы $
\omega \in [0, 1) $ должны быть равновероятными. Однако, будь они
положительными, ввиду несчётности (даже бесконечности) $ \Omega $ мы получили бы противоречие $
\mathsf P(\Omega) = \infty \neq 1 $, поэтому для любого $ \omega $ имеем $
p(\omega) \equiv 0 $. Имея лишь такой результат, мы мало чего сможем достигнуть.
Например, мы не сможем подтвердить интуитивно ясное равенство $ \mathsf P([0, 1/2))
= 1/2$.

Из этих рассуждений становится ясно, что вероятность здесь следует определять не
только на точках, но на некоторых подмножествах $ \Omega $. Этот запас множеств
должен быть, конечно, замкнут относительно операций взятия объединения,
пересечения и дополнения, то есть должен быть

\begin{definition}
	\label{def:alg}
Система $ \mathscr A $ подмножеств некоторого множества $ \Omega $ называется
\emph{алгеброй}, если
\begin{enumerate}
\item $\Omega \in \mathscr A$ (единица),
	\item\label{enum:zamk} $ A, B \in \mathscr A \Rightarrow A \cup B \in \mathscr A, A \cap B \in
		\mathscr A$,
	\item\label{enum:3} $ A \in \mathscr A \Leftrightarrow \bar A \in \mathscr A $.
\end{enumerate}
\begin{remark*}При этом в условии \ref{enum:zamk} можно обойтись замкнутостью лишь по одной из
операций, а условие \ref{enum:3} можно заменить на условие
$ A, B \in \mathscr A \Rightarrow A \setminus B
\in \mathscr A $.
\end{remark*}

\end{definition}

На самом деле для хорошей вероятностной
модели лишь этих требований к $ \mathscr A $ недостаточно. 
По индукции условие \ref{enum:zamk} распространяется на любое конечное
объединение (пересечение) множеств. 

\begin{definition}
	К требованиям определения \ref{def:alg} добавим условиям
\begin{enumerate}
	\item[\ref{enum:zamk}'.] $ \forall i \; A_i \in \mathscr A \Rightarrow \bigcup\limits_{i=1}^\infty A_i \in
		\mathscr A $.
\end{enumerate}
(которое опять можно заменить аналогом с пересечением) и получим так
называемую \emph{$ \sigma $-алгебру}.
\end{definition}

\begin{definition}
Пару $ (\Omega, \mathscr A) $, где $ \mathscr A $ --- некоторая $ \sigma
$-алгебра, определённая на $ \Omega $, называют \emph{измеримым пространством}.
\end{definition}


Брать в качестве $ \mathscr A $ (как мы раньше это делали) систему подмножеств
$ \Omega $, которая, конечно, является $ \sigma $-алгеброй, неоправданно --- пришлось
бы определять значение меры $ \mathsf P $ на бесконечном числа множеств
бесконечного числа типов.
Оказывается, для того чтобы мера была определена (почти) на всех множествах,
достаточно определить её лишь на одном типе множеств (вернёмся к этому позже). На остальные множества
мера распространится самостоятельно, имея следущие важные \textbf{свойства
вероятностной меры}.

Во-первых, дадим само 
\begin{definition}
	Пусть дано измеримое пространство $ (\Omega, \mathscr A) $. Функцию множеств
	$ \mu(A) $, $ A \in \mathscr A $ со значениями в $ [0, +\infty] $ будем называть \emph{конечно-аддитивной
	мерой}, если для любых двух непересекающихся множеств $ A $ и $ B $ из $
	\mathscr A $ 
	\[
			\mu(A+B) = \mu(A) + \mu(B).
  \]
\end{definition}
\noindent Далее, однако, будет рассматриваться лишь
	\begin{definition}
	Конечно-аддитивная мера $ \mu $, заданная на алгебре $ \mathscr A $ называется
	\emph{счётно-аддитивной} (\emph{$ \sigma $-аддитивной}), или просто
	\emph{мерой}, если для любых попарно непересекающихся множеств $ A_i \in
	\mathscr A $ таких, что $ \sum\limits_{i=1}^\infty A_i \in \mathscr A $ (для
	$ \sigma $-алгебры выполняется всегда),\footnote{Здесь и далее символ $ \sum $
		(против $ \bigcup $)
	указывает на отсутствие пересечений у системы множеств}
	\[
		\mu \left( \sum_{i=1}^\infty A_i \right) = \sum_{i=1}^\infty \mu(A_n).
	\]
	
	Мера $ \mu $ называется \emph{$ \sigma $-конечной}, если пространство $ \Omega
	$ можно представить в виде 
	\[
		\Omega = \sum_{n=1}^\infty \Omega_n, \quad \Omega_n \in \mathscr A,
	\]
	где $ \mu(\Omega_n) < \infty $ для всех $ n $.

	Наконец, счётно-аддитивную меру $ \mathsf P $ назовём \emph{вероятностной
	мерой}, или \emph{вероятностью}, если $ \mathsf P(\Omega) = 1 $.
	\end{definition}
	
	Теперь перечислим собственно свойства:
	\begin{enumerate}
		\item \textsc{Сохранение нуля}.
			\[
				\mathsf P (\varnothing) = 0.
			\]
		\item \textsc{Исключение}. 
		\[
				\mathsf P(A \cup B) = \mathsf P(A) + \mathsf P(B) - \mathsf P(A\cap B).
		\]
	\item \textsc{Монотонность}. Если $ B \subseteq A $,
	\[
			\mathsf P(B) \leqslant \mathsf P(A).
	\]
\item \textsc{Неравенство треугольника}.  
\[
		\mathsf P(A_1 \cup A_2 \cup \ldots) \leqslant \mathsf P(A_1) +
		\mathsf P(A_2) + \ldots.
\]
\end{enumerate}
Три свойства из списка очевидны. \emph{Докажем} последнее. Для этого достаточно
заметить, что $ \bigcup_{n=1}^\infty A_n = \sum_{n=1}^\infty B_n $, где $ B_1 =
A_1$, $ B_n = \bar A_1 \cap \ldots \cap \bar A_{n-1} \cap A_n $. Тогда 
\[
	\Prob \left( \bigcup_{i=1}^n A_i \right) = \Prob \left( \sum_{n=1}^\infty
	B_n\right) = \sum_{n=1}^\infty \Prob(B_n) \leqslant \sum_{n=1}^\infty
	\Prob(A_n).
\]



\subsection{Одна важная теорема и основное определение параграфа}
\begin{theorem}
	Пусть $ \mathsf P $ --- конечно-аддитивная функция множеств, заданная на
	алгебре $ \mathscr A $, причём $ \mathsf P(\Omega) = 1 $. Тогда следующие
	четыре свойсвта эквивалентны:
	\begin{enumerate}[label={\rm\roman*)}]
		\item $ \mathsf P $ $ \sigma $-аддитивна (то есть являтся вероятностью),
		\item $ \mathsf P $ непрерывна сверху, то есть для любой системы множеств из
			$ \mathscr A $ такой, что $ A_n \subseteq A_{n+1} $, $
			\bigcup_{n=1}^\infty A_n \in \mathscr A $, 
			\[
				\lim_{n} \mathsf P(A_n) = \mathsf P \left( \bigcup_{n=1}^\infty A_n
				\right),
			\]
		\item $ \mathsf P $ непрерывна снизу, то есть для любой системы множеств из
			$ \mathscr A $ такой, что $ A_n \supseteq A_{n+1} $, $
			\bigcup_{n=1}^\infty A_n \in \mathscr A $, 
			\[
				\lim_n \mathsf P(A_n) = \mathsf P \left( \bigcap_{n=1}^\infty A_n
				\right),
			\]
		\item $ \mathsf P $ непрерывна в нуле, то есть для любой системы множеств из
			$ \mathscr A $ такой, что $ A_{n} \supseteq A_{n+1} $, $
			\bigcap_{n=1}^\infty A_n = \varnothing $, 
			\[
					\lim_n \mathsf P(A_n) = 0.
			\]
	\end{enumerate}
\end{theorem}
\begin{proof}
i) $ \Rightarrow $ ii). Действительно, 
\[
	\bigcup_{n=1}^\infty A_n = \sum_{n=1}^\infty (A_n \setminus A_{n-1}),
\]
где $ A_0 $ положено пустым множеством. В этом случае 
\begin{multline*}
	\Prob \left( \bigcup_{n=1}^\infty A_n \right) = \sum_{n=1}^\infty \left[ \Prob(A_n) -
	\Prob(A_{n-1}) \right] =\\= \lim_{k\to\infty} \sum_{n=1}^k \left[ \Prob(A_n) -
\Prob(A_{n-1}) \right] = \lim_{k\to\infty} \Prob(A_k).
\end{multline*}

ii) $\Rightarrow$ iii). Заметим, что согласно правилам де Моргана
\[
	\bigcap_{n=1}^\infty A_n = A_1 \setminus \bigcup_{n=1}^\infty \left[ A_1 \setminus A_n
	\right].
\]
Тогда 
\[
	\Prob \left( \bigcap_{n=1}^\infty A_n \right) = \Prob(A_1) - \lim_{n\to\infty}
	\Prob(A_1 \setminus A_n) = \lim_{n\to\infty} \Prob(A_n).
\]

iii) $ \Rightarrow $ iv). Очевидным образом вытекает из предшествующих
рассуждений.

Наконец, iv) $ \Rightarrow $ i). Рассмотрим систему непересекающихся множеств $
\{A_i\}$.  
\[
	\Prob\left(\sum_{n=1}^\infty A_n\right) = \Prob \left( \sum_{n=1}^k A_n
	\right) + \Prob \left( \sum_{n=k+1}^\infty A_n \right).
\]
При этом $ \sum_{n=k+1}^\infty A_n $ монотонно убывает к пустому множеству при
$ n \to\infty $. Поэтому 
\begin{multline*}
	\sum_{i=1}^\infty \Prob(A_i) = \lim_n \sum_{i=1}^n \Prob(A_i) = \lim_n \Prob
	\left( \sum_{i=1}^\infty A_i \right) = \\ =
	\lim_n \left[ \Prob \left( \sum_{i=1}^\infty A_i \right) - \Prob \left(
	\sum_{i=n+1}^\infty A_i \right)  \right] =\\=
	\Prob \left( \sum_{i=1}^\infty A_i \right)  -\lim_n \Prob \left(
	\sum_{i=n+1}^\infty A_i \right) = \Prob \left( \sum_{i=1}^\infty A_i \right).
\end{multline*}


\end{proof}

\begin{definition}[основное]
	Набор объектов  
	\[
		(\Omega, \mathscr F, \Prob),
	\]
	где 
	\begin{enumerate}[label=\alph*)]
		\item $ \Omega $ --- множество точек $ \omega $,
		\item $ \mathscr F $ --- $ \sigma $-алгебра $ \Omega $,
		\item	$ \Prob $ --- вероятность на $ \mathscr F $,
	\end{enumerate}
	называется \emph{вероятностной моделью} (эксперимента), или
	\emph{вероятностным пространством}. При этом пространство \emph{исходов}
	$\Omega $ называется \emph{пространством элементарных событий}, множества $ A
	$ из $ \mathscr F $ --- \emph{событиями}, а $ \Prob(A) $ --- вероятностью
	события $ A $.
\end{definition} 
