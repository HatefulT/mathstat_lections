\section{Измеримые пространства}
\begin{lemma}
  Пусть $ \mathscr E $ --- некоторая система подмножеств из $ \Omega $. Тогда
  существуют наименьшая\footnote{то есть такая, содержащая все множества из $
  \mathscr E $, алгебра, которая содержится в любой другой подобной.} алгебра $ \alpha(\mathscr E) $ и наименьшая $ \sigma
  $-алгебра $ \sigma(\mathscr E) $, содержащие все множества из $ \mathscr E $.
\end{lemma}
\begin{proof}
  Класс всех подмножеств $ \Omega $ содержит $ \mathscr E $ и при этом является
  и алгеброй, и $ \sigma $-алгеброй, поэтому имеется как минимум один претендент
  на искомое. Легко также видеть, что условия определяют $ \alpha(\mathscr E) $
  и $ \sigma(\mathscr E) $ однозначно. 

  Рассмотрим множество $ \widetilde \Omega = \bigcup\limits_{A\in\mathscr E} A $ и алгебру ($ \sigma
  $-алгебру) всех его подмножеств. Оттуда вычленим систему $ \Sigma $ всех
  алгебр ($ \sigma $-алгебр), содержащих все множества $ \mathscr E $.
  Пересечение $ \bigcap\limits_{B\in\Sigma} B $, очевидно, и будет искомой алгеброй ($
  \sigma $-алгеброй).
\end{proof}

Для эффективного пользования алгебрами хотелось бы научиться строить минимальную
$ \sigma $-алгебру или хотя бы определять, является ли данная система таковой.
Займёмся (обозримым) вторым вопросом и введём следующее
\begin{definition}
  Система $ \mathscr M $ подмножеств $ \Omega $ называется \emph{монотонным
  классом}, если утверждения\footnote{по поводу обозначений см. сводку
  обозначений на странице \pageref{sec:tos}.}
  \begin{gather*}
      A_n \in \mathscr M, \quad n=1,2,\ldots,\\
      A_n \uparrow A \quad\text{или}\quad A_n \downarrow A
  \end{gather*}
  влекут $ A \in \mathscr M $.

  Иными словами, монотонный класс замкнут относительно операций $ \uparrow $, $
  \downarrow$.
\end{definition}

\begin{lemma}\label{the:monoton}
  Для того чтобы алгебра $ \mathscr A $ была в то же время и $ \sigma
  $-алгеброй, необходимо и достаточно, чтобы она была монотонным классом.
\end{lemma}
\begin{proof}
  Прямая импликация очевидна, проверим обратную. И вправду, алгебре не хватает замкнутости по бесконечным объединениям. Однако
  любое бесконечное объединение можно считать монотонным. Например, положив $
B_n = \bigcup\limits_{i=1}^n A_i \in \mathscr A $, получим $ B_n \subseteq B_{n+1} $,
при этом $ B_n \uparrow \bigcup\limits_{i=1}^\infty A_i \in \mathscr A $.
\end{proof}
Обозначим минимальный монотонный класс, порождённый системой множеств $ \mathscr
E$ символом $ \mu(\mathscr E) $. Доказательство существования проводится
аналогично.

\begin{theorem}
  Пусть $ \mathscr A $ --- алгебра. Тогда 
  \[
      \mu(\mathscr A) = \sigma(\mathscr A).
  \]
\end{theorem}
\begin{proof}
  Из лемм выше видно, что достаточно доказать, что $\mathscr M = \mu(\mathscr A) $ есть
  алгебра. Займёмся этим.

  Единица алгебры сохраняется. Возьмём $ A \in \mathscr M $ и покажем, что $ \bar A \in \mathscr M $.
  Рассмотрим нетривиальный случай $ A \notin \mathscr A $. Тогда $ A_i \uparrow
  A$ (случай $ A_i \downarrow A $ рассматривается аналогично) для некоторых $ A_i \in \mathscr A $, иначе была
  бы потеряна минимальность. Но 
  \[
    \bar A =\overline{\bigcup_{i=1}^\infty A_i} = \bigcap_{i=1}^\infty \bar A_i
    \in \mathscr M,
  \]
  поскольку $ \bar A_i \subseteq \bar A_{i-1} $, то есть система множеств $ \bar
  A_i$ монотонно убывает.

  Покажем, что из $ A \in \mathscr M $, $ B \in \mathscr M $ следует, что $ A
  \cup B \in \mathscr M $. Опуская тривиальный случай, будем считать, что хотя
  бы одно из множеств принадлежит системе $ \mathscr M \setminus \mathscr A $.
  Пусть $ A_i \uparrow A $, $ A_i \in \mathscr A $, а $ B \in \mathscr A $. Но
  тогда 
  \[
  (A_i\cup B) \uparrow (A\cup B) \Rightarrow (A\cup B) \in \mathscr M. 
  \]
  Не упомянутые случаи
  рассматриваются аналогично. Например, если положить $ A_i \downarrow A $, $
  B_i \uparrow B $, то $ (\bar A \cup \bar B_i) \downarrow (\bar A
  \cup \bar B) = \overline{A \cap B} \in \mathscr M $, поскольку для любого $ i
  $ выполняется $ \bar A \cup \bar B_i \in \mathscr M $ (см. выше). Значит, и $ A \cap B \in
  \mathscr M$.
\end{proof}

\begin{definition}[<<$ \pi $-$ \lambda $-системы>>]
  Пусть $ \Omega $ --- некоторое пространство. Система $ \mathscr P $
  подмножеств $ \Omega $ называется \emph{$ \pi $-системой}, если она замкнута
  относительно конечных пересечений.
  \begin{enumerate}
    \item[$\pi.$] ($ A, B \in \mathscr P) \Rightarrow (A \cap B) \in \mathscr P
      $.
  \end{enumerate}

  Система $ \mathscr L $ подмножеств $ \Omega $ называется \emph{$ \lambda
  $-системой}, если 
  \begin{enumerate}[label=$ \lambda_\alph* $.]
    \item $ \Omega \in \mathscr L $.
    \item ($ A, B \in \mathscr L $ и $ A \subseteq B $) $ \Rightarrow $ ($ B
      \setminus A \in \mathscr L$).
    \item\label{enum:lam3} ($ A_n \in \mathscr L $, $ n \in \mathbb N $, и $ A_n \uparrow A $) $
      \Rightarrow $ ($ A \in \mathscr L $).
  \end{enumerate}

  Наконец, система $ \mathscr D $ подмножеств $ \Omega $ называется \emph{$ d
  $-системой Дынкина}, или просто $ \pi $-$ \lambda $-системой, если она
  является одновременно и первой, и второй.
\end{definition}
\begin{remark*}
  Условия $ \lambda_b $, $ \lambda_c $ в совокупности можно заменить
  условиями
  \begin{enumerate}
    \item[$ \lambda'_b. $]\label{enum:lam2bis} $ A \in \mathscr L \Rightarrow \bar A\in\mathscr L $.
    \item[\lambda'_c.] ($ A_n \in \mathscr L $, $ n\in\mathbb N $) $ \Rightarrow
      $ $ \bigsqcup A_n \in \mathscr L $.
  \end{enumerate}
  \begin{proof}
  $ \lambda'_b $. Дело в том,
    что  
    \[
      B \setminus A = \overline{A\sqcup \overline B}
    \]
    при $ A \subseteq B $.
    
    $ \lambda'_c $. Пусть $ A = \bigsqcup A_n $. Положим $ B_k =
    \bigsqcup\limits_{n=1}^k A_n $,
    тогда $ B_k \uparrow A $. Аналогично проводится доказательство в обратную
    сторону, где полагается $ B_n = A_n \setminus \bigcup\limits_{i=1}^{n-1} A_i $.
  \end{proof}
\end{remark*}
Таким образом, $ \lambda $-система есть просто совокупность множеств с единицей,
замкнутая относительно бесконечного объединения попарно непересекающихся множеств
и операции дополнения.

Обсудим эти понятия. И $ \pi $-система, и $ \lambda $-система могут даже не быть
алгебрами. При этом очевидно, что любая алгебра (или $ \sigma $-алгебра) является $ \pi $-системой, а каждая
$ \sigma $-алгебра (но не просто алгебра) является $ \lambda $-системой.
Оказывается, что верна 

\begin{theorem}[о $ \pi$-$ \lambda $-системах]\vphantom{I}
  \begin{enumerate}
    \item\label{enum:pl-the-1} Всякая $ d $-система $ \mathscr E $ является $ \sigma $-алгеброй.
    \item\label{enum:pl-the-2} Пусть $ \mathscr E $ есть $ \pi $-система множеств. Тогда $
      \lambda(\mathscr E) = d(\mathscr E) = \sigma(\mathscr E) $.
  \end{enumerate}
  \begin{proof}
    \ref{enum:pl-the-1}. Система $ \mathscr E $, очевидно, является алгеброй. Докажем, что она
     замкнута относительно бесконечных объединений. Для этого достаточно
     вспомнить лемму \ref{the:monoton} и свойство \ref{enum:lam3} Действительно,
     при $ A_n \downarrow A$ последнее множество также принадлежит $ \mathscr E
     $, поскольку выполняется свойство $ \lambda'_b $, и $ \bar A_n
     \uparrow \bar A $.

     \ref{enum:pl-the-2}. Всякая $ \sigma $-алгебра, как уже отмечалось,
     является $ \lambda $-системой. Из этого заключения вместе с пунктом
     \ref{enum:pl-the-1} вытекает, что достаточно доказать, что система $
     \lambda(\mathscr E) $ является $ \pi $-системой.

     Ряды $ \pi $-системы $ \mathscr E \supseteq A,\; B,\; C_i,\; D_i$ пополнили множества $ \bar A
     $, $ \bar B $, а также $ C = \bigsqcup C_i $ и $ D = \bigsqcup D_i $.
     Больше в $ \lambda(\mathscr E) $ никаких множеств нет, иначе потерялась бы
     минимальность. При этом, например,
     \[
       \bar A \cap \bar B = \overline{A \cup B} = \overline{A \sqcup (B
       \setminus (A\cap B))},
     \]
     а  
     \[
       \left(\bigcup_{i=1}^k C_i \cap \bigcup_{i=1}^k D_i\right) \uparrow (C \cap D).
     \]
       Остальное проверяется аналогично.
  \end{proof}
\end{theorem}

\begin{lemma}
  Пусть $ \Prob $ и $ \mathsf Q $ --- две вероятностные меры, заданные на
  измеримом множестве $ (\Omega, \mathscr F) $. Пусть $ \mathscr E $ ---
  некоторая $ \pi $-система множеств из $ \mathscr F $, на которой указанные
  меры совпадают. Тогда они совпадают и на $ \sigma $-алгебре $ \sigma(\mathscr
  E)$.
\end{lemma}
\begin{proof}
  Достаточно доказать, что при данных условиях система множеств $ \mathscr L $, на которой
  совпадают меры $ \mathsf P $ и $ \mathsf Q $, является $ \lambda
  $-системой. В этом случае,
  следует из теоремы выше, $ \sigma(\mathscr E) \subseteq \mathscr L $, что и
  требовалось доказать.

  Действительно, условия $ \lambda_a $, $ \lambda'_b $ и $ \lambda'_c $
  выполняются на $ \mathscr L $. Например, 
  \[
    \mathsf P\left(\bigsqcup_{n=1}^\infty A_n\right) = \sum_{n=1}^\infty \mathsf P(A_n) =
    \sum_{n=1}^\infty \mathsf Q(A_n) = \mathsf Q\left(\bigsqcup_{n=1}^\infty A_n\right).
  \]
  Остальные свойства очевидны. Поэтому $ \mathscr L $ является $ \lambda
  $-системой.
\end{proof}

\begin{lemma}
  Пусть $ \mathscr A_1,\, \mathscr A_2,\, \ldots,\, \mathscr A_n $ --- независимые
  алгебры событий. Тогда независимыми будут и $ \sigma $-алгебры $ \sigma(\mathscr
  A_1)$, $ \sigma(\mathscr A_2) $, $ \ldots $, $ \sigma(\mathscr A_n) $.
\end{lemma}
\begin{proof}
  Пусть при $ n \neq k $
  \begin{align*}
    A &= \bigsqcup_{k=1}^\infty A_k, \quad A_k \in \mathscr A_1,\\
    B &= \bigsqcup_{n=1}^\infty B_k, \quad B_k \in \mathscr A_2.
  \end{align*}
  Тогда
  \begin{multline*}
    \mathsf P(AB) = \mathsf P \left( \left(\bigsqcup_{i=1}^\infty
      A_i\right) \cap \left( \bigsqcup_{j=1}^\infty B_j \right) 
    \right) =\\= \mathsf P \left( \bigsqcup_{i, j} (A_i \cap B_j) \right) =
    \sum_{i, j}^\infty \mathsf P(A_i)\mathsf P(B_j) =
    \mathsf P(A) \Prob(B).
  \end{multline*}
  Аналогичная выкладка верна и для более, чем двух множителей.

Кроме того, 
\begin{align*}
  \mathsf P(\bar A \bar B) = 1 - \mathsf P(A \cup B) &=\\&=
    1 - \left( \mathsf P(B)
    + \mathsf P(A) - \mathsf P(A\cap B)\right)  = \mathsf P(\bar A)
    \mathsf P(\bar B)
\end{align*}
и 
\[
    \mathsf P(\bar A B) = \mathsf P(B\setminus (A\cap B)) =\mathsf P(B) -
    \mathsf P(AB) = \mathsf P(\bar A)\mathsf P(B).
\]

Исходя из этого, ясно, что системы $ \lambda(\mathscr A_1),\,
\lambda(\mathscr A_2),\, \ldots,\, \lambda(\mathscr A_n) $ также независимы.
\end{proof} %TODO: проверить доказательство

\begin{theorem}
  Пусть $ \mathscr E $ есть $ \pi $-система множеств из $ \mathscr F $ и $
  \mathscr H $ --- совокупность тех действительнозначных $ \mathscr F
  $-измеримых функций, для которых выполнены следующие свойства:
  \begin{enumerate}
    \item если $ A \in \mathscr E $, то функция $ I_A \in \mathscr H $;
    \item если $ f \in \mathscr H $, $ h \in \mathscr H $, то $ f + h \in
      \mathscr H $ и $ cf \in \mathscr H $ для всякого $ c \in \mathbb R $;
    \item если функции $ h_n \in \mathscr H $, $ n \in \mathbb N $, $ 0
      \leqslant h_n \uparrow h $, то $ h \in \mathscr H $.
  \end{enumerate}
      Тогда класс $ \mathscr H $ содержит и все ограниченные функции, являющиеся
      измеримыми относительно $ \sigma $-алгебры $ \sigma(\mathscr E) $.
\end{theorem} % FIXME: зачем?


\subsection{Некоторые примеры}
Всякое борелевское множество измеримо по Лебегу, но обратное неверно.

\begin{lemma}
Произвольный элемент системы $ \sigma(\mathscr A) $ может быть представлен в
виде счётной последовательности операций взятия дополнения и объединения,
применённой к множествам из $ \mathscr A $ вместе с пустым множеством. 
\end{lemma}
\begin{proof}
От обратного. Изымем из $ \sigma(\mathscr A) $ все элементы, не удовлетворяющие
указанному условию. Полученная система множеств в условиях минимальности должна
потерять свойства $ \sigma $-алгебры. Но ни одно из её свойств быть потеряно не
может, иначе получим противоречие.
\end{proof}

\begin{theorem}\label{th:5}
 
\[
  \mathscr B(R^n) = \bigotimes_{k=1}^n \mathscr B(R).
\]

\end{theorem}
\begin{proof}
    Проведём доказательство по индукции. Для $ n=1 $ результат очевиден. Докажем
    для $ n=2 $.

    Очевидно, $ \mathscr B(R^2) \subseteq \mathscr B \otimes \mathscr B $.
    Докажем, что произвольный борелевский прямоугольник $ B = B_1 \times B_2 $
    принадлежит $
    \mathscr B(R^2) $, тем самым доказав всю теорему.

    Пусть $ B'_1 = B_1 \times R $ есть прообраз множества $ B_1 $ при
    проектировании на первую координату, а $ B'_2 = R \times B_2 $ --- прообраз
    $ B_2 $ при проектировании на вторую координату. Тогда $ B = B'_1
    \cap B'_2 $. Представим $ B_1 $, $ B_2 $ в виде счётного последовательного
    применения операций объединения и дополнения к системе интервалов $ I_i $.
    Эти операции по свойствам прообраза функции благополучно переносятся за прямое произведение, откуда и
    получаем результат теоремы.
\end{proof}

\begin{align*}
  \mathscr I(I_1 \times \ldots \times I_n) &= \{x\colon x = (x_1, x_2, \ldots),
  x_1 \in I_1, \ldots, x_n \in I_n\},\\
    \mathscr I(B_1 \times \ldots \times B_n) &= \{x\colon x = (x_1, x_2, \ldots),
  x_1 \in B_1, \ldots, x_n \in B_n\},\\
      \mathscr I(B^n) &= \{x\colon x=(x_1, x_2, \ldots), (x_1, \ldots, x_n) \in
      B^n\},
\end{align*}
где $ B^n $ --- борелевское множество из $ \mathscr B(R^n) $. Назовём
соответствующие минимальные $ \sigma $-алгебры $ \mathscr B(R^\infty) $, $
\mathscr B_1(R^\infty) $, $ \mathscr B_2(R^\infty) $ и докажем поочерёдно, что $ \mathscr
B(R^\infty) \subseteq \mathscr B_1(R^\infty) \subseteq \mathscr B_2(R^\infty)
\subseteq \mathscr B(R^\infty) $.

\begin{theorem}
  Тем самым будет доказано, что  
  \[
      \mathscr B(R^\infty) = \mathscr B_1(R^\infty) = \mathscr B_2(R^\infty).
  \]
\end{theorem} 
\begin{proof}
  Первое отношение очевидно. Ход доказательства второго отношения в точности
  повторяет ход доказательства теоремы \ref{th:5}, на который бесконечность
  прямого произведения никак не влияет. 

  Наконец, докажем последнюю из принадлежностей. Для этого докажем, что
  произвольный цилиндр $ \mathscr I(B^n) $ окажется в $ \sigma $-алгебре $
  \mathscr B(R^\infty) $. Спроектируем цилиндр на первые $ n $ координат и
  представим полученную проекцию $ B^n $ в виде последовательности операций дополнения и
  объединения, применённых к системе брусков $ I^n_i $. Прообраз при
  проектировании данного множества и приведёт к ответу.
\end{proof}


\subsubsection{Измеримое пространство $( R^T, \mathscr B(R^T) )$}
Здесь $ R^T $ мы называем совокупность действительнозначных функций,
определённых на произвольном множестве $ T $. Введём также обозначение $ x_t :=
x(t)$.

В этих обозначениях назовём циллиндрами следующие множества:

\begin{theorem}
  Докажем, что  
  \[
      \mathscr B(R^T) = \mathscr B_1(R^T) = \mathscr B_2(R^T).
  \]
\end{theorem}
\begin{proof}
  Доказательство полностью повторяет доказательства выше.
\end{proof}


\subsubsection{Измеримое пространство $ (C, \mathscr B(C)) $}
\begin{theorem}
  Докажем, что  
  \[
      \mathscr B(C) = \mathscr B_0(C).
  \]
\end{theorem}
\begin{proof}
$ \mathscr B(C) \subseteq \mathscr B_0(C) $. Действительно, пусть $ B =
\{x\colon x_{t_0} < b\} $ --- некоторое цилиндрическое множество. Оно, конечно,
открыто. Тогда открыто и множество $ \{x\colon x_{t_1} < b_1, \ldots, x_{t_n} <
b_n\} \in \mathscr B_0(C)$.

$ \mathscr B_0(C) \subseteq \mathscr B(C) $. Докажем, что любой открытый шар $
B_\rho $ будет принадлежать $ \mathscr B(C) $. 
\begin{multline*}
  B_\rho = \{y \in C\colon y \in S_\rho(x^0)\} = \left\{ y \in C\colon \max_t
  |y_t - x_t^0| < \rho \right\} =\\=
  \bigcap_{k} \left\{ y \in C\colon |y_{t_k} - x_{t_k}^0 | < \rho \right\} \in
  \mathscr B(C),
\end{multline*}
где пересечение берётся по всем рациональным точкам $ T $.
\end{proof}



\section{Способы задания вероятностных мер на измеримых пространствах}
Перед тем как ввести вероятностную меру на указанных выше измеримых
пространствах сделаем несколько важных замечаний.

\begin{theorem}[Каратеодори]\label{the:carath}
  Пусть $ \Omega $ --- некоторое пространство, $ \mathscr A $ --- 
алгебра его подмножеств и $ \mathscr B = \sigma(\mathscr A) $ --- наименьшая $
\sigma $-алгебра, содержащая $ \mathscr A $. Пусть $ \mu_0 $ --- $ \sigma
$-конечная мера (как $ \sigma $-аддитивная функция множеств) на $ (\Omega,
\mathscr A) $. Тогда существует и притом единственная мера $ \mu $  на $ (\Omega, \mathscr B) $, являющаяся продолжением $ \mu_0
$, то есть такая, что 
\[
    \mu(A) = \mu_0(A), \quad A \in \mathscr A.
\]
\end{theorem}
\begin{proof}
\end{proof}


\subsection{Функция распределения}
Рассмотрим простейшее измеримое пространство $ (R, \mathscr B(R)) $. 

\begin{definition}
  Назовём \emph{функцией распределения} функцию $ F\colon \mathbb R \to \mathbb
  R $, удовлетворяющую следующим \textbf{свойствам функции распределения}:
  \begin{enumerate}
    \item \textsc{Неубывание}.
    \item $F(-\infty) = 0$, $ F(\infty) = 1 $, где 
    \[
      F(-\infty) = \lim_{x\to-\infty} F(x), \quad F(\infty) = \lim_{x\to+\infty}
      F(x).
    \]
  \item \textsc{Непрерывность справа}. Функция $ F(x) $ непрерывна справа и
    имеет пределы слева в каждой точке $ x \in \mathbb R $.
  \end{enumerate}
\end{definition}

Пусть $ \mathsf P = \mathsf P(A) $ --- вероятностная мера, определённая на
борелевских множествах $ A \in \mathscr B(R) $ числовой прямой $ \mathbb R $. Положим 
\[
  F(x) = \mathsf P(-\infty, x], \quad x \in \mathbb R.
\]
\begin{proof}
  1. Очевидно.

  2. Рассмотрим, например, $\lim\limits_{x\to-\infty} \mathsf P(-\infty, x]$, а
  точнее \textsl{произвольную} подпоследовательность $ \lim\limits_{n\to\infty}
  \mathsf P(A_n)$, где $A_n := (-\infty, x(n)]$. Очевидно, $ A_n \downarrow
  \varnothing $, и доказываемое следует из теоремы о непрерывности меры (см.
  теорему \ref{the:nepr-P}).

  3. Аналогично рассмотрим произвольную подпоследовательность $ x(n) \to a+ $ и множества $ A_n := (-\infty, x(n)] $. Снова $
  A_n \downarrow A $, где $ A =  (-\infty, a] $ и снова доказательство
  непрерывности справа
  завершается ссылкой на теорему о непрерывности меры.

  Положив, однако, $ x(n) \to a- $, получим $ A_n \uparrow (-\infty, a) \neq A
  $. При этом мера одной точки может быть и не равной нулю $ \mathsf P(\{a\}) >
  0$. 
\end{proof}

Итак, каждой вероятностной мере соответствует ровно одна функция распределения.
Докажем, что верно и обратное. 
\begin{theorem}
  Пусть $ F = F(x) $ --- некоторая функция распределения. Тогда на $ (R,
  \mathscr B(R) ) $ существует и притом единственная вероятностная мера $
  \mathsf P $ такая, что для любых $ -\infty\leqslant a < b<\infty $ 
  \[
    \mathsf P(a, b] = F(b) - F(a).
  \]
\end{theorem}
\begin{proof}
Определим меру сначала на алгебре $ \mathscr A $ конечных сумм непересекающихся полуинтервалов
$ (a, b] $ прямой. На ней, конечно, 
\[
  \mathsf P_0(A) = \sum_{k=1}^n[F(b_k) - F(a_k)], \quad A \in \mathscr A.
\]
На алгебре $ \mathscr A $ эта формула однозначно определяет конечно-аддитивную функцию
множеств. Покажем, что $ \mathsf P_0(A) $ счётно-аддитивна и тем самым благодаря
теореме Каратеодори (см. теорему \ref{the:carath}) завершим
доказательство.

Согласно теореме о непрерывности меры (см. теорему \ref{the:nepr-P}) для этого
достаточно проверить непрерывность $ \mathsf P_0 $ в нуле, то есть показать,
что 
\[
    \mathsf P_0(A_n) \to 0, \quad A_n\downarrow\varnothing.
\]
Будем для начала считать, что система $ A_n $ ограничена, то есть каждое из
множеств принадлежит некоторому отрезку $ [-N, N] $. Заметим, что ввиду
непрерывности справа функции распределения $ F(x) $ верно 
\[
  \mathsf P_0(a', b] = F(b) - F(a') \to F(b) - F(a) = \mathsf P_0(a, b]
\]
при $ a' \to a+ $. Значит, для любого множества $ A_n $ ($ = (a, b] $) существует множество $
B_n $ ($ = (a', b] $) такое, что $ [B_n] \subseteq A_n $ и  
\[
  \mathsf P(A_n \setminus B_n) \leqslant \varepsilon 2^{-n}.
\]

По предположению $ \bigcap A_n = \varnothing $, а значит, и $ \bigcap [B_n] =
\varnothing $. Более того, найдётся конечное число $ n_0 $, такое что 
\[
  \bigcap_{n=1}^{n_0} [B_n] = \varnothing.
\]
Это --- следствие теоремы Гейне -- Борелля, в которой компакт $ [-N, N] $
покрывается системой открытых множеств $ [-N, N] \setminus [B_n] $.

Учитывая теперь, что $ A_n $ есть убывающая система, находим 
\begin{multline*}
  \mathsf P_0(A_{n_0}) = \mathsf P_0\left(A_{n_0} \setminus \bigcap_{k=1}^{n_0}
    B_k\right) \leqslant\\ \leqslant \mathsf P_0 \left( \bigcup_{k=1}^{n_0} (A_{n_0}
      \setminus
    B_k) \right) \leqslant \sum_{k=1}^{n_0} \mathsf P_0(A_k \setminus B_k)
    \leqslant \sum_{k=1}^{n_0} \varepsilon 2^{-k} \leqslant \varepsilon.
\end{multline*}

Откажемся теперь от предположения, что все $ A_n \subseteq [-N, N] $ для
некоторого $ N $. Зададим $ \varepsilon > 0 $ и выберем такое $ N $, что $
\mathsf P_0[-N, N] > 1 - \varepsilon/2 $. Тогда 
\[
  \mathsf P_0(A_n) = \mathsf P_0(A_n \cap [-N, N]) + \mathsf P_0(A_n \cap
  \overline{[-N, N]}) \leqslant \mathsf P_0(A_n \cap [-N, N]) + \varepsilon/2.
\]
Повторяя рассуждения выше, приходим к результату.
\end{proof}

\subsubsection{Три типа функций распределения и основные примеры}


\subsection{Мера на измеримом пространстве $ (R^n, \mathscr B(R^n)) $}
Положим, что $ \mathsf P $ --- некоторая вероятностная мера на $ (R^n, \mathscr
B(R^n))$.

Обозначим 
\[
  F_n(x_1, \ldots, x_n) = \mathsf P((-\infty, x_1] \times \ldots \times(-\infty,
  x_n]),
\]
или, в более компактной форме, 
\[
  F_n(x) = \mathsf P(-\infty, x_n].
\]

Введём разностный оператор $ \Delta_{a_ib_i}\colon R^n \to R $, действующий по
формуле ($ a_i \leqslant b_i $)  
\begin{multline*}
  \Delta_{a_ib_i} F_n(x_1, \ldots, x_n) = F_n(x_1, \ldots, x_{i-1}, b_i,
  x_{i+1}, \ldots, x_n) - \\ - F_n(x_1, \ldots, x_{i-1}, a_i, x_{i+1}, \ldots,
  x_n).
\end{multline*}







  




  

 






