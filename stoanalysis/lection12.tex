\section{Линейные СДУ с постоянными коэффициентами}

\begin{ex}
  \[
    d\xi_t = a \xi_t \, dt + b dW_t, \quad \xi_0 = \nu; a < 0
  \]
  где $\nu$ и $W_t$ -- независимы.

  Если бы $\nu$ была бы неслучайной, а $b = 0$, то это уравнение было бы обычным дифференциальным,
  а решение было бы равно
  \[
    \xi_t = e^{at} \left( \xi_0 + b \int\limits_0^t e^{-a\tau} \, dW_\tau \right),
  \]
  \begin{align*}
    M\left(e^{at}\nu\right) &= e^{at} M\nu \to 0; \quad D \left( e^{at} \nu \right) = e^{2at} D\nu \to 0 \\
    M \left( b\int\limits_0^t e^{a(t-\tau)} \, dW_t \right) &= 0; \quad 
    D \left( b\int\limits_0^t e^{a(t-\tau)} \, dW_t \right) = b^2 \int\limits_0^t e^{2a(t-\tau)} \, d\tau \to \dfrac{- b^2}{2 |a|}
  \end{align*}
  т.е. первое слагаемое не вносит вклад на $\infty$.
\end{ex}

\begin{definition}
  Система СДУ с постоянными коэффициентами называется асимптотически устойчивой, если все
  корни характеристического уравнения $|a - \lambda E| = 0$ имеют действительную часть меньше
  нуля.
\end{definition}

\[
  d\xi_t = a \xi_t \, dt + u \, dt + s \, dW_t; \quad \xi_0=\nu,
\]
где $\nu$ и $W_t$ -- независимы. Тогда решение выражается в виде:
\[
  \xi_t = \Theta(t) \left( \nu + u \int\limits_0^t \Theta^{-1}_\tau \, d\tau + s\int\limits_0^t \Theta^{-1}_\tau dW_\tau \right),
\]
где $\Theta_t$ -- решение обыкновенного дифференциального уравнения (матричного)
\[
  \Theta_t' = a \Theta_t, \quad \Theta_0 = E
\]

Характеристики этого решения удовлетворяют уравнениям метода моментов:
\[
  \begin{cases}
    \dfrac{dm_\xi(t)}{dt} = am_\xi(t) + u; \; m_\xi(0) = M\nu \\
    \dfrac{d\gamma_\xi(t)}{dt} = a \gamma_\xi(t) + \gamma_\xi(t) a^T + ss^T; \; \gamma_\xi(0) = \Sigma_\nu.
  \end{cases}
\]

\begin{theorem}
  Если система СДУ с постоянными коэффициентами асимптотически устойчива, то
  \[
    \forall M\nu, \Sigma_\nu :
    m_\xi(t) \to m_\xi = \operatorname{const}, \;
    \gamma_\xi(t) \to \gamma_\xi = \operatorname{const},
  \]
  при $t \to +\infty$. А также $m_\xi$ и $\gamma_\xi$ удовлетворяет стационарным уравнениям
  метода моментов:
  \[
    \begin{cases}
      0 = am_\xi + u; \\
      0 = a \gamma_\xi + \gamma_\xi a^T + ss^T
    \end{cases}
  \]
\end{theorem}

\begin{ex}
  Найдём стационарные характеристики $m_\xi$ и $\gamma_\xi$ для процесса, заданного СДУ:
  \[
    \begin{cases}
      d\xi_1(t) = \xi_2 \, dt + \delta \, dt, \\
      d\xi_2(t) = - \omega^2 \xi_1(t) \, dt - 2 \alpha \omega \xi_2(t) \, dt + g dW_t
    \end{cases}
  \]
  Перепишем уравнение в матричной форме:
  \[
    d \begin{pmatrix} \xi_1 \\ \xi_2 \end{pmatrix} =
    \begin{pmatrix} 0 & 1 \\ - \omega^2 & - 2 \alpha \omega \end{pmatrix} \cdot
    \begin{pmatrix} \xi_1 \\ \xi_2 \end{pmatrix} \, dt + 
    \begin{pmatrix} \delta \\ 0 \end{pmatrix} \, dt +
    \begin{pmatrix} 0 \\ g \end{pmatrix} \, dW_t
  \]

  Проверим асимптотическую устойчивость: составим характеристическое уравнение
  $|a - \lambda E| = 0$:
  \[
    \left| \begin{matrix} - \lambda & 1 \\ - \omega^2 & - 2 \alpha \omega - \lambda \end{matrix} \right| =
    \lambda^2 + 2\alpha\omega\lambda + \omega^2 =
    (\lambda + \alpha\omega)^2 + \omega^2 (1 - \alpha^2) = 0,
  \]
  и его решения: $\lambda_{1, 2} = - \alpha\omega \pm \sqrt{\omega^2 (\alpha^2 - 1)}$,
  и с учётом условия $\alpha\omega > 0$ получим следующее:
  \begin{center}
    \begin{tabular}{cc}
      % \hline
      $\omega^2 (\alpha^2 - 1) > 0$ & 2 действительных корня и оба меньше нуля \\
      $\omega^2(\alpha^2 - 1) < 0$ & 2 комплексно-сопряжённых корня с отрицательными действительными частями
      % \hline
    \end{tabular}
  \end{center}
  то есть эта система асимптотически устойчива.
  
  Тогда можем найти характеристики:
  \[
    \begin{cases}
      0 = am_\xi + u\\
      0 = a\gamma_\xi + \gamma_\xi a^T + ss^T
    \end{cases}
    \Rightarrow
    \begin{cases}
      m_\xi = \begin{pmatrix} \dfrac{2\alpha\delta}{\omega} \\ - \delta \end{pmatrix} , \\
      \gamma_\xi = \begin{pmatrix}
        \dfrac{g^2}{4\alpha\omega} & 0 \\
        0 & \dfrac{g^2}{4\alpha\omega^3}
      \end{pmatrix} 
    \end{cases}
  \]
\end{ex}


\begin{theorem}
  Если $\xi_t$ удовлетворяет условиям асимптотической устойчивости, $\nu$ не зависит от $W_s$, то
  характеристическая функция $\xi_t$ имеет вид:
  \[
    f_\xi(\lambda) = M e^{i\lambda^T \xi} = f_0 \left( \Theta_t^T \lambda\right) \exp \left\{ i\lambda^T \check{m}_\xi(t) - \dfrac{1}{2} \lambda^T \check{\gamma}_\xi (t) \lambda \right\},
  \]
  где $f_0$ -- характеристическая функция $\nu$;
  $\check{m}_\xi, \check{\gamma}_\xi$ -- характеристики процесса с нулевым начальным условием.
\end{theorem}

\textit{Пояснение.} Решение дифференциального уравнения:
\[
  \xi_t = \Theta_t \nu + \int\limits_0^t \Theta_t \Theta^{-1}_\tau u \, d\tau + \int\limits_0^t \Theta_t \Theta^{-1}_\tau s dW_\tau,
\]
характеристическая функция суммы независимых случайных величин равна произведению
характеристических функций этих случайных величин. Слагаемое посередине -- детерминированная часть,
и её характеристическая функция равна $e^{i\lambda \dots}$:
\begin{align*}
  M e^{i\lambda^T \Theta_t \nu} = f_\nu \left( \Theta_t^T \lambda \right), \\
  M e^{i\lambda^T \int\limits_0^t \Theta_t \Theta^{-1}_\tau s dW_\tau} = e^{i \lambda^T \Sigma \lambda/2}
\end{align*}

\begin{ex} Найти закон распределения сечения для процесса $\xi_t$:
  \[
    d\xi_t = a \xi_t \, dt + b \, dW_t, \quad \xi_0 \sim R[-\alpha, \alpha]
  \]
  
  \[
    \begin{cases}
      \Theta_t' = a \Theta_t, \\
      \Theta_0 = 1
    \end{cases}
    \Rightarrow
    \Theta_t = e^{at}
  \]
  \[
    f_0 (\lambda) = \int\limits_{-\alpha}^{\alpha} \dfrac{1}{2\alpha} e^{i\lambda x} \, dx = \dfrac{\sin(\alpha \lambda)}{\alpha\lambda}
  \]
  \[
    \begin{cases}
      \dfrac{d \check{m}_\xi}{dt} = a \check{m}_\xi + 0, \; \check{m}_\xi(0) = 0 \\
      \dfrac{d \check{\gamma}_\xi}{dt} = 2a \check{\gamma}_\xi + b^2; \; \check{\gamma}_\xi(0) = 0
    \end{cases}
    \Rightarrow
    \begin{cases}
      \check{m}_\xi(t) = 0, \\
      \check{\gamma}_\xi(t) = \dfrac{b^2}{2|a|} \left( 1 - e^{2at} \right) 
    \end{cases}
  \]

  Тогда:
  \[
    f_\xi(\lambda) = \dfrac{\sin\left(\alpha\lambda e^{at}\right)}{\alpha\lambda e^{at}} e^{- \dfrac{\lambda^2 b^2}{4|a|} (1 - e^{2at})}
  \]
  при $t \to + \infty$:
  \[
    f_{\xi_\infty}(\lambda) = e^{- \dfrac{b^2}{4|a|} \lambda^2} \sim N\left(0; \sqrt{\dfrac{b^2}{2|a|}}\right)
  \]

  \begin{remark*}
    Сечения $\xi_t$ при $t \to +\infty$ будут нормальными, если $\Theta_t \to 0$.
  \end{remark*}
\end{ex}


\section{Задача прогнозирования}

\[
  d\xi_t = A(t, \xi_t) \, dt + B(t, \xi_t) \, dW_t, \; \xi_0=\nu,
\]
$\nu$ и $W_t$ -- независимы.


