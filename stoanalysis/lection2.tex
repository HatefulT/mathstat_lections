\section{Мартингалы. Субмартингалы}

% TODO начало лекции

\paragraph{Свойства мартингалов}

\begin{enumerate}
  \item % TODO дописать

  \item Пусть $S_n = \sum_{j=0}^n \xi_j$, где $\xi_i$ -- попарно независимы,
    тогда $S_n$ -- мартингал $\Leftrightarrow M\xi_j = 0 \forall j$;
    \begin{proof}
      \begin{multline*}
        M(S_{n+1} | \mathcal{F}_n) = M(S_n + \xi_{n+1} | \mathcal{F}_n)
        = M(S_n | \mathcal{F}_n) + M(\xi_{n+1} | \mathcal{F}_n) = \\
        = S_n + M\xi_{n+1} \equiv S_n \Rightarrow M\xi_{n+1} = 0 \forall n.
      \end{multline*}
      (но $M\xi_0$ может быть не равен $0$)
    \end{proof}

  \item Если $\eta$ -- интегрируемая случайная величина ($M|\eta| < \infty$), тогда
    последовательность $\xi_n = M( \eta | \mathcal{F}_n )$ является мартингалом.
    \begin{proof}
      \[
        M(\xi_{n+1} | \mathcal{F}_n) = M( M(\eta | \mathcal{F}_{n+1}) | \mathcal{F}_n )
        = M(\eta | \mathcal{F}_n) = \xi_n.
      \]
    \end{proof}
\end{enumerate}

\begin{theorem}[Неравенство Йенсена]
  Если $g(x)$ -- выпуклая вниз функция, тогда $g(M\xi) \leqslant Mg(\xi)$, причём
  это неравенство сохраняется и для условного математического ожидания:
  пусть $\mathcal{A}$ -- некоторая произвольная алгебра, тогда
  $g(M(\xi|\mathcal{A})) \leqslant M(g(\xi) | \mathcal{A})$.

  На самом деле эта теорема является свойством интеграла Лебега -- свойством
  сходимости под интегралом Лебега.
\end{theorem}


\subsection{Субмартингалы, теорема Дуба, квадратическая характеристика}

\begin{definition}
  Стохастическая последовательность $\{Y_n\}_{n \geqslant 0}$ называется
  \emph{субмартингалом} относительно потока $ \left\{ \mathcal{F}_n \right\}_{n\geqslant 0} $,
  если
  \[
    M(Y_{n+1} | \mathcal{F}_n) \geqslant Y_n - \text{$P$-п.н.}.
  \]

  И называется \emph{супермартингалом}, если
  $M(Y_{n+1} | \mathcal{F}_n) \leqslant Y_n.$
\end{definition}


\begin{theorem}[Дуб]
  Всякий субмартингал раскладывается $P$-п.н. единственным образом в сумму мартингала
  и предсказуемой стохастической последовательности:
  \[
    Y_n = \underbrace{m_n}_{\text{мартингал}} + \underbrace{A_n}_{\text{предск.}}, n \geqslant 0.
  \]
  при этом $ \left\{ A_n \right\}  $ называется компенсатором $Y_n$.
\end{theorem}
\begin{proof}
  \[
    Y_n = Y_0 + \sum_{j=0}^{n-1} (Y_{j+1} - Y_j)
    = \underbrace{Y_0 + \sum_{j=0}^{n-1} (Y_{j+1} - M(Y_{j+1}|\mathcal{F}_j))}_{m_n}
    + \underbrace{\sum_{j=0}^{n-1} (M(Y_{j+1} | \mathcal{F}_j) - Y_j)}_{A_n},
  \]
  причём $A_n$ -- предсказуема, так как содержит $Y_j$ только до $Y_n$, и является
  $\mathcal{F}_n$-измеримой;
  $m_n$ -- мартингал, т.к.
  \[
    M( m_{n+1} | \mathcal{F}_n ) =
    M\left( m_n + Y_{n+1} - M(Y_{n+1} |\mathcal{F}_n) | \mathcal{F}_n \right) = m_n + 0.
  \]
\end{proof}

\begin{corollary}
  Если $\{m_n\}$ -- мартингал, тогда, в силу неравенства Йенсена,
  \[
    M(m_{n+1}^2 | \mathcal{F}_n) \geqslant \left( \underbrace{M(m_{n+1}|\mathcal{F}_n)}_{m_n} \right)^2
  \]
  или $Y_n = m_n^2$ -- субмартингал
\end{corollary}

\begin{definition}
  Компенсатор субмартингала $m_n^2$ называется \emph{квадратической характеристикой} $m_n$:
  \[
    \langle m \rangle_n = A_n, \quad 
    m_n^2 = \xi_k + A_k,
  \]
  причём
  \[
    \langle m \rangle_n =
    \sum_{k=0}^{n-1} \left[M( m_{k+1}^2 | \mathcal{F}_k) - m_k^2 \right] =
    \sum_{k=0}^{n-1} \left[ M( m_{k+1}^2 - m_k^2 | \mathcal{F}_k ) \right] =
    \sum_{k=0}^{n-1} \left[ M( (m_{k+1}^2 - m_k^2)^2 | \mathcal{F}_k ) \right] 
  \]
  (последнее несложно доказать, если раскрыть квадрат разности, а далее
  $M(m_k m_{k+1} | \mathcal{F}_k) = m_k^2$)
\end{definition}

\begin{ex}
  Пусть $m_k = \sum_{j=0}^k \xi_j$, где $\xi_0, \xi_1, \dots, \xi_n$ -- независимые
  случайные величины с $M\xi_i = 0, i \geqslant 1$.
  \[
    \langle m\rangle_n = \sum_{j=0}^{n-1} M \left( (m_{j+1} - m_j)^2 | \mathcal{F}_j \right) 
    = \sum_{j=0}^{n-1} M \left( \xi_{j+1}^2 | \mathcal{F}_j \right) 
    = \sum_{j=0}^{n-1} M\left(\xi_{j+1}^2\right) = \sum_{j=0}^{n-1} D\xi_{j+1}
  \]
  -- некоторая постоянная (если $M\xi_0 \neq 0$, то первое слагаемое не будет равно дисперсии
  $\xi_0$).
\end{ex}

\begin{theorem}
  Если последовательность $ \left\{ X_k \right\}_{k \geqslant 0} $ -- мартингал и 
  $\exists \alpha > 1 : \sup_n M|X_k|^\alpha < \infty$ (равномерно интегрируема ???), то
  $\exists X_{\infty}, $ что
  \begin{enumerate}
    \item $X_n \toPN X_\infty$;
    \item $X_n \xRightarrow[]{\text{ср.}} X_\infty$, ($M|X_n - X_\infty| \to 0$);
    \item $X_n = M(X_\infty | \mathcal{F}_n)$.
  \end{enumerate}

  Использовать эту теорему можно для проверки условий усиленного закона больших чисел.
\end{theorem}

\begin{lemma}[Кронекер]
  Пусть $\sum_{k=1}^\infty a_k < \infty$, $b_n \uparrow \infty$ (монотонно стремиться к $\infty$),
  тогда $\dfrac{1}{b_n} \sum_{k=1}^n a_k b_n \to 0$.
\end{lemma}

\begin{proof}
  $\varepsilon_k = \dfrac{\xi_k - a}{k}, M\varepsilon_k = 0$, $X_n = \sum_{k=1}^n \varepsilon_k$ -- мартингал.

  $MX_n^2 = M \left( \sum_{k=1}^n \varepsilon_k^2 \right) = \sum_{k=1}^n M\varepsilon_k^2 = \sum_{k=1}^n \dfrac{M(\xi_k - a)^2}{k^2} \leqslant C \sum_{k=1}^n \dfrac{1}{k^2} \leqslant C \zeta(2)$,
  где $\zeta(x)$ -- это функция Римана.

  Тогда $\exists X_\infty : P( X_n \to X_\infty ) = 1$;

  \[
    \dfrac{1}{n} \sum_{k=1}^n \xi_k = \dfrac{1}{n} \sum_{k=1}^n (\xi_k - a) + a
    = a + \underbrace{\dfrac{1}{n} \sum_{k=1}^n \dfrac{\xi_k-a}{k} k}_{\to 0 \text{(по лемме Кронекера)}}
  \]
\end{proof}

\begin{theorem}
  Если $ \left\{ X_n \right\}_{n \geqslant 0} $ -- квадратично интегрируемый мартингал и $P( \langle X\rangle_n \to \infty ) = 1$, то $\dfrac{X_n}{\langle X\rangle_n} \toPN 0$.
\end{theorem}
\begin{ex}
  Дана авторегрессионная последовательность:
  $X_{n+1} = \theta X_n + \varepsilon_{n+1}, n \geqslant 0, X_0 = 0$,
  где $\varepsilon_j$ -- последовательность независимых случайных величин с $M\varepsilon_k = 0$.

  \[
    \hat{\theta} = \dfrac{ \sum_{k=0}^{n-1} X_{k+1} X_k }{ \sum_{k=0}^{n-1} X_k^2 } =
    \dfrac{ \sum_{k=0}^{n-1} (\theta X_k + \varepsilon_{k+1}) X_k }{ \sum_{k=0}^{n-1} X_k^2 } =
    \theta + \dfrac{ \sum_{k=0}^{n-1} \varepsilon_{k+1} X_k }{ \sum_{k=0}^{n-1} X_k^2 } =
    \theta + \dfrac{m_n}{\langle m \rangle_n}
  \]

  \[
    \dfrac{m_n}{\langle m \rangle_n} \toPN 0 \Leftrightarrow \hat{\theta} -
    \text{сильно состоятельная}
  \]
\end{ex}
\begin{proof}
  $\mathcal{F}_k = \sigma(\varepsilon_0, \varepsilon_1, \dots, \varepsilon_k)$
  
  Докажем, что $m_n = \sum_{k=0}^{n-1} \varepsilon_{k+1} X_k$ -- действительно является мартингалом:
  \[
    M(m_{n+1} | \mathcal{F}_n) =
    M \left( \sum_{k=0}^n X_k \varepsilon_{k+1} | \mathcal{F}_n \right) =
    M \left( m_n + X_n \varepsilon_{n+1} | \mathcal{F}_n \right) =
    m_n + X_n \cdot \cancelto{0}{M\varepsilon_{n+1}}
  \]

  \[
    \langle m \rangle_n = \sum_{k=0}^{n-1} M \left(  X_k^2 \varepsilon^2_{k+1} | \mathcal{F}_n \right)  = \sum_{k=0}^{n-1} X_k^2 \underbrace{M\varepsilon_{k+1}^2}_{D\varepsilon = 1} = \sum_{k=0}^{n-1} X_k^2
  \]

  т.о. если $\sum X_k^2 \to \infty$, то $\hat{\theta} \toPN 0$;

  % изучить экспериментально, к чему сходиться последовательность $\dfrac{m_n}{ \sqrt{\langle m\rangle_n} }$
\end{proof}
