\section{Ортогональная случайная мера}
Комбинация случайной величины и некоторой меры. Пусть дано вероятностное
пространство $ (\Omega, \mathscr{F}, \mathsf{P}) $, а также $ (E, \mathscr{E})
$, где $ \mathscr{E} $ --- $ \sigma $-алгебра подмножеств. Построим
\emph{ортогональную случайную меру} $ z(\omega, \Delta) $, $ \Delta \in \mathscr
E$, $ z\in\mathbb C $, $ z $ --- случайная величина.

Назовём некоторые \textbf{свойства ортогональной случайной меры}.
\begin{enumerate}
  \item Для любого $ \Delta $ справедливо $
    MZ(\omega,\Delta)\overline{Z(\omega,\Delta)} <\infty $.
  \item Для любого $ \Delta $ справедливо $MZ(\omega,\Delta) = 0$.
  \item Для $\Delta_1 \cap \Delta_2 = \varnothing$ справедливо
  \[
      Z(\omega, \Delta_1 + \Delta_2) = Z(\omega,\Delta_1) + Z(\omega, \Delta_2).
  \]
  \item $ \Delta_i \cap \Delta_j = \varnothing $, $ i\neq j $
  \[
    \Delta = \bigcup_k \Delta_k  M \biggl|Z(\omega, \Delta) - \sum_{k=1}^n
    Z(\omega,\Delta_k) \biggr|^2 \to 0
  \]
  при $ n\to\infty $.
\item \textsc{Ортогональность.} Для $ \Delta_1 \cap \Delta_2 = \varnothing $
  справедливо
\[
  M\big[Z(\omega,\Delta_1) \cdot \overline{Z(\omega,\Delta_2)}\big] = 0.
\]
При этом 
\[
  M Z(\omega, \Delta)\overline{Z(\omega, \Delta)} = m(\Delta) \text{ ---
  \emph{структурная функция меры}}.
\]
\end{enumerate}

\begin{ex}
  Пуассоновская случайная мера. $ \nu(\omega,\Delta) $ --- число событий в
  пуассоновском потоке на $ \Delta $, $ E = [0, \infty) $, $
  \mathscr{E} $ --- борелевские подмножества. Для $ \Delta \in
  \mathscr{E} $ $ \pi(\Delta) < \infty $. При этом  
  \[
    P(\nu(\omega,\Delta)=k) = \frac{(\pi(\Delta))^k}{k!}e^{-\pi(\Delta)}, \quad
    k = 0,1, \ldots.
  \]
  
  Проверим свойства.
  \begin{enumerate}
    \item $ M\nu^2(\omega,\Delta) = \text{существ.} $ \checkmark
    \item $ M\nu(\omega,\Delta) = \pi(\Delta) \neq 0 $. \cross
    \item[3, 4.] автоматически. \checkmark
    \item[5.] $ M\nu(\omega,\Delta_1)\overline{\nu(\omega,\Delta_2)} =
      M\nu(\omega,\Delta_1)\nu(\omega, \Delta_2) = 0 $ (независимые величины).
      \checkmark
  \end{enumerate}
  
   %TODO дописать
\end{ex}
 
\begin{ex}
  Винеровская случайная мера. Пусть $ W_t $ --- стационарный винеровский
  процесс. 
  \[
      W(\omega, \Delta) = W_b - W_a
  \]
 \[
   E = [0, +\infty) \quad \Delta = [a, b]
 \]
 $ \varepsilon $ 
 На $ \varepislon $ продолжаем по теореме Каритоодори. 
 \[
     MW(\omega, \Delta) \overline(\omega,\Delta) = M(W_b - W_a)^2 = m(\Delta) =
     \sigma^2(b-a) = (b-a).
 \]
  
\[
    MW(\omega, \Delta) = M(W_b - W_a) = 0
\]
 
\[
  MW(\omega,\Delta_2) W(\omega, \Delta_2) = 0 (\text{незав., поскольку $
  \Delta_1 \cap \Delta_2 = \varnothing $}).
\]
\end{ex} 

Всё понятие было бы абстрактным, если бы не следующий пример.
\begin{ex}
  $ \xi_n = \sum_{k=1}^N Z_k e^{i\lambda_k n}, $ $ \lambda_i \in (-\pi, \pi) $.
  $ MZ_i = 0 $, $ MZ_iZ_i = \sigma^2 $, $ MZ_iZ_j = 0 $, откуда $ MZ(\omega,\Delta) = 0
  $.
   
 \[
   Z(\omega, \Delta) = \sum_{k=1}^N z_k I(\lambda_k \in\Delta).
 \]
 
\[
  MZ(\omega, \Delta_2)\overline{Z(\omega,\Delta_2)} = M\left(\sum_{k=1}^N
  Z_kI(\lambda_k \in\Delta_2)\right) \left( \sum_{j=1}^n Z_j I(\lambda_j \in
  \Delta_2) \right)  = 0.
\]
 
\begin{multline*}
  MZ(\omega, \Delta)\overline{Z(\omega,\Delta)} = M\sum_{k=1}^NZ_k
  I(\lambda_k\in\Delta)\sum_{j=1}^N\overline{Z}_j I(\lambda_j \in \Delta) =\\=
  \sum_k \sum_j I(\lambda_k\in\Delta) I(\lambda_j\in\Delta) MZ_k\overline{Z_j} =
  \sum_{k=1}^n \sigma^2_k I(\lambda_k \in \Delta) = G(\Delta)
\end{multline*}
 
 
\[
    G(\Delta) = \int dG(\lambda),
\]
где $ G(\lambda) = \sum_{k=1}^n \sigma_k^2 I(\lambda - \lambda_k) $.

 
\begin{align*}
  K_\xi(n) &= \int\limits_{-\pi}^{\pi}e^{i\lambda n}\,dG(\lambda), \\
  \xi_n &= \int\limits_{-\pi}^{\pi}e^{i\lambda n} Z(d\lambda)= e^{i\lambda_1 n}
  Z_1 + e^{i\lambda_N} %TODO: дописать
\end{align*}
\end{ex}
%TODO: рис ко всем примерам


\subsection{Интеграл от неслучайной функции по ОСМ}
$ \mathscr H $ --- гильбертово пространство комплексные с. в. $ M\xi\bar{\xi} < \infty $ 
\[
  (\xi, \eta) = M\dot{\xi} \dot{\bar{\eta}} = M\xi\bar{\eta} - M\xi\cdot
  M\bar{\eta},
\]
 
$\mathscr{L}_2$ --- пространство комплексных (неслуч) функций на $ E $. 
\[
  \int_E f(\lambda)\overline{f(\lambda)} m(d\lambda) < \infty,
\]
 
\[
  (f, g) = \int_E f(\lambda)\overline{g(\lambda)}md(\lambda)
\]
 
\[
    \int_E f(\lambda)Z(d\lambda)
\]
1. $ f(\lambda) = \int_{k=1}^n f_k I(\lambda \in \Delta_k) $,
$ \Delta_k \cap \Delta_j = \varnothing $, $ k\neq j $.
 
\[
  \int_E f(\lambda)Z(d\lambda) = \sum_{k=1}^n f_k Z(\omega,\Delta_k)
\]

Свойства.
\begin{enumerate}
  \item $ MI(f) = 0 $,  
  \[
    MI(f) = \sum_{k=1}^n f_k MZ(\omega, \Delta_k) = 0.
  \]
\item $ I(\alpha f +\beta g) = \alpha I(f) + \betaa I(g) $
\item
  \begin{multline*}
    \cov(I(f), I(g)) = MI(f) \overline{I(g)} = \\ =
    M \left[ \sum_{k=1}^n f_k
      Z(\omega, \Delta_k) \overline{\sum_{j=1}^n g_j Z\left(\omega,\Delta_j\right)}
    \right] = \sum_{k=1}^n f_k \bar{g}_k MZ(\omega, \Delta_k)
    \overline{Z(\omega,\Delta_k)},
  \end{multline*}
\[
  g = \sum_{k=1}^n g_k I(\lambda \in \Delta_k).
\]
 
\[
  \int_E f(\lambda)\overline{g(\lambda)} m(d\lambda) = \sum_{k=1}^n f_k\bar{g}_k
  m(d\lambda).
\]
\end{enumerate}

\paragraph{2.} $ f $ --- произвольная неслучайная функция. Пусть $ f_n $ ---
последовательность простых функций $ \| f - f_n \|_{\mathscr{L}_2} \to 0 $. 
\begin{multline*}
  \|I(f) - I(g)\|_{\mathscr{H}} = M \left[\sum_{k=1}^N(f_k - g_k)Z(\omega,
  \Delta_k)\cdot \sum_{j=1}^N(f_j - g_j)Z(\omega,\Delta_j)\right] = \\ =
  \sum_{k=1}^N
  |f_k - g_k|^2m(\Delta_k) = \int_E |f-g|^2 m(d\lambda) =
  \|f-g\|_{\mathscr{L}_2}.
\end{multline*}
Отсюда $ \|I(f_n) - I(f_m)\|_\mathscr{H} \to 0 $, откуда %TODO: дописать

\begin{theorem}
  Если $ \{\xi_n\} $ --- ССП, то существует ортогональная случайная мера $
  Z_\xi(\omega, \Delta) $, такая что $ \xi_n =
  \int\limits_{-\pi}^{\pi}e^{i\lambda n} \,dG_\xi(\lambda) $, где $ \Delta \in
  \mathscr{E} $, $ E = [-\pi, \pi] $, $ \mathscr{E} $ --- борелевские множества,
 \[
     G_\xi(\Delta) = \int_\Delta d G_\xi(\lambda).
 \]
 --- структурная функция. То есть спектральная функция автоков функции является
 структурной функцией меры $ Z_\xi $.
\end{theorem}

\begin{theorem}
  Если $ \mathscr{H} $ --- пространство из линейной комбинации сечения ССП $
  \{\xi_n\} $ и их ср.кв.пределов  
  \[
    \eta \in \mathscr{H} \implies \exist \Psi(\lambda)\colon \eta =
    \int\limits_{-\pi}^{\pi}\Psi(\lambda)Z_\xi(d\lambda). 
  \]
\end{theorem}
\begin{remark*}
   
 \[
   \int\limits_{-\pi}^{\pi}(e^{i\lambda n} + e^{2i\lambda n})Z_\xi(d\lambda) =
   \xi_{-n} + \xi_n.
 \]
\end{remark*} 

\begin{definition}
  $ \{\Dzeta_n\} $ называется \emph{стационарным линейным преобразованием} $
  \{\xi_n\} $ в спектральной области, если $ \Dzeta_n =
  \int\limits_{-\pi}^{\pi}e^{i\lambda n}\Phi(\lambda)Z_\xi(d\lambda) $.

  $ \Phi(\lambda) $ --- частотная характеристика линейного преобразования, 
  \[
      \int\limits_{-\pi}^{\pi}|\Phi(\lambda)|^2 dG(\lambda) < \infty.
  \]
\end{definition}

\begin{theorem}
  %TODO: дописать

   
 \[
   \Dzeta_n = \int\limits_{-\pi}^{\pi}e^{i\lambda n}\Phi(\lambda)
   Z_\xi(d\lambda),
 \]
 
\end{theorem}
\begin{proof}
  \begin{multline*}
    K_\Dzeta(n+m, m) = \cov(\Dzeta_{n+m}, \Dzeta_m) = \cov \left(
      \int\limits_{-\pi}^{\pi}e^{i\lambda(n+m)}\Phi(\lambda)Z_\xi(d\lambda),
    \int\limits_{-\pi}^{\pi}e^{i\lambda m}\Phi(\lambda) Z_\xi(d\lambda)\right) =
    \\ =
    \int\limits_{-\pi}^{\pi}e^{i\lambda(n+m)} \Phi(\lambda)e^{i\lambda m}
      \Phi(\lambda)\,dG_\xi(\lambda) = \int\limits_{-\pi}^{\pi}e^{i\lambda n}
      |\Phi(\lambda)|^2\,dG_\xi(\lambda)
  \end{multline*}
  не зависит от $ m $. 
  \[
    \Dzeta_n = \int\limits_{-\pi}^{\pi}e^{i\lambda n} Z_\dzeta(d\lambda) =
    \int\limits_{-\pi}^{\pi}e^{i\lambda n}\Phi(\lambda)Z_\xi(d\lambda).
  \]
  
\[
  K_\dzeta(n) = \int\limits_{-\pi}^{\pi}e^{i\lambda n}
  |\Phi(\lambda)|^2\,dG_\xi(\lambda) = \int\limits_{-\pi}^{\pi}e^{i\lambda
  n}\,dG_\xi(\lambda).
\]
\end{proof}

\begin{example}
  Пусть $ \{\varepsilon_n\} $ --- стационарный белый шум. Тогда  
  \[
    \Phi(\lambda) = \sum_{m=-\infty}^\infty e^{-i\lambda m} h(m),
  \]
   
 \[
   h(m) = \frac{1}{2\pi}\int\limits_{-\pi}^{\pi}e^{i\lambda
   m}\Phi(\lambda)\,d\lambda
 \]
  
\[
  \Dzeta_n = \int\limits_{-\pi}^{\pi}e^{i\lambda n}\Phi(\lambda)Z_\xi(d\lambda)
  = \sum_{m=-\infty}^\infty h_m
  \int\limits_{-\pi}^{\pi}e^{i\lambda(n-m)}Z_\xi(d\lambda) =
  \sum_{m=-\infty}^{\infty}h_m\varepsilon_{n-m}
\]
(скользящее среднее). 
 
\[
  g_\varepsilon(\lambda) = \frac{1}{2\pi} \to g_\dzeta(\lambda) =
  |\Phi(\lambda)|^2 \frac{1}{2\pi}
\]
\end{example}

\begin{example}
  $\ARMA(p, q)$.  $ \xi_n - \sum_{k=1}^p b_k \xi_{n-k} = \sum_{k=1}^q a_k
  \varepsilon_{n-k} $, $ n \in \mathbb Z $.

   
 \[
   \xi_n - \sum_{k=1}^p \xi_{n-k} b_k = \int\limits_{-\pi}^{\pi}lr(e^{i\lambda
   n} - \sum_{k=1}^p b_k e^{i\lambda(n-k)})Z_\xi(d\lambda)
 \]
 --- левая часть.
\end{example}

Прогнозирование достаточно лёгкое. У фильтрации сложное практическое применение,
и она решается другим способом.


