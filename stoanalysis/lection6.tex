\paragraph{Resum\'e}

Если ССП $\xi_n = \sum_{k=-\infty}^\infty a_k \varepsilon_{n-k}$,
тогда $K_\xi(n) = \sum_{k=-\infty}^\infty a_{k+n} \bar{a}_k$

% TODO дописать начало, Андрей сфоткал

$\exists g_\xi(\lambda) \geqslant 0 \Leftrightarrow \xi_n = \sum_{k=-\infty}^\infty$ -- 
существование спектральной плотности эквивалентно представимости этой последовательности в 
виде двустороннего скользящего среднего.

\[
  \xi_n = \underbrace{\sum_{k=0}^{+\infty} a_k \varepsilon_{n-k}}_\text{регулярная часть (физически реализуемый фильтр)} 
  + 
  \underbrace{\sum_{k=-1}^{-\infty} a_k \varepsilon_{n-k}}_\text{сингулярная}
\]
Есть ещё такая теорема --
\emph{теорема Вольда}, которая говорит о разложимости в виде: $\xi_n = \xi_n^P + \xi_n^S$.

Сингулярная часть физически не реализуемая.

\begin{theorem}[Колмогорова]
  ССП $\xi_n$ представима в виде одностороннего скользящего среднего тогда и только тогда, когда
  его спектральная плотность принадлежит классу таких, у которых интеграл от логарифма больше 
  минус бесконечности:
  \[
    \xi_n = \sum_{k=0}^{+\infty} a_k \varepsilon_{n-k} \Leftrightarrow
    \int\limits_{-\pi}^\pi \ln g_\xi(\lambda) \, d\lambda > -\infty
  \]

  В частности, $g_\xi(\lambda) > 0$ почти всюду.

  Здесь можно еще сказать про принадлежность этой функции классу Харви.
\end{theorem}

\begin{remark}
  Почти периодическая последовательность $\xi_n = \sum_{k=1}^N \sigma_k e^{i\lambda_k n}$
  не является регулярной, также как и не имеет спектральной плотности.
\end{remark}

\begin{remark}
  Рассмотрим последовательность с автоковариационной функцией вида:
  \[
    K_\xi(n) = \begin{cases}
      1, n = 0, \\
      \dfrac{\sin na}{na}, n\neq 0
    \end{cases}
  \]
  тогда $g_\xi(\lambda) = \dfrac{1}{2\pi} I(|\lambda| < a) \geqslant 0$.
  $\xi_n$ может быть представлена в виде двустороннего скользящего среднего, но
  не является регулярной.
\end{remark}

\section{Прогнозирование регулярных ССП}

Введём обозначение:
$\xi^m = \left( \xi_{m}, \xi_{m-1}, \dots \right) $ -- вектор, содержащий информацию о прошлом, а
$\varepsilon^m = \left( \varepsilon_m, \varepsilon_{m-1}, \dots \right) $.

Пусть $\mathcal{H}(\xi^m)$ -- гильбертово пространство линейных комбинаций $\xi^m$,
тогда: $\mathcal{H}(\xi^m) = \mathcal{H}(\varepsilon^m).$

В силу стационарности, задачу прогнозирования можно поставить проще: $\hat{\xi}_{n+m}$ выразить
через $\xi_m, \xi_{m-1}, \dots$ и эта задача эквивалентна тому, что:
$\hat{\xi}_n$ через $\xi_0, \xi_{-1}, \dots$.

\begin{multline*}
  \hat{\xi}_n = M(\xi_n | \xi_0, \xi_{-1}, \dots) =
  M \left( \left. \sum_{k=0}^{+\infty} a_k \varepsilon_{n-k} \right| \varepsilon_0, \varepsilon_1, \dots \right) =
    M \left( \left. \sum_{k=0}^{+\infty} a_k \varepsilon_{n-k} + \sum_{k=0}^{n-1} a_k \varepsilon_{n-k} \right| \varepsilon_0, \varepsilon_{n-1}, \dots \right) = \\
      = \sum_{k=n}^{+\infty} a_k\varepsilon_{n-k} + \sum_{k=0}^{n-1} a_k M\varepsilon_{n-k}
      = \sum_{k=n}^{+\infty} a_k \varepsilon_{n-k}
\end{multline*}
-- предсказали через $\varepsilon$, но надо через $\xi$.

Посчитаем еще ошибку:
\[
  \Delta = M(\hat{\xi}_n - \xi_n)^2 = M |\sum a_k \varepsilon_{n-k}|^2 = \sum_{k=0}^{n-1} |a_k|^2
\]

\begin{theorem}
  Если регулярный ССП $\xi_n = \sum_{k=0}^{+\infty} a_k \varepsilon_{n-k}$ с
  спектральной плотностью $g_\xi(\lambda) = \dfrac{1}{2\pi} |\varphi(e^{i\lambda})|^2,$
  причём $\varphi(z) = \sum_{k=0}^{+\infty} a_k z^k$ имеет радиус сходимости больше 1,
  не имеет нулей в $|z| \leqslant 1$, тогда:
  \[
    \hat{\xi}_n = \int\limits_{-\pi}^\pi \hat{\varphi}_n(\lambda) Z_\xi(d\lambda),
  \]
  где $\hat{\varphi}_n(\lambda) = e^{i\lambda n} \dfrac{\varphi_n(e^{-i\lambda)}}{\varphi(e^{-i\lambda})}$, $\varphi_n(z) = \sum_{k=n}^{+\infty} a_k z^k$.
\end{theorem}
\begin{proof}
  \[
    \hat{\xi}_n = \sum_{k=n}^\infty a_k \varepsilon_{n-k} =
    \sum_{k=n}^\infty a_k \int\limits_{-\pi}^\pi e^{-i\lambda (n-k)} \, Z_\varepsilon(d\lambda)
    =
    \int\limits_{-\pi}^\pi e^{i\lambda n} \left( \sum_{k=n}^\infty a_k e^{-i\lambda k} \right) \, Z_\varepsilon(d\lambda)
  \]
  Причём т.к.
  \[
    \xi_n = \sum_{k=0}^\infty a_k \varepsilon_{n-k} = \int\limits_{-\pi}^\pi e^{i\lambda n} \underbrace{\sum_{k=0}^\infty a_k e^{-i\lambda k} \, Z_\varepsilon(d\lambda)}_{Z_\xi(d\lambda)}
  \]
  тогда можно выразить:
  \[
    \hat{\xi}_n = \int\limits_{-\pi}^\pi e^{i\lambda n} \dfrac{\sum_{k=n}^\infty a_k e^{-i\lambda k}}{\sum_{k=0}^\infty a_k e^{-i\lambda k}} Z_\xi(d\lambda)
  \]
\end{proof}

\begin{corollary}
  Если $\hat{\varphi}_n (\lambda) = C_0 + C_1 e^{-i\lambda} + C_2 e^{-2i\lambda} + \dots$, то
  $\hat{\xi}_n = c_0 \xi_0 + c_1 \xi_{-1} + \dots$.
\end{corollary}

\begin{ex}
  $g_\xi(\lambda) = \dfrac{1}{2\pi} (5 + 4 \cos \lambda) = \dfrac{1}{2\pi} (5 + 2e^{i\lambda} + 2e^{-i\lambda})$
  Необходимо проверить условия теоремы, для этого надо найти <<корень>> этой функции, т.е.
  представить $g_\xi(\lambda) = \dfrac{1}{2\pi} |\varphi(e^{-i\lambda})|^2$.

  Найдем представление в виде: $(a+be^{-i\lambda})(a+be^{i\lambda}) = a^2 + b^2 + 2ab \cos\lambda$,
  тогда сравнивая с выражением для $g_\xi$ получаем следующие возможные пары решений $(a; b)$:
  $(2, 1), (1, 2), (-1, -2), (-2, -1)$.
  У полученной функции $\varphi(z) = a+bz$ не должно быть нулей в круге $|z| \leqslant 1$, что
  есть то же самое, что $\left|- \dfrac{a}{b}\right| > 1$, тогда получается, что подходят только
  решения $(2, 1)$ и $(-2, -1)$:
  \[
    \varphi(z) = 2 + z = \sum_{k=0}^\infty a_k z^k \rightarrow 
    \varphi(e^{-i\lambda}) = 2 + e^{-i\lambda}.
  \]
  \[
    \varphi_1(z) = z, \quad
    \varphi_n (z) = 0, n\geqslant 0.
  \]

  \begin{multline*}
    \hat{\xi}_1 = \int\limits_{-\pi}^\pi e^{i\lambda} \dfrac{\varphi_1(e^{-i\lambda}}{\varphi(e^{-i\lambda}} \, Z_\xi(d\lambda) = \int\limits_{-\pi}^\pi \dfrac{Z_\xi(d\lambda)}{2+e^{-i\lambda}} =
    \dfrac{1}{2} \int\limits_{-\pi}^\pi \dfrac{Z_\xi(d\lambda)}{1 + e{-i\lambda} / 2} =
    \dfrac{1}{2} \int\limits_{-\pi}^\pi \sum_{k=0}^\infty \left( - \dfrac{1}{2} e^{-i\lambda} \right)^k Z_\xi(d\lambda) = \\
    = \sum_{k=0}^\infty \dfrac{(-1)^k}{2^{k+1}} \int\limits_{-\pi}^\pi e^{-i\lambda k} Z_\xi(d\lambda) =
    \sum_{k=0}^\infty \dfrac{(-1)^k}{2^{k+1}} \xi_{-k}
  \end{multline*}

  Но для $n > 1$: $\hat{\xi}_n = 0$. Почему так происходит?
  Так как
  \[
    g_\xi(\lambda) = \dfrac{1}{2\pi} \sum_{n=-\infty}^{+\infty} e^{- i\lambda n} K_\xi(n) = \dfrac{1}{2\pi} (5+2e^{-i\lambda} + 2e^{i\lambda}),
  \]
  т.е. $K_\xi(0) = 5, K_\xi(\pm 1) = 2,$ а при $|n| > 1: K_\xi(n) = 0$.
\end{ex}

\begin{ex}
  $K_\xi(n) = a^{|n|}, |a| < 1$. Найти прогноз $\hat{\xi}_n = M(\xi_n | \xi_0, \xi_{-1}, \dots).$

  \begin{multline*}
    g_\xi(\lambda) = \dfrac{1}{2\pi} \sum_{n=-\infty}^\infty e^{- i\lambda n} a^{|n|} =
    \dfrac{1}{2\pi} \sum_{n=0}^\infty e^{- i\lambda n} a^n + \dfrac{1}{2\pi}\sum_{n=-1}^{-\infty} e^{-i\lambda n} a^{-n} =
    \dfrac{1}{2\pi} \left( \dfrac{1}{1 - ae^{-i\lambda}} + \dfrac{ae^{i\lambda}}{1 - ae^{i\lambda}} \right) = \\
    = \dfrac{1}{2\pi} \dfrac{1 - a^2}{(1-ae^{-i\lambda})(1-ae^{i\lambda})} = 
    \dfrac{1}{2\pi} \left| \dfrac{\sqrt{1-a^2}}{1-ae^{-i\lambda}} \right|^2
  \end{multline*}
  Тогда $\varphi$ -- не имеет нулей
  в $|z| \leqslant 1$ ($1/a > 1$).
  \[
    \varphi(e^{-i\lambda}) = \dfrac{\sqrt{1-a^2}}{1 - ae^{-i\lambda}} = 
    \sqrt{1-a^2} = \sum_{k=0}^\infty a^k e^{-i\lambda k}
  \]
  \[
    \hat{\varphi}_n(\lambda) = e^{i\lambda n} \dfrac{\varphi_n}{\varphi} = e^{i\lambda n} \dfrac{\sum_{k=n}^\infty a^k e^{-i\lambda k}}{\sum_{k=0}^\infty a^k e^{-i\lambda k}} = e^{i\lambda n} a^n e^{-i\lambda n} = a^n
  \]
  Тогда
  \[
    \hat{\xi}_n = \int\limits_{-\pi}^\pi \hat{\varphi}_n(\lambda) Z_\xi(d\lambda) = 
    \int\limits_{-\pi}^\pi \underbrace{1}_{e^{i\lambda 0}} a^n Z_\xi(d\lambda) = a^n \xi_0.
  \]
\end{ex}

\section{Задача фильтрации}

Имеется двумерная последовательность $(\theta_n, \xi_n)$, причём $\theta_n$ -- наблюдаемая
компонента, а $\xi_n$ -- ненаблюдаемая. Задача фильтрации формулируется так:
\[
  \hat{\xi}_n = M(\xi_n | \theta_n, \theta_{n-1}, \dots)
\]

\[
  \theta_n = \int\limits_{-\pi}^\pi e^{i\lambda n} Z_\theta(d\lambda), \quad
  \xi_n = \int\limits_{-\pi}^\pi e^{i\lambda n} Z_\xi(d\lambda)
\]

\begin{definition}
  \emph{Взаимная спектральная функция}:
  $G_{\xi \theta} (\Delta) = M Z_\xi(\Delta) \overline{Z_\theta(d\lambda)}.$

  $R_{\xi\theta}(n) = \cov (\xi_{n+k}, \theta_k) = \int\limits_{-\pi}^\pi e^{i\lambda n} G_{\xi\theta}(d\lambda)$
\end{definition}

По сути, задача фильтрации состоит в нахождении такой функции $\hat{\varphi}_n$, что:
$\hat{\xi}_n = \int\limits_{-\pi}^\pi \hat{\varphi}_n(\lambda) Z_\theta(d\lambda)$,
причём $\hat{\varphi}_n(\lambda) = \mathcal{H}(\theta^n) = \mathcal{H}( \theta_n, \theta_{n-1}, \dots )$,
причём $\hat{\xi}_n - \xi_n \perp \mathcal{H}(\theta^n)$.

\begin{theorem}
  Если $G_{\xi\theta} (\lambda)$ имеет спектральную плотность $g_{\xi\theta} (\lambda)$, тогда 
  $\hat{\varphi}_n(\lambda) = e^{i\lambda n} \dfrac{g_{\xi\theta}(\lambda)}{g_\theta(\lambda)}$.
\end{theorem}
\begin{proof}
  Ортогональность $\hat{\xi}_n - \xi_n \perp \mathcal{H}(\theta^n)$ означает, что 
  $0 = M(\hat{\xi}_n - \xi_n) \bar{\theta}_m \forall m \leqslant n$ тогда:
  \[
    M\hat{\xi}_n \bar{\theta}_m = \int\limits_{-\pi}^\pi e^{-i\lambda m} \hat{\varphi}_n (\lambda) \, \underbrace{g_\theta(\lambda) d\lambda}_{dG_\theta(\lambda)}
  \]
  \[
    M\xi_n \hat{\theta}_m = \int\limits_{-\pi}^\pi e^{i\lambda(n-m)} g_{\xi\theta} (\lambda) d\lambda
  \]
  Ортогональность тогда будет означать, что
  \[
    \int\limits_{-\pi}^\pi e^{i\lambda (n-m)} \left(\underbrace{
      e^{-i\lambda m}\hat{\varphi}_n (\lambda) g_\theta(\lambda) -
      g_{\xi\theta}(\lambda)
  }_{=0}\right) d\lambda = 0
  \]
\end{proof}
