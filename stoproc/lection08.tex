\section{Винеровский процесс}

\paragraph{Броуновское движение}

Пусть $\bar{\xi}(t) = \left( \xi_1(t), \xi_2(t), \dots, \xi_n(t) \right) $
-- координаты частицы.
Предположения:
\begin{enumerate}
  \item Движения в ортогональных направлениях независимы;
  \item $\xi_k(t) \sim N(0, \sqrt{D\xi_k(t)})$;
  \item $\bar{\xi}(t_1)$ и $\bar{\xi}(t_2) = \bar{\xi}(t_1)$ -- независимы;
  \item Закон распределения $\bar{\xi}(t+h) - \bar{\xi}(t)$ не должен зависеть от $t$.
    Из последнего и первого получается, что $\xi_k (t+h) = \xi_k(t) + \xi_k(t+h) - \xi_k(t)$,
    тогда $D\xi_k(t+h) = D\xi_k(t) + D\xi_k(h)$, но этому свойству удовлетворяет только
    линейная функция: $D\xi_k(t) = \sigma^2 t$.
\end{enumerate}
-- эти тредования сформулировал не Броун, а Винер.

\paragraph{Эквивалентные определения}
\begin{definition}\label{lec_8:viner}
  $\bar{W}(t) = \left( W_1(t), W_2(t), \dots \right) $ называется винеровским процессом,
  если:
  \begin{enumerate}
    \item $\bar{W}(0) = 0$;
    \item Для $t_1 < t_2 < \dots < t_N$: $\bar{W}(t_1)$, $\bar{W}(t_2) - \bar{W}(t_1)$, $\dots$,
      $\bar{W}(t_N) - \bar{W}(t_{N-1})$ -- независимы.
    \item $\bar{W}(t) - \bar{W}(s) \sim N(0, \sigma^2 |t-s| E_n)$.
  \end{enumerate}
\end{definition}

Пусть $W_t$ -- скалярный винеровский процесс. Из определения \ref{lec_8:viner}
получаем, что
\begin{enumerate}
  \item $W_0 = 0$;
  \item $W_t$, $W_{t_2} - W_{t_1}$, $\dots$, $W_{t_N} - W_{t_{N-1}}$ -- независимы;
  \item $W_t - W_s \sim N(0, \sigma^2 \sqrt{|t-s|})$
\end{enumerate}

\begin{utv}
  Вместо этого определения можно взять за определение следующие свойства:
  \begin{enumerate}
    \item $W_0 = 0$;
    \item $k_W(t, s) = \sigma^2 \cdot \min(t, s)$;
    \item $W_t - W_s \sim N(0, \sigma^2 \sqrt{|t-s|})$.
  \end{enumerate}
\end{utv}
\begin{proof} Положим $ t > s $.

  Из первого правое:
  $k_W(t,s) = \cov(W_t, W_s) = \cov(W_s + W_t - W_s, W_s) = \cov(W_s, W_s) + \cancelto{0}{\cov(W_t - W_s, W_s)}
  = DW_s = \sigma^2 s$;

  В обратную сторону. Для любых $ t > s $ имеем
  \[
    \sigma^2 s = k_W(t, s) = \cov(W_t - W_s, W_s) + DW_s = \cov(W_t
    - W_s, W_s) + \sigma^2 s,
  \]
  откуда $ W_s $ и $ W_s - W_t $ некоррелированы. Учитывая ещё, что обе СВ имеют
  нормальное распределение, заключаем, что они независимы. Если есть ещё $ p > t
  > s$, то аналогичными рассуждениями доказываем то же самое.
  % TODO дописать % ещё немного можно
\end{proof}

\paragraph{Конечномерные распределения винеровского процесса}
Для $t_1 < t_2 < \dots < t_N$ рассмотрим случайный вектор $\bar{\xi}^T = (\xi_{t_1}, \xi_{t_2}, \dots, \xi_{t_N})$. Обозначим
\[
  \eta = A\bar{\xi} = \begin{pmatrix}
    1 & 0 & 0 & 0 \\
    -1 & 1 & 0 & 0 \\
    0 & -1 & 1 & 0 \\
    0 & 0 & -1 & 1
    \end{pmatrix} \bar{\xi} = \begin{pmatrix}
    \xi_{t_1} \\
    \xi_{t_2} - \xi_{t_1} \\
    \xi_{t_3} - \xi_{t_2} \\
    \dots
  \end{pmatrix}
\]

\[
  p_{\bar{\xi}}(\bar{x}) = p_{\eta} (A\bar{x}) = \prod_{k=1}^{N} p_{\xi_{t_k} - \xi_{t_{k-1}}} (x_k - x_{k-1})
  = \prod_{k=1}^{N} \dfrac{1}{\sqrt{2\pi} \sigma \sqrt{t_k - t_{k-1}}} e^{- \dfrac{(x_k - x_{k-1})^2}{2\sigma^2 (t_k - t_{k-1})}}
\]

\paragraph{Свойства винеровского процесса}
\begin{enumerate}
  \item $\forall t > 0 : P(W_t > 0) = P(W_t < 0) = \dfrac{1}{2}$;
  \item $P(W_t > x) = \int\limits_0^{+\infty} p_{W_t}(y) \, dy = 1 - \Phi \left( \dfrac{x}{\sigma\sqrt{t}} \right) $;
  \item В частности, $P( | W_t | > 3\sigma\sqrt{t} ) = 0.003$;
  \item Если обозначить $\tau_x = \inf \left\{ t, W_t \geqslant x \right\} $,
    тогда $P(\tau_x < s) = 2 \left( 1 - \Phi \left( \dfrac{x}{\sigma \sqrt{s}} \right)  \right) $.
    \begin{proof}
      С одной стороны, так как в винеровском процессе можно переносить начало координат,
      то если он уже побывал в точке $x$, тогда
      \[
        P(W_t \geqslant x | \tau_x < t) = P(W_t < x | t_x < t) = \dfrac{1}{2}
      \]
      .
      С другой стороны, 
      \[
        \dfrac{1}{2} = P(W_t \geqlant x | t_x < t) = \dfrac{P(W_t \geqslant x, \tau_x < t)}{P(\tau_x < t)}
      \]
      отсюда и следует утверждение этого свойства.
    \end{proof}
    Тогда можно найти плотность распределения $\tau_x$:
    \[
      p_{\tau_x} ( t) = 2 \left( \dfrac{e^{- \dfrac{x^2}{2\sigma^2 t}}}{\sqrt{2\pi}} \right) \dfrac{x}{\sigma} \dfrac{1}{2 t \sqrt{t}}
      = \dfrac{x}{\sigma t \sqrt{t} \sqrt{2\pi}} e^{-\dfrac{x^2}{2\sigma^2 t}}
    \]
    Причём, $M\tau_x = \infty$ (так как соответствующий интеграл расходится),
    то есть винеровский процесс достигнет любой отметки $x$, но неизвестно когда.
\end{enumerate}


\begin{definition}
  Случайные процессы $\xi_t$ и $\eta_t$ называются \emph{стохастически эквивалентными}, если
  $\forall t : P(\xi_t = \eta_t) = 1$.
\end{definition}

% дописать пример
\begin{ex}
  $t \in [0, 1] = T$, $\Omega = [0, 1]$, $\xi_t (\omega) = \begin{cases}
    0, &t \neq \omega \\
    \tau(\omega), &t = \omega
  \end{cases}$
  $M|\xi_t - 0|^2 = 0$
\end{ex}

\begin{definition}
  Случайные процессы $\xi_t$ и $\eta_t$ \emph{неразличимы}, если 
  $P(\forall t \xi_t = \eta_t) = 1$.
\end{definition}

Обозначим случайный процесс 
$\xi_t \in \mathcal{L}_2$ (квадратично-интегрируемые: $M\xi_t^2 < \infty$).


\begin{definition}
  Случайная величина $\eta$ называется средне-квадратичным пределом $\xi_t$ при $t\to t$,
  записывается как $\eta = \underset{t\to t_0}{l.i.m.} \bar{\xi}_t$,
  если $M|\bar{\xi}_t -\bar{\eta}|^2 = M \sum_k |\xi_k(t) - \eta_k| \to 0$.
\end{definition}

\begin{ex}
  $a = t_0 < t_1 < t_2 < \dots < t_N = b$,
  \[
    \underset{\max_k |\Delta t_k| \to 0}{l.i.m.} \sum_{k=0}^{N-1} |W_{t_{k+1}} - W_{t_k}|^2
    = \sigma^2 (b-a)
  \]
  \begin{proof}
    \[
      M \sum_{k=0}^{N_1} (W_{t_{k+1} - W_{t_k}})^2 = \sum_{k=0}^{N-1} M(W_{t_{k+1}} - W_{t_k})^2
      = \sum_{k=0}^{N-1} \sigma^2 (t_{k+1} - t_{k}) = \sigma^2 (b-a).
    \]
    Тогда выражение $M \left|\sum_{k=0}^{N-1} (W_{t_{k+1}} - W_{t_k})^2 - \sigma^2 (b-a)\right|^2$:
    \begin{multline*}
      M \left|\sum_{k=0}^{N-1} (W_{t_{k+1}} - W_{t_k})^2 - \sigma^2 (b-a)\right|^2 
      = D \left( \sum_{k=0}^{N-1} (W_{t_{k+1}} - W_{t_k})^2 \right) = \\
      = \sum_{k=0}^{N-1} D \left( W_{t_{k+1}} - W_{t_k} \right)^2
      = \sum_{k=0}^{N-1} \left[
        M(W_{t_{k+1}} - W_{t_k})^4 - \left( M(W_{t_{k+1}} - W_{t_k})^2 \right) \right] = \\
      = \sum_{k=0}^{N-1} \left[ 3\sigma^4 (t_{k+1} - t_k)^2 - \sigma^4 (t_{k+1} - t_k)^2 \right]
      \leqslant 2 \max |\Delta t_k| \cdot \sum_{k=0}^{N-1} (t_{k+1} - t_k) \to 0
    \end{multline*}
  \end{proof}
\end{ex}

\begin{remark}
  Если $\bar{\eta} = \underset{t \to t_0}{l.i.m.} \bar{\xi}_t$, то $\bar{eta}$
  единственнен с точностью до стохастической эквивалентности.
\end{remark}

\begin{remark}
  Если $\bar{\eta} = \underset{t \to t_0}{l.i.m.} \bar{\xi}_t \Leftrightarrow \eta_k = \underset{t \to t_0}{l.i.m.} \xi_k(t)$. Именно поэтому далее будем рассматривать только одномерные процессы.
\end{remark}

\begin{theorem}
  Верны следующие утверждения:

  \begin{enumerate}
    \item $\exists \underset{t \to t_0}{l.i.m.} \xi_t = \eta \Leftrightarrow
      \begin{cases}
        \exists \lim_{t\to t_0, \tau \to t_0} k_{\xi} (t, \tau); \\
        \exists \lim_{t\to t_0} M \xi_t;
      \end{cases}$;
  \item $\exists \underset{t \to t_0}{l.i.m.} \xi_t = \eta \Leftrightarrow
    \exists \lim_{t\to t_0, \tau \to t_0} M(\xi_t \xi_\tau) = A$.
  \end{enumerate}
\end{theorem}

Доказывается эта теорема с помощью \emph{стохастического критерия Коши}.

\begin{theorem}[Стохастический критерий Коши]
  $\exists \underset{t \to t_0}{l.i.m.} \xi_t = \eta \Leftrightarrow
  \exists \underset{t \to t_0, \tau \to t_0}{l.i.m.} |\xi_t - \xi_\tau| = 0$
  ($\underset{t \to t_0, \tau \to t_0}{l.i.m.} |\xi_t - \xi_\tau| = 0
  \Leftrightarrow \lim_{t\to t_0, \tau\to t_0} M(\xi_t - \xi_\tau)^2 = 0$)
\end{theorem}

\subsecion{Непрерывность случайного процесса}

\begin{definition}
  Случайный процесс $\xi_t$ называется непрерывным в среднеквадратическом смысле
  в точке $t_0$, если:
  \[
    \underset{t \to t_0, \tau \to t_0}{l.i.m.} \xi_t = \xi_{t_0}.
  \]
\end{definition}

\begin{theorem}
  Случайный процесс $\xi_t$ непрерывен в точке $t_0$ тогда и только тогда, когда
  \[
    \begin{cases}
      \exists \lim_{t \to t_0} M\xi_t = M\xi_{t_0}, \\
      \exists \lim_{t \to t_0, s\to t_0} k_\xi(t, s) = D\xi_{t_0}.
    \end{cases}
  \]
\end{theorem}

\begin{ex}
  \begin{enumerate}
    \item Рассмотрим $W_t$, тогда $MW_t = 0$, $k_W(t, s) = \sigma^2 \min(t, s)$,
      поэтому $W_t$ непрерывен в средне квадратическом;
    \item $\xi_t$ -- пуассоновский процесс с интенсивностью $\lambda$. Все траектории у 
      этого процесса разрывны, но если вспомнить $M\xi_t = \lambda t, k_\xi(t, s) = \lambda \cdot \min(t, s)$, то есть $\xi_t$ непрерывен в средне квадратичном.
      Этот факт иллюстрируется тем, что в любой точке $t_0$ то, что
      данная траектория имеет скачок в $t_0$ имеет вероятность $0$.
  \end{enumerate}
\end{ex}
