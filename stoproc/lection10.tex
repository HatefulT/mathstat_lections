\section{Лекция 10 -- 2024-05-03 -- Интегрирование в средне квадратическом}

$T = [a, b]$, на $[a, b]^2$ задана функция $\varphi(t, s)$.

Для точек $a = \tau_0 \leqslant \tau_0^* \leqslant \dots \leqslant \tau_{n-1}^* \leqslant \tau_{n} = b$
$d = \max_k \Delta \tau_k$ ($\Delta \tau_k = \tau_{k+1} - \tau_k$)

\begin{definition}
  Случайный процесс $\xi_t$ интегрируем на $[a, b]$ с весом $\varphi(t, s)$ в средне квадратическом, 
  если
  \[
    \exists \zeta_t : M \left| \sum_{k=0}^{n-1} \varphi(t, \tau^*_k) \xi_{\tau_k^*} - \zeta_t \right|^2 \to 0, d \to 0
  \]
\end{definition}

Варианты весов:
\begin{enumerate}
  \item $\varphi(t, s) = \lambda \xrightarrow \zeta_t = \int_a^b \lambda \xi_\tau d\tau$;
  \item $\varphi(t, s) = H(t-s)$ (функция Хевисайда), тогда
    $\zeta_t = \int_a^b H(t-\tau) \xi_\tau \, d\tau = \int_a^t \xi_\tau \, d\tau$.
\end{enumerate}

Помним, что всю эту теорию мы строим для процессов с интегрируемым квадратом:
$M |\xi_t|^2 < \infty$.

\begin{theorem}
  Случайный процесс $\xi_t$ интегрируем на $[a, b]$ в средне квадратическом с весом $\varphi(t, \tau)$
  тогда и только тогда, когда
  \[
    \begin{cases}
      \exists \int_a^b \varphi(t, \tau) m_\xi(\tau) \, d\tau, \\
      \exists \int_a^b \int_a^b \varphi(t, \tau) \varphi(s, u) K_\xi(\tau, u) \, d\tau du.
    \end{cases}
  \]
\end{theorem}

\begin{corollary}
  В условиях теоремы, $\int_a^b \varphi(t, \tau) m_\xi(\tau) \, d\tau = M\zeta_t$.

  А $\int_a^b \int_a^b \varphi(t, \tau) \varphi(s, u) K_x(\tau, u) \, d\tau du = K_\zeta(t, s)$.

  В частности, если $\varphi(t, \tau) = H(t-\tau)$, то
  $M\zeta_t = M \int_a^t \xi_\tau \, d\tau = \int_a^t M\xi_\tau \, d\tau$.

  $\int_a^t \int_a^s K_\xi(\tau, u) \, d\tau du = K_\zeta(t, s)$.

  $M\xi_t M\zeta_s - M\xi_t M\zeta_s = R_{\xi \zeta} (t, s) = \int_a^s K_\xi(t, \tau) \, d\tau$
\end{corollary}
\begin{proof}
  Докажем, что $\int_a^b \varphi(t, \tau) m_\xi(\tau) \, d\tau = M\zeta_t$:
  \begin{multline*}
    0 \leqslant \left| \sum_{k=0}^{n-1} \varphi(t, \tau_k^*) M\xi_{\tau_k^*} \Delta \tau_k - M\zeta_t \right|^2
    = \left| M \left( \sum_{k=0}^{n-1} \varphi(t, \tau_k^*) \xi_{\tau_k^*} \Delta \tau_k - M\zeta_t \right) \right|^2 \leqslant  \\
    \leqslant M \left| \sum_{k=0}^{n-1} \varphi(t, \tau_k^*) \xi_{\tau_k^*} \Delta \tau_k - \zeta_t \right|^2 \to 0, d\to 0
  \end{multline*}

  Докажем ещё, что $M\xi_t M\zeta_s - M\xi_t M\zeta_s = R_{\xi \zeta} (t, s) = \int_a^s K_\xi(t, \tau) \, d\tau$:
  \begin{multline*}
    \left| \sum_{k=0}^{n-1} \varphi(s, \tau_k^*) K_\xi(t, \tau_k^*) \Delta\tau_k - R_{\xi \zeta}(t, s) \right|^2 = \\
    = \left| \sum_{k=0}^{n-1} \varphi(s, \tau_k^*) \left[ M\xi_t \xi_{\tau_k^*} - M\xi_t M\xi_{\tau_k^*} \right] \Delta\tau_k - M\xi_t\zeta_s + M\xi_t M\zeta_s \right|^2 = \\
    = \left| M \left[\xi_t \left( \sum_{k=0}^{n-1} \varphi(s, \tau_k^*) \xi_{\tau_k^*} \Delta\tau_k - \zeta_s \right) \right] - M\xi_t \left( \sum_{k=0}^{n-1} \varphi(s, \tau_k^*) M\xi_{\tau_k^*} \Delta\tau_k - M\zeta_s \right)  \right|^2 \leqslant \\
    \leqslant 2 \left| M\xi_t \left( \sum_{k=0}^{n-1} \varphi(s, \tau_k^*) \xi_{\tau_k^*} \Delta\tau_k - \zeta_s \right) \right|^2 + 2 \left| M\xi_t \left( \sum_{k=0}^{n-1} \varphi(s, \tau_k^*) M\xi_{\tau_k^*} \Delta\tau_k - M\zeta_s \right) \right|^2 \leqslant \\
    \leqslant 2 M(\xi_t)^2 \cdot M \left( \sum - \zeta_s \right)^2 + \left( M\xi_t \right)^2 \cdot
    M \left( \sum - \zeta_s \right) \to 0, d\to 0
  \end{multline*}
\end{proof}

\begin{ex}
  $W_t$ -- стандартный винеровский процесс. $MW_t = 0$ -- интегрируем на (любом) отрезке, 
  $k_W(t, s) = \min(t, s)$ -- интегрируема на (любом) прямоугольнике, поэтому согласно критерию
  интегрируемости, винеровский процесс интегрируем на любом отрезке:
  $\zeta_t = \int_0^t W_\tau \, d\tau$ \[
    M\zeta_t = \int_0^t m_W(\tau) \, d\tau = 0,
  \]
  \[
    k_\zeta(t, s) = \int_0^t \int_0^s \min(u, v) \, du dv
    = \begin{cases}
      2 \int_0^s du \int_0^u v \, dv + \int_s^t du \int_0^s v \, dv
      = \dfrac{s^3}{3} + \dfrac{s^2 (t-s)}{2}, &s < t \\
      \dfrac{t^3}{3} + \dfrac{t^2 (s-t)}{2}, &t < s.
    \end{cases}
  \]
  Тогда:
  \[
    k_\zeta(t, s) = \dfrac{1}{3} \left( \min(t, s) \right)^3 + \dfrac{1}{2} \left(\min(t, s)\right)^2 |t-s|
  \]
\end{ex}

\begin{remark}
  \begin{enumerate}
    \item Если $\xi_t$ непрерывен в средне квадратическом на $[a, b]$, 
      то он интегрируем в средне квадратическом на $[a, b]$ с любым весом.
    \item $\exists$ средне квадратическая производная $ \left( \int_0^t \xi_\tau \, d\tau \right)' = \xi_t $ $P_{п.н.}$
    \item $\int_0^t \xi_\tau' \eta_\tau \, d\tau = \xi_t \eta_t - \xi_0 \eta_0 - \int_0^t \xi_\tau \eta_\tau' \, d\tau$ $P_{п.н.}$ (при условии, что всё что надо существует).
  \end{enumerate}
\end{remark}

\begin{theorem}
  Если $\xi_t$ интегрируем в средне квадратическом на отрезке $[a, b]$ с весом $\varphi(t, s) = H(t-s)$ и является стационарным в широком смысле с корреляционной функцией $k_\xi(\tau)$,
  то:
  \[
    K_\zeta(t, s) = \int_0^t (t-\tau) k_\xi(\tau) \, d\tau + \int_0^s (s-\tau) k_\xi(\tau) \, d\tau
    - \int_0^{t-s} (t-s-\tau) k_\xi(\tau) \, d\tau,
  \]
  \[
    D\zeta_t = K_\zeta (t, t) = 2 \int_0^t (t-\tau) k_\xi(\tau) \, d\tau
  \]
\end{theorem}
\begin{proof}
  В интеграле $k_\zeta(t, s) = \int_0^t \int_0^s K_\xi(u, v) \, du dv$
  сделаем замену $u-v = \tau, u+v = w \Leftrightarrow
  u = \dfrac{\tau+\omega}{2}, v = \dfrac{\omega-\tau}{2}$, $|J| = 1/2$:
  (для определенности $s>t$)
  \[
    k_\zeta(t, s) = \iint\limits_{1+2+3} k_\xi(\tau) \dfrac{1}{2} \, d\tau d\omega
  \]
  Этот интеграл, если рисовать графически, разбивается на три.
  \begin{align*}
    \iint\limits_1 k_\xi(\tau) \dfrac{1}{2} \, d\tau d\omega &=
    \dfrac{1}{2} \int_0^t k_\xi(\tau) \, d\tau \int_\tau^{2t-\tau} d\omega = \int_0^t (t-\tau) k_\xi(\tau) \, d\tau, \\
    \iint\limits_2 k_\xi(\tau) \dfrac{1}{2} \, d\tau d\omega &=
    \dfrac{1}{2} \int_{t-s}^0 k_\xi(\tau) \, d\tau \int_{-\tau}^{2t-\tau} d\omega = \int_{t-s}^0 t k_\xi(\tau) \, d\tau, \\
    \iint\limits_3 k_\xi(\tau) \dfrac{1}{2} \, d\tau d\omega &=
    \dfrac{1}{2} \int_{-s}^{t-s} k_\xi(\tau) \, d\tau \int_{-\tau}^{2s+\tau} d\omega
    = \int_{-s}^{t-s} (s+\tau) k_\xi(\tau) \, d\tau \\
    &= \int_{-s}^0 (s+\tau) k_\xi(\tau) \, d\tau + \int_0^{t-s} (s+\tau) k_\xi(\tau) \, d\tau = \\
    &= - \int_{0}^s (s-\tau) k_\xi(\tau) \, d\tau + \int_0^{t-s} (s+\tau) k_\xi(\tau) \, d\tau
  \end{align*}
\end{proof}

\begin{ex}
  Телеграфная волна. (из Венцель-Гончаров)
  Пусть есть некий случайный процесс, называемый телеграфной волной. Этот процесс 
  может принимать значения $ \left\{ a, -a \right\} $, причём он переключается в случайные моменты
  $\tau_i$ времени, эти моменты времени распределены также, как при пуассоновском потоке событий.

  В силу симметрии, $M\xi_t = 0$. Найдём автоковариационную функцию:
  \[
    K_\xi(t, s) = M(\xi_t \cdot \xi_s)
  \]
  $\xi_t \cdot \xi_s$ -- случайная величина, принимающая значения $ \left\{ a^2, -a^2 \right\} $.
  Вероятности этих значений зависят от того, сколько событий (переключений) произошло между
  моментами времени $t$ и $s$.
  Найдём вероятность того, что количество событий чётное:
  \[
    P_\text{ч} = \sum_{k=0}^{\infty} p_{2k} = \sum_{k=0}^{\infty} \dfrac{(\lambda(s-t))^{2k}}{(2k)!} e^{-\lambda(s-t)} = \dots = \dfrac{1}{2} + \dfrac{1}{2} e^{-2\lambda (s-t)}.
  \]
  Тогда вероятность нечётного количества событий:
  \[
    P_\text{неч} = \dfrac{1}{2} - \dfrac{1}{2}e^{-2\lambda (s-t)}.
  \]
  Тогда можно уже посчитать автокорреляционную функцию:
  \[
    K_\xi(t, s) = a^2 \left( \dfrac{1}{2} + \dfrac{1}{2} e^{-2\lambda (s-t)} \right) 
    - a^2 \left( \dfrac{1}{2} - \dfrac{1}{2} e^{-2\lambda (s-t)} \right) 
    = a^2 e^{-2\lambda (s-t)}, s > t
  \]
  При $s < t$ получаем зеркальную ситуацию, поэтому:
  \[
    K_\xi(t, s) = a^2 e^{-2\lambda |s-t|}
  \]

  Автокорелляционная функция непрерывна, поэтому этот случайный процесс непрерывен.

  Проверим дифференцируемость:
  \[
    \dfrac{\partial K_\xi(t, s)}{\partial t} = \begin{cases}
      -2\lambda a^2 e^{-2\lambda(t-s)}, &t > s \\
      2\lambda a^2 e^{-2\lambda(s-t)}, &t < s
    \end{cases}
  \]
  -- разрывна на прямой $t=s$, поэтому этот процесс недифференцируем.

  Интегрируемость:
  \[
    K_\zeta(t, s) = \int_0^t (t-\tau) a^2 e^{-2\lambda \tau} \, d\tau
    + \int_0^s (s-\tau) a^2 e^{-2\lambda \tau} \, d\tau
    - \int_0^{t-s} (t-s-\tau) a^2 e^{-2\lambda\tau} \, d\tau.
  \]
  Рассмотрим, например, первый интеграл:
  \[
    \int_0^t (t-\tau) a^2 e^{-2\lambda \tau} \, d\tau
    = - \dfrac{1}{2\lambda} (t-\tau) e^{-2\lambda \tau} |_0^t - \dfrac{1}{2\lambda} \int_0^t e^{-2\lambda\tau} \, d\tau
    = \dfrac{t}{2\lambda} + \dfrac{1}{4\lambda^2} e^{-2\lambda\tau} |_0^t
    = \dfrac{t}{2\lambda} + \dfrac{1}{4\lambda^2} e^{-2\lambda t} - \dfrac{1}{4\lambda^2}
    = \dfrac{1}{4\lambda^2} \left( e^{-2\lambda t} - 1 + 2\lambda t\right) 
  \]

  Остальные аналогично, поэтому запишем ответ:
  \[
    K_\zeta(t, s) = \dfrac{a^2}{4\lambda^2} (e^{-2\lambda t} - 1 + 2\lambda t)
    + \dfrac{a^2}{4\lambda^2} (e^{-2\lambda s} - 1 + 2\lambda s)
    - \dfrac{a^2}{4\lambda^2} (e^{-2\lambda(t-s)} - 1 + 2\lambda(t-s))
    = \dfrac{a^2}{4\lambda^2} \left( e^{-2\lambda t} + e^{-2\lambda s} + 4\lambda s - 1 - e^{-2 \lambda (t-s)} \right), (t > s).
  \]
  При любых $t$ и $s$:
  \[
    K_\zeta(t, s)
    = \dfrac{a^2}{4\lambda^2} \left( e^{-2\lambda t} + e^{-2\lambda s} + 4\lambda \min(t, s) - 1 - e^{-2 \lambda (t-s)} \right).
  \]

  \[
    D\zeta_t = \dfrac{a^2}{4\lambda^2} \left( 2 e^{-2\lambda t} - 2 + 4\lambda t \right) 
  \]
\end{ex}
