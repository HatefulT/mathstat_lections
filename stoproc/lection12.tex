\section{Эргодические случайные процессы}

\begin{definition}
  Случайный процесс $\xi_t$, интегрируемый на $[0, l]$ называется
  \emph{эргодическим} относительно математического ожидания, если
  \[
    M \left| \dfrac{1}{l} \int\limits_0^l \xi_u \, du - m_\xi \right|^2 \to 0,
  \]
  или, что тоже самое,
  \[
    \dfrac{1}{l} \int\limits_0^l \xi_u \, du \toP m_\xi.
  \]
\end{definition}

% TODO дописать что-то после определения

\begin{theorem}[необходимое и достаточное условие эргодичности]
  Интегрируемый случайный процесс $\xi_t$ с постоянным математическим ожиданием $m_\xi$
  является эргодическим относительно математического ожидания тогда и только тогда,
  когда 
  \[
    \dfrac{1}{l^2} \int\limits_0^l \int\limits_0^l k_\xi(t, s) \, dt ds \to 0.
  \]
\end{theorem}
\begin{proof}
  \[
    M \dfrac{1}{l} \int\limits_0^l \xi_s \, ds = \dfrac{1}{l} \int\limits_0^l M\xi_s \, ds = m_\xi.
  \]
  % TODO дописать
\end{proof}

\begin{theorem}[достаточное условие эргодичности]
  Если интегрируемый случайный процесс $\xi_t$ имеет постоянное математическое ожидание и
  $\exists \lim_{|t-s| \to +\infty} k_\xi(t, s) = 0$, то $\xi_t$ является эргодическим относительно
  математического ожидания $m_\xi$.
\end{theorem}
\begin{proof}
  \[
    \exists \lim_{|t-s| \to +\infty} k_\xi(t, s) = 0
    \Leftrightarrow
    \forall \varepsilon > 0 \exists l_0 : |t-s|>l_0 \Rightarrow |k_\xi(t, s)| < \varepsilon.
  \]
  \[
  \left| \dfrac{1}{l^2} \int\limits_0 \int\limits_0^l k_\xi(t, s) \, dt ds \right]
  \leqslant \dfrac{1}{l^2} \iint_{D_1} |k_\xi(t, s)| \, dt ds + \dfrac{1}{l^2} \iint_{D_2} |k_\xi(t, s)| \, dt ds < \varepsilon + \dfrac{1}{l^2} C \cdot 2 l_0 l \to 0, l\to+\infty.
  \]
\end{proof}

\begin{ex}
  $k_\xi(\tau) = \cos \tau$ не стремиться к нулю, 
  \begin{multline*}
    \dfrac{1}{l^2} \int\limits_0^l \int\limits_0^l \cos(t-s) \, dt ds
    = \dfrac{1}{l^2} \int\limits_0^l dt \left(-\sin(t-s) |_0^l \right)
    = \dfrac{1}{l^2} \int\limits_0^l \left[ \sin(t) - \sin (t-l) \right] \, dt = \\
    = \dfrac{1}{l^2} \left[ - \cos t + \cos(t-l) \right] |_0^l
    = \dfrac{1}{l^2} \left( 1 - \cos l + 1 - \cos l \right) 
    = \dfrac{2 (1 - \cos l)}{l^2} \to 0, l \to + \infty
  \end{multline*}
\end{ex}

\begin{ex}
  Пуассоновский процесс $M\xi_t = \lambda t$, $D\xi_t = \lambda t$.
  Так как математическое ожидание непостоянно, рассмотрим случайный процесс
  $\mathring{\xi}_t = \xi_t - \lambda t$ -- центрированная часть пуассоновского процесса, 
  его ковариационная функция: $k_{\mathring{\xi}}(t, s) = \lambda \min(t, s)$, достаточное условие не выполнено,
  \[
    \dfrac{\lambda}{l^2} \int\limits_0^l \int\limits_0^l \min(t, s) \, dt ds
    = \dfrac{2\lambda}{l^2} \int\limits_0^l dt \int\limits_0^t s \, ds
    = \dfrac{2\lambda}{l^2} \int\limits_0^s \dfrac{t^2}{2} \, dt
    = \dfrac{\lambda l^3}{l^2} = \lambda l \to +\infty,
  \]
  то есть ни пуассоновский ни центрированная его часть не являются эргодическими процессами.
\end{ex}

\begin{ex}
  Пуассоновский поток электронов на анод. На анод падают электроны, дают ему некий заряд,
  но со временем устройство разряжается. $X_t$ -- суммарное напряжение в момент $t$:
  $X_t = \sum_{k=1}^{Y} X_k(t) = \sum_{k=1}^{Y} e^{-\alpha (t-\tau_k)} I(t > \tau_k)$, $Y$ --
  число событий в потоке на $[0, t]$.

  Через характеристические функции можно доказать, что $M \sum_{k=0}^\tau \xi_k = M\tau \cdot M\xi_k$, $D \sum_{k=1}^\tau \xi_k = M\tau \cdot D \xi_k + D\tau \cdot \left( M\xi_k \right)^2$.

  % Так как для малых отрезков времени $[t_1, t_2]$ верно, что на отрезке $[]$
  \[
    MX_t = MY \cdot M e^{-\alpha(t-\tau_k)} I(t > \tau_k) = \lambda t \cdot \int\limits_0^t \dfrac{1}{t} e^{-\alpha(t-x)} I(t > x) \, dx
    = \dfrac{\lambda}{\alpha} t \cdot \dfrac{1}{t} e^{-\alpha (t-x)} |_0^t
    = \dfrac{\lambda}{\alpha} \left( 1 - e^{-\alpha t} \right) \to \dfrac{\lambda}{\alpha}.
  \]
  Приближенно можно считать, что математическое ожидание постоянно.
  Ковариационная функция: $K_x (t_1, t_2) = \cov (X_{t_1}, X_{t_2})$,
  несложно получить следующее представление для $X_{t_2}$ при $t_2 > t_1$:
  $X_{t_2} = X_{t_1} e^{-\alpha(t - t_1)} + U(t_1, t_2)$, 
  причём слагаемые независимы. Поэтому,
  \[
    K_X(t_1, t_2) = e^{-\alpha |t_2-t_1|} DX_{\min(t_1, t_2)}.
  \]

  Посчитаем дисперсию:
  \[
    DX_t = MY \cdot DX_k(t) + DY \cdot (MX_k(t))^2
    = \lambda t M \left( X_k(t) \right)^2
    = \lambda t \cdot \dfrac{1}{t} \int\limits_0^t e^{-2\alpha(t-x)} \, dx
    = \dfrac{\lambda}{2\alpha} \left( 1 - e^{-2\alpha t} \right) 
  \]

  Собирая всё вместе:
  \[
    K_X(t_1, t_2) = \dfrac{\lambda}{2\alpha} \left(1 - e^{-2\alpha \min(t_1, t_2)}\right)
    e^{-\alpha |t_2-t_1|}
  \]
  это скобка ограничена, второй множитель стремиться к нулю при $t_2 - t_1 \to +\infty$,
  поэтому выполнено достаточное условие эргодичности.
\end{ex}

\subsection{Белый шум}

\begin{ex}
  Случайная телеграфная волна. Есть последовательность независимых случайных величин
  $\xi_0, \xi_1, \xi_2, \dots, M\xi_i = m_\xi, D\xi_i = D\xi$.
  $ \left\{ \tau_k \right\}  $ -- пуассоновский поток интенсивности $\lambda$.
  В моменты времени $t = \tau_k$ случайная величина $\xi_t$ меняет своё значение на $\xi_k$ до следующего $\tau_{k+1}$.

  $M\xi_t = M\xi_i = m_\xi$, обозначим $\mathring{\xi}_t = \xi_t - m_\xi$,
  $t_1 < t_2 : k_\xi(t_1, t_2) = \cov (\xi_{t_1}, \xi_{t_2})
  = M \mathring{\xi}_{t_1} \cdot \mathring{\xi}_{t_2} (I_{H_0} + I_{H_1}) = $,
  где $H_0$ -- событие,
  заключающееся в том, что на $(t_1, t_2)$ -- не случилось ни одного события, $H_1$ -- случилось 
  хотя бы одно.
  $ = D\xi_i \cdot P(H_0)$,
  тогда $K_\xi(t, s) = D_\xi \cdot e^{-\lambda |t-s|}$.

  Предельный случай $\lambda \to +\infty, D_\xi \to +\infty$, причём
  $\dfrac{D_\xi}{\lambda \pi} \to C$: 
  $k_\xi(\tau) = D_\xi e^{-\lambda |\tau|} = \dfrac{D_\xi}{\lambda \pi} 2\pi \cdot \underbrace{\dfrac{\lambda}{2} e^{-\lambda|\tau|}}_{\text{двусторонний закон Лапласа}} = C 2\pi \delta(\tau)$,
  $s_\xi(\omega) = C$.
\end{ex}

\begin{definition}
  Центрированный случайный процесс $\xi_t$ называется \emph{белым шумом интенсивности $C$}, если
  $s_\xi(\omega) = C$. 

  Если $C = 1$, то он называется \emph{стандартным белым шумом}.
\end{definition}

\paragraph{Свойства белого шума.}
\begin{enumerate}
  \item Сечения $\xi_{t_1}$ и $\xi_{t_2}$ некоррелированны при $t_1 \neq t_2$;
    \begin{proof}
      $\cov(\xi_{t_1}, \xi_{t_2}) = K_\xi(t_1, t_2) = C\cdot 2\pi \cdot \delta(t_2 - t_1) = 0$.
    \end{proof}
  \item Дисперсия белого шума $D\xi_t = +\infty$;
    \begin{proof}
      $D\xi_t = k_\xi(t, t) = 2\pi C \delta(0) = \infty$.
    \end{proof}
  \item $\xi_t$ можно рассматривать как обобщённую производную винеровского процесса с 
    $\sigma = \sqrt{2\pi C}$.
    \begin{proof}
      $K_W (t, s) = \sigma^2 \min(t, s)$, 
      \[
        \dfrac{\partial }{\partial t} k_\xi(t, s) = \begin{cases}
          0, &t > s \\
          \sigma^2 = 2\pi C, & t < s
        \end{cases}
      \]
      \[
        \dfrac{\partial^2 }{\partial t \partial s} k_\xi(t, s) = 2\pi C \delta(t-s).
      \]
    \end{proof}
\end{enumerate}

\subsection{Понятие о стохастических дифференциальных уравнений}

Рассмотрим уравнение: $\xi_t' = -a \xi_t + \varepsilon_t, \xi_0 = \nu$, где $\varepsilon_t$ -- случайный шум.
Об этом шуме можно выдвинуть естественные предположения:
\begin{enumerate}
  \item $M\varepsilon_t = 0$;
  \item $t \neq s : M\varepsilon_t \varepsilon_s = 0$;
  \item $D\varepsilon_t < \infty$;
  \item $\nu$ и $\xi_t$ независимы.
\end{enumerate}
оказывается, что данные предположения ничего не дают.
Решение этого уравнения представляется в виде свёртки фундаментального решения с правой частью:
\[
  \xi_t = \nu e^{-a t} + \int\limits_0^t e^{-a(t-\tau)} \varepsilon_\tau \, d\tau.
\]
Обозначим свёртку за $\gamma_t = \int\limits_0^t e^{-a(t-\tau)} \varepsilon_\tau \, d\tau$:
\[
  k_1 (t, s) = \cov \left( e^{-a(t-\tau)} \varepsilon_\tau, e^{-a(s-} \right) 
\]
\begin{align*}
  M\gamma_t &= M\int\limits_0^t e^{-a(t-\tau)} \varepsilon_\tau \, d\tau = 0, \\
  k_\gamma(t, s) &= \cov( \dots ) = 0
\end{align*}

То есть получается, что такая система просто не замечает такого рода шум. То есть условия, наложенные
на шум, надо изменить. Отказались от условия $D\varepsilon_t < \infty$.
То есть $\varepsilon_t$ -- белый шум.

Формальная запись:
$d\xi_t = - a\xi_t dt + \varepsilon_t dt, d\xi_t = -a \xi_t dt + dW_t$. То есть если нужен шум,
то его надо моделировать винеровским процессом. Больший смысл имеет выражение $\xi_t = - a \int\limits_0^t \xi_s \, ds + W_t$.
